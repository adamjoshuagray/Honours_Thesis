\documentclass{unswmaths}
\usepackage[a4paper]{geometry}
\usepackage{fancyhdr}
\pagestyle{fancy}
\begin{document}

\setlength\parindent{0pt}
\setlength{\parskip}{5mm plus4mm minus3mm}

\unswtitle{Adam J. Gray}{3329798}{Thesis Introduction}{Fractional Differential Equations}
\fancyfoot[l]{Adam J. Gray}
\fancyfoot[r]{\today}
\fancyhead[l]{The University of New South Wales}
\fancyhead[r]{Fractional Differential Equations}

\section{Introduction}
\subsection{Historical Overview and Motivation}
\subsubsection{L'Hopital \& Leibniz}
Fractional calculus is almost as old as calculus itself. The 30th September 1695 is often cited as the the birthday of the field,
as it is the date on which Leibniz, in reply to L'Hopital's question about the operator $ \frac{d^\frac{1}{2}}{dx^\frac{1}{2}} $, 
wrote, ``It will lead to a paradox, from which one day useful consequences will be drawn.''
\subsubsection{Leibniz, Wallis \& Bernoulli}
Leibniz, in letters addressed to John Wallis and Daniel Bernoulli in 1697, proposed a 
formulation for the fractional derivitive of an exponential function.
He proposed that 
$$
    \frac{d^r}{dx^r} e^{mx} = m^r e^{mx}.
$$
Keeping in mind that this is was the late 17th century and so Fourier had yet to 
be born, let alone develop the idea of a Fourier decomposition, so there was
no ``obvious'' way to extend this definition to other functions.
\subsubsection{Euler}
A crucial function to almost all formulations of fractional calculus, is the gamma function, which extends the factorial function
to non-integer arguments. Although the problem of extending the factorial function had been considered by Daniel Bernoulli and Christian Goldbach in the 1720's, it was eventually Euler who in a two letters, dated 13th October 1729 
and 8th January 1730 respectively, gave two different representations of the factorial which could easily be extended
to non-integer values.
They were
\begin{align*}
	n! &= \prod_{k=1}^\infty \frac{\left( 1 + \frac{1}{k} \right)^n}{1 + \frac{n}{k}} \\
	\text{ and } \\
	n! &= \int_0^1 (-\ln s)^n ds 
\end{align*}
Euler made swift use of the gamma function by generalizing the derivitive of the power function.
Euler noticed that 
$$
	\frac{d^n}{dx^n} x^m = \frac{m!}{n+1!} x^{m - n}
$$
provided that $ n \leq m + 1 $.
The obvious extension was to take the factorials and replace them with gamma functions to get
$$
    \frac{d^r}{dx^r} x^m = \frac{\Gamma (m+1)}{\Gamma (m - r + 1)} x^{m-r}.
$$
Beutiful results like 
$$
    \frac{d^\frac{1}{2}}{dx^\frac{1}{2}} x = \sqrt{\frac{4x}{\pi}}
$$
became immediate. 
A useful point to note here is that my taking $ r = -1 $ we have
\begin{align*}
    \frac{d^{-1}}{dx^{-1}} x^m  &= \frac{\Gamma(m+1)}{\Gamma(m + 2)} x^{m + 1} \\
                                &= \frac{m!}{(m+1)!}x^{m+1} \\
                                &= \frac{1}{m+1} x^{m+1} \\
                                &= \int_0^x t^m dt
\end{align*}
which is consistent with the fundamental theorem of calculus.

Tangentially it is worth noting that Taylor series were formally introduced
by English mathematician Brook Taylor in 1715. Although it did not happen,
an enterpising mathematician, may have seen that Euler's definition could be
extended at least in some formal sense by using Taylor expansions.
Let us consider $ e^{mx} $ and write
$$
    e^{mx} = \sum_{k = 0}^\infty \frac{m^k}{k!} x^k.
$$
It would be natural to write \footnote{By $ \sum_{k = r}^\infty $ we mean a sum starting at $ k = r $ and adding in increments of $ 1 $. }
\begin{align}
    \label{eq:Euler_Leibniz_Sum}
    \frac{d^r}{dx^r} e^{mx} &= \sum_{k = r}^\infty \frac{d^r}{dx^r} \frac{m^k}{k!} x^k \\
                            &= \sum_{k = r}^\infty \frac{m^k}{\Gamma(k+1)} \frac{\Gamma(k+1)}{\Gamma(k - r + 1)} x^{k-r} \nonumber \\             
                            &= \sum_{k = r}^\infty \frac{m^k}{\Gamma(k - r + 1)}x^{k-r} \nonumber \\
                            &= m^r \sum_{k = r}^\infty \frac{m^{k-r}}{\Gamma(k - r + 1)}x^{k-r}. \nonumber
\end{align}
Letting $ j = k - r $ we have
\begin{align*}
    m^r \sum_{k = r}^\infty \frac{m^{k-r}}{\Gamma(k - r + 1)}x^{k-r}
        &= m^r \sum_{j = 0}^\infty \frac{m^{j}}{\Gamma(j + 1)}x^{j} \\
        &= m^r \sum_{j = 0}^\infty \frac{m^{j}}{j!}x^{j} \\
        &= m^r e^{mx}
\end{align*}
which suggests that in some formal sense Euler's derivitive is consistent with Leibniz's.

Regardless of the potential utility  of this observation, there is no historical evidence that the author
can find that suggests such observations were ever made.

Euler's definition did gain traction, however, and it was published in S. F. Lacroix's  book
\emph{Trait\'{e} du Calcul Diff\'{e}rentiel et du Calcul Int\'{e}gral.}
\subsubsection{Fourier}
It appears that Joseph Fourier was the first to suggest any kind of \emph{general}
fractional differential operator for a broader class of functions. He did this by noting
that for \emph{sufficiently nice} functions,
\begin{align*}
    f(x) = \frac{1}{2\pi} \int_{-\infty}^\infty f(z) \int_{-\infty}^{\infty} \cos(ux - uz)dudz
\end{align*}
so for an arbitrary $ r $ we could have that
\begin{align*}
    \frac{d^r}{dx^r}f(x) = \frac{1}{2\pi} \int_{-\infty}^\infty f(z)\int_{-\infty}^\infty \cos(ux - uz + r\frac{\pi}{2})dudz
\end{align*}
\subsubsection{Laplace}
\subsubsection{Abel \& Liouville}
\subsubsection{Riemann}
\subsubsection{Grunwald \& Letnikov}
\subsubsection{Mittag-Laffler}
\subsubsection{Caputo}
\end{document}
