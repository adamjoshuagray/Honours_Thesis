\documentclass[pdf]{beamer}

\usepackage{mathtools}

\newcommand{\laplace}[1]{ \mathcal{L} \left\{ #1 \right\} }
\newcommand{\fourier}[1]{ \mathcal{F} \left\{ #1 \right\} }
\newcommand{\mellin}[1]{ \mathcal{M} \left\{ #1 \right\} }
\newcommand{\rld}[4]{ \left( \prescript{}{#1}{\mathcal{D}_{#2}^{#3}} #4 \right) }
\newcommand{\rli}[3]{ \left( I_{#1}^{#2} #3 \right) }
\newcommand{\der}[3]{ \frac{d^{#3}#1}{d#2^{#3}} }
\newcommand{\capder}[4]{ \left( \prescript{C}{#1}{\mathcal{D}_{#2}^{#3}} #4 \right) }
\newcommand{\fracdelta}[4]{ \left( \prescript{}{#1}{\Delta^{#2}_{#3} } #4 \right) }

\newcommand{\lra}{\longrightarrow}
\newcommand{\ra}{\rightarrow}
\newcommand{\lla}{\longleftarrow}
\newcommand{\la}{\leftarrow}

%Analysis
\newcommand{\Rl}{\mathbb{R}}
\newcommand{\Cplx}{\mathbb{C}}
\newcommand{\Itgr}{\mathbb{Z}}
\newcommand{\Ntrl}{\mathbb{N}}
\newcommand{\Ind}{\mathbbm{1}}
\newcommand{\Hlbt}{\mathcal{H}}
\newcommand{\im}{\operatorname{im}}

\mode<presentation>{}
\title{Accelerating Numerical Solutions to Fractional Differential Equations}
\subtitle{Adams Moulton Bashforth Method}
\author[Adam J. Gray]{Adam J. Gray\\{\small Supervised by: Dr Chris Tisdell}}
\institute{
	School of Mathematics and Statistics \\
	University of New South Wales
}

\begin{document}

\begin{frame}
	\titlepage
\end{frame}


\begin{frame}{Problem}
    \begin{align*}
        \capder{0}{x}{\alpha}{y}(x) = f(x,y)
    \end{align*}
    
    \begin{align*}
        y^{(k)}(0) = a_{k} & & 0 \leq \alpha \leq \lceil k \rceil
    \end{align*}
\end{frame}

\begin{frame}{Similar Problems}
    \begin{align*}
        \frac{dy}{dx} = f(x,y)
    \end{align*}
    
    \begin{align*}
        y(0) = y_0
    \end{align*}
\end{frame}

\begin{frame}{Explanation of (Explicit) Euler Method}
    The Euler method is designed to solve problems of the form
    \begin{align*}
        y'(x) = f(x,y(x)) & & y(x_0) = y_0.
    \end{align*}
    
    From first principles
    \begin{align*}
        y'(x_0) = \lim_{h \lra 0}\frac{f(x_0 + h) - f(x_0)}{h}
    \end{align*}
    therefore
    \begin{align*}
        \lim_{h \lra 0}\frac{f(x_0 + h) - f(x_0)}{h} = f(x_0, y(x_0))
    \end{align*}
    and by rearranging and choosing some small but non-zero h,
    \begin{align*}
        y(x_0 + h) = hf(x_0, y(x_0)) + y(x_0).
    \end{align*}
\end{frame}

\begin{frame}{Explanation of (Explicit) Euler Method (cont)}
    Restating this iteratively we get
    \begin{align*}
        y_{n+1} = y_{n} + hf(x_n,y_n)
    \end{align*}
\end{frame}

\begin{frame}{Computational Complexity}
    \begin{tabular}{|c|c|}
        \hline
        Scheme & Complexity \\
        \hline
        Euler & $O(n)$ \\
        Runge-Kutta &$ O(n)$ \\
        Adams Moulton &$ O(n)$ \\
        Adams Bashforth &$ O(n)$ \\
        etc & O(n) \\
        \hline
    \end{tabular}
\end{frame}

\begin{frame}{Truncation Errors}
    \textbf{Local Truncation Error}
    
    This is the amount of error built up in each step.
    From a Taylor series expansion we have,
    \begin{align*}
        y(x_0 + h) = y(x_0) + hy'(x_0) + \frac{1}{2}h^2y''(x_0) + O(h^3)
    \end{align*}
    and thus
    \begin{align*}
        e_{local} = \frac{1}{2}h^2y''(x_0) + O(h^3)
    \end{align*}
    so long as $ y $ has bounded third derivative.
\end{frame}

\begin{frame}{Truncation Errors}
    
    \textbf{Global Truncation Error}
    
    The number of steps is 
    \begin{align*}
        n = \frac{x - x_0}{h}
    \end{align*}
    and the local truncation error is $ O(h^2) $ so we can expect
    the global truncation error to be $ O(h) $. 
    
    This result holds for \emph{sufficiently nice} $ f $ and $ y $.
    
\end{frame}

\begin{frame}{Computational Complexity and Truncation Errors}
    \begin{tabular}{|c|c|c|}
        \hline
        Scheme & Complexity & Global Truncation Error \\
        \hline
        Euler & $O(n)$ & $ O(h) $ \\
        Runge-Kutta (RK4) & $O(n)$ & $O(h^4)$ \\
        \hline
    \end{tabular}
\end{frame}

\begin{frame}{The Fractional Case}
    
\end{frame}



\end{document}
