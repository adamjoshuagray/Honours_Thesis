
\documentclass[pdf]{beamer}

\usepackage{amsmath}
\usepackage{amssymb}
\usepackage{amsthm}
\usepackage{mathtools}

\usetheme{Dresden}
\usecolortheme{beaver}

%Analysis
\newcommand{\Rl}{\mathbb{R}}
\newcommand{\Cplx}{\mathbb{C}}
\newcommand{\Itgr}{\mathbb{Z}}
\newcommand{\Ntrl}{\mathbb{N}}
\newcommand{\Ind}{\mathbbm{1}}
\newcommand{\Hlbt}{\mathcal{H}}
\newcommand{\im}{\operatorname{im}}

%Algebra
\newcommand{\Grp}{\mathcal{G}}

%Misc
\newcommand{\lra}{\longrightarrow}
\newcommand{\ra}{\rightarrow}
\newcommand{\lla}{\longleftarrow}
\newcommand{\la}{\leftarrow}

%Stats \ Prob
\newcommand{\E}[1]{\mathbb{E} \left[ #1 \right]}
\newcommand{\Var}[1]{\operatorname{Var} \left[ #1 \right] }
\newcommand{\Cov}[2]{\operatorname{Cov} \left[ #1, #2 \right] }
\newcommand{\Filt}{\mathcal{F}}

%Fractional Differential Equations

\newcommand{\laplace}[1]{ \mathcal{L} \left\{ #1 \right\} }
\newcommand{\fourier}[1]{ \mathcal{F} \left\{ #1 \right\} }
\newcommand{\mellin}[1]{ \mathcal{M} \left\{ #1 \right\} }
\newcommand{\rld}[3]{ \left( \mathcal{D}_{#1}^{#2} #3 \right) }
\newcommand{\rli}[3]{ \left( I_{#1}^{#2} #3 \right) }
\newcommand{\der}[3]{ \frac{d^{#3}#1}{d#2^{#3}} }
\newcommand{\capder}[3]{ \left( \prescript{C}{}{\mathcal{D}_{#1}^{#2}} #3 \right) }

\mode<presentation>{}
\title{A Solution to a Fractional Differential Equation}
\subtitle{The Laplace Transform Method}
\author{Adam J. Gray}
\institute{
	School of Mathematics and Statistics \\
	University of New South Wales
}

\begin{document}

\begin{frame}
	\titlepage
\end{frame}

\begin{frame}{The Goal}
	We aim to get a solution to the following fractional differential equation (in terms of Caputo derivatives)
	\begin{align}
		\label{eq:fde-1}
		\left( \prescript{C}{}{\mathcal{D}_0^\alpha}y \right)(t) = \beta y(t) 
	\end{align}

	along with the initial conditions 
	\begin{align}
		\label{eq:fde-1-ic}
		y^{(k)}(0) = 
		\begin{cases}
			1 & k = 0 \\
			0 & 1 \leq k \leq \lfloor \alpha \rfloor - 1  
		\end{cases}
	\end{align}
\end{frame}


\begin{frame}{Motivations}
	\begin{block}{Cauchy Formula for Repeated Integration}
		\begin{align*}
			\int_{a}^{x} \int_{a}^{y_1} \cdots \int_a^{y_{n-1}} f(y_n) dy_n \cdots dy_2 dy_1 = \frac{1}{(n-1)!} \int_a^x(x-t)^{n-1}f(t)dt
		\end{align*}
	\end{block}
	\pause
	The idea is to replace the factorials with gamma functions to define an integral of arbitrary order
	\pause
	\begin{block}{Riemann-Liouville Fractional Integral}
		\begin{align*}
			\rli{a}{\alpha}{f}(x) = \frac{1}{\Gamma(\alpha)} \int_a^x(x-t)^{\alpha-1}f(t)dt
		\end{align*}
	\end{block}
\end{frame}

\begin{frame}{Motivations (Derivatives)}
	\begin{block}{Riemann-Liouville Fractional Derivative}
		\begin{align*}
			\rld{a}{\alpha}{f}(x) &= \frac{d^{\lceil \alpha \rceil}}{dx^{\lceil \alpha \rceil}} \rli{a}{\lceil \alpha \rceil - \alpha}{f}(x) \\
				&= \frac{1}{\Gamma(1 - \alpha)}\frac{d^{n}}{dx^n} \int_a^x \frac{f(t)dt}{\frac{(x-t)^{\alpha - n + 1}}{}}
		\end{align*}
		where $ n - 1 < \alpha \leq n $
	\end{block}
\end{frame}

\begin{frame}{Motivations (Derivatives)}
	\begin{block}{Caputo Fractional Derivative}
		\begin{align*}
			\capder{a}{\alpha}{f}(x) &= \rli{a}{\lceil \alpha \rceil - \alpha}{\frac{d^{\lceil \alpha \rceil}}{dx^{\lceil \alpha \rceil}}f}(x) \\
				&= \frac{1}{\Gamma(1 - \alpha)} \int_a^x \frac{\frac{d^{t}}{dt^n}f(t)dt}{(x-t)^{\alpha - n + 1}}
		\end{align*}
		where $ n - 1 < \alpha \leq n $
	\end{block}
\end{frame}

\begin{frame}{ Riemann-Liouville vs Caputo Derivative}
	\begin{alertblock}{Note!}
		The Caputo derivative and the and the Riemann-Liouville derivatives are note the same.
		In general 
		\begin{align*}
			\capder{a}{\alpha}{f}(x) \neq \rld{a}{\alpha}{f}(x).
		\end{align*}
	\end{alertblock}
	The reason is exatly the same reason that in general
	\begin{align*}
		f(x) \neq \int_a^x f'(t) dt.
	\end{align*}
	In some sense if you differentiate first you "loose information" about the function.
\end{frame}

\begin{frame}{ Riemann-Liouville vs Caputo Derivative}

	The Caputo derivative is often used in fractionial differential equations because it
	can be coupled with integer order initial conditions, whereas often the Riemann-Liouville
	derivative can't be coupled with integer order intial conditions.

\end{frame}
	


\end{document}