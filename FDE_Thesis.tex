\documentclass{unswmaths}
\usepackage{mathtools}
\usepackage{unswshortcuts}

\author{Adam J. Gray}
\studentno{3329798}
\subject{Fractional Differential Equations}
\title{Honours Thesis}
\supervisor{Dr Chris Tisdell}


\begin{document}

\unswtitle

\setlength\parindent{0pt}
\setlength{\parskip}{5mm plus4mm minus3mm}

\section*{Abel's Integral Equation}

We wish to consider a simple integral equation of the form
\begin{align}
	\label{eqn:abel}
	\frac{1}{\Gamma(\alpha)}\int_a^x \frac{\phi(t)dt}{(x-t)^\alpha} = f(x) 	& & x \geq 0, 0 \leq \alpha \leq 1
\end{align}
We call this integral equation an Abel integral equation. It is worth noting that there are many forms of Abel's integral
equation and we are just considering one form here.

We wish to layout a simple method for solving Abel's integral equation.

Firstly let's consider the integral 
\begin{equation}
	\label{eqn:abelint}
	I(x) := \int_a^x \frac{f(s)ds}{(x-s)^{1-\alpha}}.
\end{equation}
Now by substituting \eqref{eqn:abel} into \eqref{eqn:abelint} we get 
\begin{align*}
	I(x) &= \frac{1}{\Gamma(\alpha)} \int_a^x \frac{1}{(x-s)^{1-\alpha}} \left( \int_a^s \frac{\phi(t)dt}{(s-t)^\alpha} \right) ds \\
		&= \frac{1}{\Gamma(\alpha)} \int_a^x \left( \int_a^s \frac{\phi(t)dt}{(x-s)^{1-\alpha}(s-t)^\alpha} \right) ds
\end{align*}
Now noting that the region of integration in $ \mathbb{R}^2 $ is just
\begin{align*}
	a &\leq s \leq x \\
	a &\leq t \leq s 
\end{align*}
which is equivalent to 
\begin{align*}
	t &\leq s \leq x \\
	a &\leq t \leq x 
\end{align*}
we can write 
\begin{align}
	\frac{1}{\Gamma(\alpha)} \int_a^x \left( \int_a^s \frac{\phi(t)dt}{(x-s)^{1-\alpha}(s-t)^\alpha} \right) ds 
		&= \frac{1}{\Gamma(\alpha)} \int_a^x \left( \int_t^x \frac{\phi(t)ds}{(x-s)^{1-\alpha}(s-t)^\alpha} \right) dt \nonumber \\
		\label{eqn:prebeta}
		&= \frac{1}{\Gamma(\alpha)} \int_a^x \phi(t) \left( \int_t^x (x-s)^{\alpha-1}(s-t)^{-\alpha} ds\right) dt. 
\end{align}
Now performing the substitution $ \tau = \frac{s-t}{x-t} $ yields 
\begin{align*}
	\int_t^x (x-s)^{\alpha-1}(s-t)^{-\alpha} ds &= \int_0^1 \tau^{-\alpha} (1-\tau)^{\alpha - 1} d\tau \\
		&= B(1-\alpha,\alpha) \\
		&= \Gamma(1-\alpha)\Gamma(\alpha)
\end{align*}
and so \eqref{eqn:prebeta} becomes
\begin{align*}
	\frac{1}{\Gamma(\alpha)} \int_a^x \phi(t) \left( \int_t^x (x-s)^{\alpha-1}(s-t)^{-\alpha} ds\right) dt
		&= \frac{1}{\Gamma(\alpha)} \int_a^x \phi(t) \Gamma(\alpha)\Gamma(1-\alpha) dt \\
		&= \Gamma(1-\alpha)\int_a^x \phi(t) dt. 
\end{align*}
So we have that 
\begin{align*}
	\int_a^x \frac{f(s)ds}{(x-s)^{1-\alpha}} &= \Gamma(1-\alpha)\int_a^x \phi(t) dt
\end{align*}
and by differentiating we get
\begin{align*}
	\phi(x) &= \frac{1}{\Gamma(1-\alpha)} \frac{d}{dx} \int_a^x \frac{f(s)ds}{(x-s)^{1-\alpha}}
\end{align*}
\nocite{Samko1993}
\qed


\section*{Solution to a Simple Fractional Differential Equation}

We aim to get a solution to the following fractional differential equation (in terms of Caputo derivatives)

\begin{align}
	\label{eq:fde-1}
	\left( \prescript{C}{}{\mathcal{D}_0^\alpha}y \right)(t) = \beta y(t) 
\end{align}

along with the initial conditions 
\begin{align}
	\label{eq:fde-1-ic}
	y^{(k)}(0) = 
	\begin{cases}
		1 & k = 0 \\
		0 & 1 \leq k \leq \lfloor\alpha \rfloor - 1  
	\end{cases}
\end{align}

has the solution $ y(t) = E_\alpha \left( \beta t^\alpha \right) $. Where $ E_\alpha $ is the one parameter Mittag-Lefler function.

This solution can be arrived at by a Laplace transform method. For completeness we define the following fractional
integrals and derivatives.

\begin{definition}[Fractional Derivatives and Integrals]
	For $ \alpha > 0 $ we define
	\begin{align*}
		(I_{a+}^{\alpha}f)(x) := \frac{1}{\Gamma(\alpha)}\int_a^x \frac{f(t)}{(x-t)^{1 - \alpha}} dt \\
		(\mathcal{D}_{a+}^{\alpha}f)(x) := \frac{1}{\Gamma(n-\alpha)} \frac{d^n}{dx^n}\int_a^x \frac{f(t)}{(x-t)^{\alpha - n + 1}} dt \\
		(\prescript{C}{}{\mathcal{D}}_{a+}^\alpha f)(x) := I_{0}^{n-\alpha} \frac{d^n}{dx^n}f(x) 
	\end{align*}
	where $ n  = \lfloor \alpha \rfloor + 1$.
	We will refer to $ I_{a+}^\alpha f$ as the (Riemann Louiville) integral $ f $ of over $ \alpha $ (based at $ a $).
	Likewise we refer to $ \mathcal{D}_{a+}^\alpha f $ as the (Riemann Louiville) derivative of order $ \alpha $ (based at $ a $).
	We also refer to $ \prescript{C}{}{\mathcal{D}}_{a+}^\alpha f $ as the Caputo derivative of order $ \alpha $ (based at $ a $).
	
\end{definition}

The motivation for these definitions are based off the Cauchy formula for repeated integration, and in the case
of the Caputo derivative, practical considerations. \cite{Samko1993, Podlubny1999} 

For the rest of our considerations in this section we will take $ a = 0 $ (based at 0). 

We now consider the Laplace transform of the fractional integration and differentiation operators.

\begin{lemma}
\label{lem:lap_rli}
	The Laplace transform of the Riemann-Liouville integral of a fuction $ f $ is as follows
	$$
		\mathcal{L} \left\{ I_0^\alpha f \right\}  = s^{-\alpha} \mathcal{L} \left\{ f \right\}.
	$$
\end{lemma}
\begin{proof}
	Since 
	$$
		 (I_0^\alpha f)(t) = \frac{1}{\Gamma(\alpha)} \int_0^x f(u) (t-u)^{\alpha - 1} du
	$$
	is just $ \frac{1}{\Gamma(\alpha)} $ times the convolution of $ f $ with $ t^{\alpha - 1} $ then by the convolution theorem
	for Laplace transforms we have that 
	\begin{align*}
		\mathcal{L} \left\{ I_0^\alpha f \right\} &= \frac{1}{\Gamma(\alpha)} \mathcal{L} \left\{ \int_{0}^{t} f(u) (t-u)^{\alpha - 1} du \right\} \\
			&= \frac{1}{\Gamma(\alpha)} \mathcal{L} \left\{ f(t) \right\} \underbrace{\mathcal{L} \left\{ t^{\alpha - 1} \right\}}_{=s^{-\alpha} \Gamma(\alpha)} \\
			&= s^{-\alpha} \mathcal{L} \left\{ f \right\}.
	\end{align*}
\end{proof}

\begin{lemma}
\label{lem:lap_rld}
	The Laplace transform the of the Riemann-Liouville derivative of a function $ f $ is as follows
	\begin{align*}
		\mathcal{L} \left\{\mathcal{D}_0^\alpha f\right\} = s^\alpha \mathcal{L} \left\{ f \right\} - \sum_{k=0}^{n-1} s^{k} \left( \mathcal{D}_0^{\alpha-k-1} f\right)(0).
	\end{align*}
\end{lemma}
\begin{proof}
	See that
	\begin{align*}
		\laplace{ \rld{0}{\alpha}{f} } &= \laplace{ \der{}{t}{n} \rli{0}{n-\alpha}{f} } \\
			&= s^n\laplace{\rli{0}{n-\alpha}{f}} - \sum_{k=0}^{n-1} s^k \der{}{t}{n-k-1} \rli{0}{n-\alpha}{f}(0)
	\end{align*}
	and by applying the result of \ref{lem:lap_rli} we get
	\begin{align*}
			\mathcal{L} \left\{\mathcal{D}_0^\alpha f\right\} &= s^\alpha \laplace{f} - \sum_{k=0}^{n-1} s^{k} \rld{0}{\alpha - k - 1}{f}(0). 
	\end{align*}
\end{proof}

\begin{lemma}
\label{lem:lap_cap}
	The Laplace transform of the Caputo derivative of  a function $ f $ is given as follows
	\begin{align*}
		\laplace{\capder{0}{\alpha}{f}} = s^{\alpha - n} \left[ s^n \laplace{f} - \sum_{k=0}^{n-1} s^{n-k-1} \left( \der{f}{t}{k} \right)(0) \right].
	\end{align*}
\end{lemma}
\begin{proof}
	See that
	\begin{align*}
		\laplace{\capder{0}{\alpha}{f}} &= \laplace{  \rli{0}{n-\alpha}{\der{f}{t}{n}}} \\
			&= \frac{1}{\Gamma(n-\alpha)}\laplace{ \int_0^t (t-u)^{n-\alpha-1} \der{f}{t}{n} du} \\ 
	\end{align*}
	which is the Laplace transform of a convolution so
	\begin{align*}
		\Gamma(n-\alpha)\laplace{ \int_0^t (t-u)^{n-\alpha-1} \der{f}{t}{n} du} &= \laplace{t^{n-\alpha-1}} \laplace{\der{f}{t}{n}} \\
		&= \frac{1}{n-\alpha} \left( s^{-(n-\alpha)} \Gamma(n-\alpha) \right) \\
		& \ \ \ \times \left( s^n \laplace{f} - \sum_{k=0}^{n-1} s^{n-k-1} \left( \der{f}{t}{k} \right)(0) \right) \\
		&= s^{\alpha - n} \left[ s^n \laplace{f} - \sum_{k=0}^{n-1} s^{n-k-1} \left( \der{f}{t}{k} \right)(0) \right].
	\end{align*}	
\end{proof}

We now define the Mittag-Lefler function and calculate its Laplace transform.

\begin{definition}
	The one parameter Mittag-Lefler $ E_\alpha $ function is defined by its power series.
	$$
		E_\alpha(t) = \sum_{k=0}^{\infty} \frac{t^k}{\Gamma(\alpha k + 1)}
	$$
\end{definition}
It is clear to see the definition of this function is inspired by the exponential function. Before we can calculate the 
Laplace transform of the Mittag-Lefler function we have to prove a simple lemma about the convergence of the 
series which is used in its definition.

\begin{lemma}
\label{lem:mit_conv}

	The series
	$$
		\sum_{k=0}^{\infty} \frac{t^k}{\Gamma(\alpha k + 1)} 
	$$
  	converges absolutely for all $ t \in \mathbb{R} $.
\end{lemma}
\begin{proof}
	Let $ a_k = \frac{t^k}{\Gamma(\alpha k + 1) }$ and see that
	$$ \lvert \frac{a_{k+1}}{a_k} \rvert = |t| \frac{\Gamma(\alpha k + 1) }{\Gamma(\alpha(k+1) + 1)} $$
	and that hence 
	$$
		\lim_{k \longrightarrow \infty} \lvert \frac{a_{k+1}}{a_k} \rvert = 0
	$$
	for all $ t \in \mathbb{R} $ so by the ratio test, the series $ \sum_{k=0}^{\infty} \frac{t^k}{\Gamma(\alpha k + 1)}  $
	converges for all $ t \in \mathbb{R} $.
\end{proof}

Using this lemma we can then go on to state and prove the following lemma.

\begin{lemma}
\label{lem:lap_mit}
	\begin{align*}	
		\laplace{ E_\alpha (\beta t^\alpha)} = \frac{s^{\alpha - 1}}{s^\alpha - \beta}
	\end{align*}
\end{lemma}
\begin{proof}
	See that
	\begin{align*}
		\laplace{ E_\alpha (\beta t^\alpha)} = \int_0^\infty e^{-st} \sum_{k=0}^\infty \frac{(\beta t^\alpha)^k}{\Gamma(\alpha k+1)} dt
	\end{align*}
	and because the series converges absolutely for all $ t \in \mathbb{R} $ (lemma \ref{lem:mit_conv}) we may interchange the integral
	and the sum to get
	\begin{align*}
		\int_0^\infty e^{-st} \sum_{k=0}^\infty \frac{(\beta t^\alpha)^k}{\Gamma(\alpha k+1)} dt &= \sum_{k=0}^\infty \int_0^\infty e^{-st} \frac{(\beta t^\alpha)^k}{\Gamma(\alpha k + 1)} dt \\
			&= \sum_0^\infty \frac{\beta^k}{\Gamma(\alpha k + 1)} \int_0^\infty e^{-st} t^{\alpha k} dt. \\
	\end{align*}
	By performing the change of variables $ x =st $ we get that 
	\begin{align*}
		\sum_0^\infty \frac{\beta^k}{\Gamma(\alpha k + 1)} \int_0^\infty e^{-st} t^{\alpha k} dt 
			&= \sum_0^\infty \frac{\beta^k s^{-(k+1)}}{\Gamma(\alpha k + 1)} \underbrace{\int_0^\infty e^{-x} x^{\alpha k} dx}_{\Gamma(\alpha k + 1)} \\
			&= \sum_{k=0}^\infty \beta^{k} s^{-(\alpha k + 1)} \\
			&= \frac{s^{\alpha-1}}{s^\alpha - \beta}.		
	\end{align*}
	So we have that 
	\begin{align*}	
		\laplace{ E_\alpha (\beta t^\alpha)} = \frac{s^{\alpha - 1}}{s^\alpha - \beta}
	\end{align*}	
	as required.
\end{proof}

We now have sufficient tools to attack the original problem, that is finding a solution to \eqref{eq:fde-1}, \eqref{eq:fde-1-ic}.

\begin{lemma}
	The FDE defined in \eqref{eq:fde-1} and \eqref{eq:fde-1-ic}, restated here for completeness 
	\begin{align*}
		\left( \prescript{C}{}{\mathcal{D}_0^\alpha}y \right)(t) = \beta y(t) 
	\end{align*}

	along with the initial conditions 
	\begin{align*}
		y^{(k)}(0) = 
		\begin{cases}
			1 & k = 0 \\
			0 & 1 \leq k \leq \lfloor \alpha \rfloor - 1  
		\end{cases}
	\end{align*}
	has solution $ y(t) = E_\alpha \left( \beta t^\alpha \right) $.
\end{lemma}
\begin{proof}
	Taking the Laplace transform of both sides of \eqref{eq:fde-1} yields
	\begin{align*}
		\laplace{\capder{0}{\alpha}{y}} &= \beta \laplace{y} \\
		s^{-(n+\alpha)} \left[s^n \laplace{y} - \sum_{k=0}^{n-1} s^{n-k-1} y^{(k)}(0) \right] &= \beta \laplace{y}
	\end{align*}
	by the result of lemma \ref{lem:lap_cap}. 
	Then taking into account \eqref{eq:fde-1-ic} we get
	\begin{align*}
		s^{-(n+\alpha)} \left[s^n \laplace{y} - s^{n-1}\right] &= \beta \laplace{y}
	\end{align*}
	and so 
	\begin{align*}
		\laplace{y} = \frac{s^{\alpha-1}}{s^\alpha - \beta}.
	\end{align*}
	By using the result of lemma \ref{lem:lap_mit} we have that 
	\begin{align*}
		y(t) = E_\alpha(\beta t^\alpha)
	\end{align*}
\end{proof}

\section*{Solution to a Multi-Order Fractional Differential Equation}

This section follows the technique outlined in \cite{Podlubny1999}.

We wish to consider the following differential equation
\begin{align}
	\label{eq:fde-multi-order}
	\rld{0}{\Lambda}{y}(t) + \rld{0}{\lambda}{y}(t) = f(t)
\end{align}

where $ 0 < \lambda < \Lambda < 1 $.


Firstly note that this differential equation is in terms of Riemann-Liouville derivatives. If we were to specify
initial conditions we would be compelled to specify them in terms of fractional derivatives, so we leave them
unspecified here to see the solution in general.

Again we will introduce a definition and prove a lemma which we will need to get a solution to \ref{eq:fde-multi-order}

\begin{definition}[Two Paramter Mittag-Lefler Function]
	\label{def:mit-lef-2}
	We define the two paramter Mittag-Lefler function with the power series
	\begin{align*}
		E_{\alpha, \gamma}(t) &:= \sum_{k=0}^\infty \frac{t^k}{\Gamma(\alpha k + \gamma)}.
	\end{align*}
	Note that this is just a generalisation of the one paramter Mittag-Lefler function, in that
	$ E_{\alpha}(t) = E_{\alpha, 1}(t) $.
\end{definition}

The follopwing lemma is essentially a generalisation of lemma \ref{lem:lap_mit}.
\begin{lemma}
	\label{lem:lap_mit_2}
	The Laplace transform of $ t^{\alpha m + \gamma - 1}E_{\alpha, \gamma}^{(m)}(t) $ is given by
	\begin{align*}
		\laplace{ t^{\alpha m + \gamma - 1}E_{\alpha,\gamma}^{(m)} (\beta t^\alpha)} = \frac{m!s^{\alpha-\gamma}}{(s^\alpha - \beta)^{m+1}}
	\end{align*}
\end{lemma}
\begin{proof}
	Firstly see that
	\begin{align*}
		E_{\alpha,\gamma}^{(m)}(t) &= \sum_{k=m}^{\infty} \frac{\frac{k!}{(k-m)!}t^{k-m}}{\gamma(\alpha k + \gamma)} \\
			&= \sum_{k=0}^{\infty} \frac{(k+m)!t^k}{k!\Gamma(\alpha k + \gamma)}
	\end{align*}
	so we have that
	\begin{align*}
		E_{\alpha, \gamma}^{(m)}(\beta t^\alpha) &= \sum_{k=0}^{\infty} \frac{(k+m)!t^{\alpha k} \beta^k}{k! \Gamma(\alpha (k+m) + \gamma)}.
	\end{align*}
	We can then write that
	\begin{align*}
		\laplace{t^{\alpha m + \gamma - 1}E_{\alpha, \gamma}^{(m)}(t)} &= \int_0^\infty t^{\alpha m + \gamma - 1}  \sum_{k=0}^{\infty} \frac{(k+m)!t^{\alpha k} \beta^k}{k! \Gamma(\alpha (k+m) + \gamma)} \\
			&= \sum_{k=0}^\infty \frac{\beta^k (k+m)!}{\Gamma(\alpha(k+m) + \gamma) k!} \underbrace{\int_0^\infty e^{-st} t^{\alpha (k+m) + \gamma - 1}dt}_{\circledast}.
	\end{align*}
	Considering just $ \circledast $ and performing the substitution $ x = st $ we get that 
	\begin{align*}
		\circledast &= s^{-\alpha(k+m) - \gamma} \int_0^\infty e^{-x} x^{\alpha (k+m) + \gamma - 1} dx \\
			&= s^{-\alpha(k+m) - \gamma} \Gamma(\alpha(k+m) + \gamma)
	\end{align*}
	and so 
	\begin{align*}
		\laplace{t^{\alpha m + \gamma - 1}E_{\alpha, \gamma}^{(m)}(t)} = s^{-\alpha m - \gamma}\sum_{k=0}^\infty \left(\frac{\beta}{s^\alpha}\right)^k\frac{(k+m)!}{k!} .
	\end{align*}
	Now by the derivative rule for geometric series we get
	\begin{align*}
		\sum_{k=0}^\infty \left(\frac{\beta}{s^\alpha}\right)^k\frac{(k+m)!}{k!} &= \frac{m!}{(1-\frac{\beta}{s^\alpha})^{m+1}} \\
			&= \frac{s^{\alpha(m+1)} m!}{(s^\alpha - \beta)^{m+1}}
	\end{align*}
	and so 
	\begin{align*}
		\laplace{t^{\alpha m + \gamma - 1}E_{\alpha, \gamma}^{(m)}(t)} = \frac{m!s^{\alpha-\gamma}}{(s^\alpha - \beta)^{m+1}}.
	\end{align*}
\end{proof}

\begin{lemma}
	The fractional differential equation, \ref{eq:fde-multi-order}, restated here for completeness,
	\begin{align*}
		\rld{0}{\Lambda}{y}(t) + \rld{0}{\lambda}{y}(t) = f(t)
	\end{align*}
	has solution, given by
	\begin{align*}
		y(t) = C g(t) + \int_0^t g(t-\tau)f(\tau) d\tau
	\end{align*}
	where
	\begin{align*}
		C &= \rld{0}{\Lambda-1}{y}(0) + \rld{0}{\lambda-1}{y}(0) \\
		g(t) &= t^{\Lambda - 1} E_{\Lambda - \lambda, \Lambda}(-t^{\Lambda - \lambda}).
	\end{align*}
\end{lemma}

\begin{proof}

	Taking the Laplace transform of both sides of \ref{eq:fde-multi-order} and using the result of lemma \ref{lem:lap_rld}
	we get that 
	\begin{align*}
		\laplace{\rld{0}{\Lambda}{y}(t)} + \laplace{\rld{0}{\lambda}{y}(t)} &= \laplace{f(t)} \\
		s^\Lambda Y(s) + s^\lambda Y(s) - \rld{0}{\Lambda-1}{y}(0) - \rld{0}{\lambda-1}{y}(0) &= F(s).
	\end{align*}
	Note that $$ C = \rld{0}{\Lambda-1}{y}(0) + \rld{0}{\lambda-1}{y}(0) $$ is a constant so we write
	\begin{align*}
		Y(s) &= \frac{C + F(s)}{s^\Lambda + s^\lambda} \\
			&= \left( C + F(s)\right) \frac{s^{-\lambda}}{s^{\Lambda-\lambda} + 1}.
	\end{align*}
	
	Let $$ G(s) = \frac{s^{-\lambda}}{s^{\Lambda-\lambda} + 1} $$
	and by using lemma \ref{lem:lap_mit_2} with $ \alpha = \Lambda - \lambda $ and $ \gamma = \Lambda $
	we get that $$ g(s) = t^{\Lambda  -1}E_{\Lambda - \lambda, \Lambda}(-t^{\Lambda - \lambda}) $$ where 
	$$ \laplace{g(t)} = G(s) $$.
	
	Then using the Laplace convolution theorem we get that 
	\begin{align*}
		y(t) = C g(t) + \int_0^t g(t-\tau)f(\tau) d\tau
	\end{align*}
	where
	\begin{align*}
		C &= \rld{0}{\Lambda-1}{y}(0) + \rld{0}{\lambda-1}{y}(0) \\
		g(t) &= t^{\Lambda - 1} E_{\Lambda - \lambda, \Lambda}(-t^{\Lambda - \lambda}).
	\end{align*}
\end{proof}

\section*{Existence and Uniquness of Fractional Differential Equations}

After looking at the solution to a couple of fractional differential equations 
we wish to consider the existence an uniqueness of solutions to a class fractional differential equations. 
This generalizes a result and technique of Tisdell \cite{Tisdell2012} but a similar result for Miller-Ross sequential
fractional differential equations can be found in \cite{Podlubny1999}.


\begin{theorem}[Existence and Uniqueness]
\label{thm-existence-uniq}
	Define
		$$ S:= \{ (t,p) \in \Rl^2 : t \in [0, a], p \in \Rl \} $$
	Let $ f : S \lra \Rl $ be continuous. If there is a positive constant L such that 
		$$ |f(t,u) - f(t,v)| \leq L|u-v|, \text{ for all } (t,u), (t,v) \in S $$
	and a set of constants $ \{ \alpha_j \}_{j = 1}^{N} $, $ \{ \beta_j \}_{j=1}^N $
	such that
	$$
		\sum_{j=2}^N \left|\frac{\beta_j}{\beta_1}\right| a^{\alpha_1 - \alpha_j} < 1
	$$
	then the following initial value problem has a unique solution on $ [0, a] $.
	\begin{align}
		\label{eq-fde-ivp-1}
		\sum_{j=1}^{N} \beta_j\capder{0}{\alpha_j}{x}(t) = f(t,x(t)) \\
		\label{eq-fde-ivp-ic-1}
		x(0) = A_0, x_1(0) = A_1, \ldots, x^{n_N}(0) = A_{n_N}
	\end{align}
	where $ \alpha_1 > \alpha_2 > \ldots > \alpha_N $
	and $ n_j = \lceil \alpha_j \rceil - 1 $.
\end{theorem}
To do this we will need several lemmas. 

\begin{lemma}
	The IVP defined \eqref{eq-fde-ivp-1}, \eqref{eq-fde-ivp-ic-1} is equivalent to the integral equation
	\begin{align*}
		x(t) &= \sum_{k=1}^{n_1}\frac{A_kt^k}{k!} + \frac{1}{\beta_1} \Bigl( \frac{1}{\Gamma(\alpha_1)}\int_{0}^{t} (t-s)^{\alpha_1 - 1}f(s,x(s))ds \\
			& \ \ \ - \sum_{j=2}^{N}\beta_j \frac{1}{\Gamma(\alpha_1 - \alpha_j)}
			\int_{0}^{t}(t-s)^{\alpha_1 - \alpha_j - 1}\left(x(s) - \sum_{k=1}^{n_j}\frac{A_ks^k}{k!} \right) ds \Bigr)
	\end{align*}
\end{lemma}
\begin{proof}
	Apply $ \rli{0}{\alpha}{} $ to both sides of \eqref{eq-fde-ivp-1} and recognize that
	$$
		\rli{0}{\alpha}{\capder{0}{\alpha}{x}}(t) = x(t) + \sum_{k=0}^{n} \frac{x^{(k)}(0)t^{k}}{k!}
	$$
	where $ n = \lceil \alpha \rceil - 1 $.
\end{proof}


\begin{lemma}	
\label{lem-rli-mit-lef-1}
	\begin{align*}
		\rli{0}{\xi}{E_\alpha(\gamma t^\alpha)} \leq t^\xi E_\alpha(\gamma t^\alpha)
	\end{align*}
\end{lemma}
\begin{proof}
	See that
	\begin{align*}
		\rli{0}{\xi}{E_\alpha(\gamma t^\alpha)} &= \frac{1}{\Gamma(\xi)} \int_0^t E_\alpha(\gamma s^\alpha)(t-s)^{\xi - 1} ds \\
			&= \frac{1}{\Gamma(\xi)} \int_0^t \sum_{k=0}^\infty \frac{\gamma^k s^{\alpha k}}{\Gamma(\alpha k + 1)} (t-s)^{\xi - 1} ds \\
			&= \frac{1}{\Gamma(\xi)} \sum_{k=0}^\infty \frac{\gamma^k}{\Gamma(\alpha k + 1)} \underbrace{\int_0^t s^{\alpha k}(t-s)^{\xi - 1} ds}_\circledast.
	\end{align*}
	Letting $ \tau = \frac{s}{t} $ we have that 
	\begin{align*}
		\circledast &= \int_0^1 (t\tau)^{\alpha k} (t - t\tau)^{\xi - 1} t d\tau \\
			&= t^{\alpha k + \xi}\int_0^1 (\tau)^{\alpha k} (1 - 1\tau)^{\xi - 1} d\tau \\
			&= t^{\alpha k + \xi} B(\alpha k + 1, \xi) \\
			&= t^{\alpha k + \xi} \frac{\Gamma(\alpha k + 1) \Gamma(\xi)}{\Gamma(\alpha k + \xi + 1)}.
	\end{align*}
	This means that 
	\begin{align*}
		\rli{0}{\xi}{E_\alpha(\gamma t^\alpha)} &= \sum_{k=0}^\infty \frac{\gamma^k t^{\alpha k + \xi}}{\Gamma(\alpha k + \xi + 1)} \\
			&= t^{\xi}\sum_{k=0}^\infty\frac{\gamma^k t^{\alpha k}}{\Gamma(\alpha k + \xi + 1)} \\
			&\leq t^{\xi}\sum_{k=0}^\infty\frac{\gamma^k t^{\alpha k}}{\Gamma(\alpha k + 1)} \\
			&= t^{\xi} E_\alpha(\gamma t^\alpha).
	\end{align*}
\end{proof}


\begin{lemma}	
\label{lem-rli-mit-lef-2}
	\begin{align*}
		\rli{0}{\alpha}{E_\alpha(\gamma t^\alpha)} = \frac{1}{\gamma} \left( E_\alpha(\gamma t^\alpha) - 1 \right)
	\end{align*}
\end{lemma}
\begin{proof}
	See that
	\begin{align*}
		\rli{0}{\alpha}{E_\alpha(\gamma t^\alpha)} &= \frac{1}{\Gamma(\alpha)} \int_0^t E_\alpha (\gamma s^\alpha)(t-s)^{\alpha - 1} ds \\
			&= \frac{1}{\Gamma(\alpha)} \sum_{k=0}^\infty \frac{\gamma^k}{\Gamma(\alpha k + 1)} \underbrace{\int_0^t s^{\alpha k} (t-s)^\alpha ds}_{\circledast}.
	\end{align*}

	Letting $ \tau = \frac{s}{t} $ we have that 
	\begin{align*}
		\circledast &= \int_0^1 (t\tau)^{\alpha k}(t-t\tau)^{\alpha - 1} t d\tau \\
			&= t^{\alpha (k + 1)}\int_0^1 \tau^{\alpha k}(1-\tau)^{\alpha - 1} d\tau \\
			&= t^{\alpha (k + 1)}B(\alpha k + 1, \alpha) \\
			&= t^{\alpha (k + 1)} \frac{\Gamma(\alpha k + 1) \Gamma(\alpha)}{\Gamma(\alpha(k + 1) + 1)}.
	\end{align*}
	This then means that 
	
	\begin{align*}
		\rli{0}{\alpha}{E_\alpha(\gamma t^\alpha)} &= \sum_{k=0}^\infty \frac{\gamma^k t^{\alpha(k+1)}}{\Gamma(\alpha(k + 1) + 1)} \\
			&= \frac{1}{\gamma}\sum_{k=1}^\infty \frac{\gamma^k t^{\alpha k}}{\Gamma(\alpha k+ 1)} \\
			&= \frac{1}{\gamma}\left( \sum_{k=0}^\infty \frac{\gamma^k t^{\alpha k}}{\Gamma(\alpha k+ 1)} - 1\right) \\
			&= \frac{1}{\gamma}\left( E_\alpha(\gamma t^\alpha) - 1 \right).
	\end{align*}
\end{proof}


\begin{proof}[Proof of theorem \ref{thm-existence-uniq}]

	To arrive at this we only have to prove that the map
	\begin{align*}
		[Fx](t) &:= \sum_{k=1}^{n_1}\frac{A_kt^k}{k!} + \frac{1}{\beta_1} \Bigl( \frac{1}{\Gamma(\alpha_1)}\int_{0}^{t} (t-s)^{\alpha_1 - 1}f(s,x(s))ds \\
			& \ \ \ - \sum_{j=2}^{N} \frac{\beta_j}{\Gamma(\alpha_1 - \alpha_j)}\int_{0}^{t}(t-s)^{\alpha_1 - \alpha_j - 1}\left(x(s) - \sum_{k=1}^{n_j}\frac{A_ks^k}{k!} \right) ds \Bigr)
	\end{align*}
	is contractive in the metric space $ \left( C[0,a], d^{\alpha_1}_\gamma \right) $ where 
	$$ d_\gamma^{\alpha_1}(x,y) = \max_{t \in [0, a]} \frac{|x(t) - y(t)|}{E_{\alpha_1}(\gamma t^{\alpha_1})}. $$
	To see this note that
	\begin{align*}
		d_\gamma^{\alpha_1}(Fx,Fy) &= \max_{t \in [0, a]}  \frac{1}{E_{\alpha_1}(\gamma t^{\alpha_1})} 
			\left| \frac{1}{\beta_1} \right| \Bigl| \frac{1}{\Gamma(\alpha_1)}\int_0^t (t-s)^{\alpha_1 - 1} (f(s,x(s)) - f(s,y(s))ds \\ 
			& \ \ \ - \sum_{j=2}^N \frac{\beta_j}{\Gamma(\alpha_1 - \alpha_j)} \int_0^t (t-s)^{\alpha_1 - \alpha_j - 1}(x(s) - y(s)) ds \Bigr| \\
			&\leq \max_{t \in [0, a]} \frac{1}{E_{\alpha_1}(\gamma t^{\alpha_1}) | \beta_1 |} \Big(
			 \frac{1}{\Gamma(\alpha_1)}\int_0^t (t-s)^{\alpha_1 - 1} |f(s,x(s)) - f(s,y(s))|ds \\ 
			& \ \ \ + \sum_{j=2}^N \frac{|\beta_j|}{\Gamma(\alpha_1 - \alpha_j)} \int_0^t (t-s)^{\alpha_1 - \alpha_j - 1}|x(s) - y(s))| ds \Bigr).
	\end{align*}
	By exploiting the Lipshitz condition we can further write that 
	\begin{align*}
		d_\gamma^{\alpha_1}(Fx,Fy) &\leq \max_{t \in [0, a]} \frac{1}{E_{\alpha_1}(\gamma t^{\alpha_1})|\beta_1|} \Bigl(
			\frac{L}{\Gamma(\alpha_1)}\int_0^t (t-s)^{\alpha_1 - 1} |x(s) - y(s)|ds \\ 
			& \ \ \ + \sum_{j=2}^N \frac{|\beta_j|}{\Gamma(\alpha_1 - \alpha_j)} \int_0^t (t-s)^{\alpha_1 - \alpha_j - 1}|x(s) - y(s))| ds \Bigr) \\
			&= \max_{t \in [0, a]} \frac{1}{E_{\alpha_1}(\gamma t^{\alpha_1})|\beta_1|} \Bigl(
			\frac{L}{\Gamma(\alpha_1)}\int_0^t (t-s)^{\alpha_1 - 1} \frac{|x(s) - y(s)|}{E_{\alpha_1}(\gamma s^{\alpha_1})}E_{\alpha_1}(\gamma s^{\alpha_1})ds \\ 
			& \ \ \ + \sum_{j=2}^N \frac{|\beta_j|}{\Gamma(\alpha_1 - \alpha_j)} \int_0^t (t-s)^{\alpha_1 - \alpha_j - 1}\frac{|x(s) - y(s))|}{E_{\alpha_1}(\gamma s^{\alpha_1})}E_{\alpha_1}(\gamma s^{\alpha_1}) ds \Bigr) \\
			&\leq d_\gamma^{\alpha_1}(x,y) \max_{t \in [0, a]} \frac{1}{E_{\alpha_1}(\gamma t^{\alpha_1})|\beta_1|} \Bigl(
			\frac{L}{\Gamma(\alpha_1)}\int_0^t (t-s)^{\alpha_1 - 1} E_{\alpha_1}(\gamma s^{\alpha_1}) ds \\
			& \ \ \ + \sum_{j=2}^N \frac{|\beta_j|}{\Gamma(\alpha_1 - \alpha_j)} \int_0^t (t-s)^{\alpha_1 - \alpha_j - 1}E_{\alpha_1}(\gamma s^{\alpha_1}) ds \Bigr) \\
			&= d_\gamma^{\alpha_1}(x,y) \max_{t \in [0, a]} \frac{1}{E_{\alpha_1}(\gamma t^{\alpha_1})|\beta_1|} \Bigl(
			L \rli{0}{\alpha_1}{E_{\alpha_1}(\gamma t^{\alpha_1}} \\
			& \ \ \ + \sum_{j=2}^N |\beta_j| \rli{0}{\alpha_1 - \alpha_j}{E_{\alpha_1}(\gamma t^{\alpha_1})} \Bigr). \\
	\end{align*}
	We can now use the results of lemmas \ref{lem-rli-mit-lef-1} and \ref{lem-rli-mit-lef-2} to write
	\begin{align*}
		d_\gamma^{\alpha_1}(Fx,Fy) &\leq d_\gamma^{\alpha_1}(x,y) \max_{t \in [0, a]} \frac{1}{E_{\alpha_1}(\gamma t^{\alpha_1})|\beta_1|} \Bigl(
			\frac{L}{\gamma}\left( E_{\alpha_1}(\gamma t^{\alpha_1}) - 1 \right) \\
			& \ \ \ + \sum_{j=2}^N |\beta_j| t^{\alpha_1 - \alpha_j} E_{\alpha_1}(\gamma t^{\alpha_1}) \Bigr) \\
			&= d_\gamma^{\alpha_1}(x,y) \max_{t \in [0, a]} \frac{1}{|\beta_1|} \Bigl(
			\frac{L}{\gamma}\left( 1- \frac{1}{E_{\alpha_1}(\gamma t^{\alpha_1})} \right) + \sum_{j=2}^N |\beta_j| t^{\alpha_1 - \alpha_j}\Bigr) \\
	\end{align*}
	and finally we get that 
	\begin{align*}
		d_\gamma^{\alpha_1}(Fx,Fy) &\leq d_\gamma^{\alpha_1}(x,y) \frac{1}{|\beta_1|}\left( \frac{L}{\gamma} + \sum_{j=2}^N |\beta_j| a^{\alpha_1 - \alpha_j} \right).
	\end{align*}
	By choosing $ \gamma $ sufficiently large we get that 
	$$
		\frac{1}{|\beta_1|}\left( \frac{L}{\gamma} + \sum_{j=2}^N |\beta_j| a^{\alpha_1 - \alpha_j} \right) < 1
	$$
	and so $ F $ is a contractive mapping and thus the IVP defined in \eqref{eq-fde-ivp-1}, \eqref{eq-fde-ivp-ic-1} has a unique solution on $ [0, a] $.
\end{proof}

Note that although existence is resolved (by virtue of the solutions given above)
for the differential equations in (\ref{eq:fde-1}, \ref{eq:fde-1-ic}) and \ref{eq:fde-multi-order}, this 
guarentees uniqueness on some closed interval starting at $ 0 $ for both cases. Its also important
to note that this result can be extended to differential equations involving Riemann-Liouville derivatives, by 
virtue of the correspondence between the Caputo derivative and the Riemann-Liouville derivative \cite{Podlubny1999}. 

\section*{Solution to a Singular Fractional Differential Equation}

We wish to consider the following fractional diffferential equation,  
\begin{align}
\label{eq:fde-singular}
	t^{\alpha + 1} \rld{0}{\alpha + 1}{y}(t) + t^\alpha \rld{0}{\alpha}{y}(t) = f(t)
\end{align}
along with the condition that
\begin{align}
	\left[ \rld{0}{\alpha-k-1}{f}(t)t^{r+\alpha-k-1} \right]_{t=0}^{t \lra \infty} = 0
\end{align}
for all $ 0 \leq k \leq n - 1 $ and suitable $ r $.

To attack this problem we are going to need to consider Mellin transforms and prove several lemmas about Mellin transforms
and Riemann-Liouville fractional derivatives. These results follow closely those in \cite{Podlubny1999}.

\begin{definition}[Mellin Transform]
	We define the Mellin transform of a function $ f $ as
	\begin{align*}
		\tilde{F}(r) = \mellin{f(t)} = \int_0^\infty f(t)t^{r-1} dt.
	\end{align*}
\end{definition}
In this case $ r $ may be complex and we require $ \sigma_1 < \Re(r) < \sigma_2 $ 
where $ \sigma_1 $ and $ \sigma_2 $ are chosen such that 
\begin{align*}
	\int_0^1|f(t)|t^{\sigma_1 - 1}dt < \infty && \int_1^\infty |f(t)|t^{\sigma_2 - 1}dt < \infty
\end{align*}

\begin{definition}
		We define the inverse Mellin transform of $ \tilde{F}(r) $ as
		\begin{align*}
			f(t) = \frac{1}{2 \pi i} \int_{\sigma - i\infty}^{\sigma + i \infty} \tilde{F}(r)t^{-r} dr 
		\end{align*}
		where $ \sigma_2 < \sigma < \sigma_2 $.
\end{definition}
A proof that this is in fact a valid inverse is a well known result and not provided here.

\begin{definition}[Mellin Convolution]
	We define the Mellin convolution of two functions, $ f $ and $ g $, by
	\begin{align*}
		f(t) * g(t) = \int_0^\infty f(t\tau)g(\tau) d\tau.
	\end{align*}
\end{definition}

\begin{theorem}[Mellin Convolution Theorem]
\label{thm:mel-conv}
	The Mellin transform of the Mellin convolution of two functions has a simple expression given by
	\begin{align*}
		\mellin{f(t) * g(t)} = \tilde{F}(r)\tilde{G}(1-r)
	\end{align*}
\end{theorem}
Again this is a well known result and not proved here.

\begin{lemma}
	\label{lem:mel-power}
	The Mellin transform of $ t^\alpha f(t) $ is given by
	\begin{align*}
		\mellin{t^\alpha f(t)} = \tilde{F}(r + \alpha).
	\end{align*}
\end{lemma}
The proof of this follows immediatly from the definition of the Mellin transform.

\begin{lemma}[Mellin Transform of Integer Order Derivatives]
	The Mellin transform of $ f^{(n)}(t) $ is given by
	\begin{align*}
		\mellin{f^{(n)}(t)} = \sum_{k=0}^{n-1} \frac{\Gamma(1-r+k)}{\Gamma(1-r)} \left[ f^{(n-k-1)}(t)t^{r-k-1}\right]_{t=0}^{t\lra\infty} + \frac{\Gamma(1-r+n)}{\Gamma(1-r)}F(r-n).
	\end{align*}
\end{lemma}
This is a well known result and not proved here.

\begin{lemma}[Mellin Transform of the Riemann-Liouville Fractional Integral]
\label{lem:mel-rl-int}
	The Mellin transform of $ \rli{0}{\alpha}{f}(t) $ is given by
	\begin{align*}
		\mellin{\rli{0}{\alpha}{f}(t)} = \frac{\Gamma(1-r-\alpha)}{\Gamma(1-r)} \tilde{F}(r+\alpha)
	\end{align*}
\end{lemma}
\begin{proof}
	Firstly note that 
	\begin{align*}
		\rli{0}{\alpha}{f}(t) = \frac{1}{\Gamma(\alpha)} \int_0^t (t-\tau)^{\alpha-1}f(\tau) d\tau
	\end{align*}
	and with the change of variables $ u =\frac{\tau}{t} $ we can rewrite this as 
	\begin{align*}
		\rli{0}{\alpha}{f}(t) = \frac{t^\alpha}{\Gamma(\alpha)} \underbrace{\int_0^1 (1-u)^{\alpha-1}f(tu) du}_{\circledast}.
	\end{align*}
	If we define a function 
	\begin{align*}
		g(t) :=
		\begin{cases}
			(1-t)^{\alpha-1} & 0 \leq t \leq 1 \\
			0 & \text{ otherwise}
		\end{cases}
	\end{align*}
	and note that
	\begin{align*}
		\mellin{g(t)} &= \int_0^\infty g(t) t^{r-1} dt \\
			&= \int_0^1 (1-t)^{\alpha-1}t^{r-1}dt \\
			&= B(\alpha, r).
	\end{align*}
	Combining this with the result of theorem \ref{thm:mel-conv} we get that
	\begin{align*}
		\circledast &= \mellin{f*g} \\
			&= F(r)B(\alpha,1-r)
	\end{align*}
	and with the result of lemma \ref{lem:mel-power} we have
	\begin{align*}
		\mellin{\rli{0}{\alpha}{f}(t)} &= \frac{1}{\Gamma(\alpha)} F(r+\alpha)B(\alpha,1-r-\alpha) \\
			&= \frac{\Gamma(1-r-\alpha)}{\Gamma(1-r)}F(r+\alpha)
	\end{align*}
\end{proof}

\begin{lemma}
\label{lem:mel-rl-der}
	The Mellin transform of the Riemann-Liouville derivative is given by
	\begin{align*}
		\mellin{\rld{0}{\alpha}{f}(t)} &= \sum_{k=0}^{n-1} \frac{\Gamma(1-r+k)}{\Gamma(1-r)} 
			\left[ \rld{0}{\alpha-k-1}{f}(t)t^{r-k-1} \right]_{t=0}^{t \lra \infty} \\
			&= \ \ \ + \frac{\Gamma(1-r+\alpha)}{\Gamma(1-r)}F(r-\alpha).
	\end{align*}
\end{lemma}
\begin{proof}
	Firstly note that $$ \rld{0}{\alpha}{f}(t) = \frac{d^n}{dt^n} \left[ \rli{0}{n-\alpha}{f}(t) \right] $$
	so we have that
	\begin{align*}
		\mellin{\rld{0}{\alpha}{f}(t)} &= \mellin{\frac{d^n}{dt^n} \left[ \rli{0}{n-\alpha}{f}(t) \right]} \\
	\end{align*}
	and by using the results of lemma \ref{lem:mel-rl-int} and lemma \ref{lem:mel-power} we get that
	\begin{align*}
		\mellin{\frac{d^n}{dt^n} \left[ \rli{0}{n-\alpha}{f}(t) \right]} 
			&= \sum_{k=0}^\infty \frac{\Gamma(1-r+k)}{\Gamma(1-r)} \left[ \frac{d^{n-k-1}}{dt^{n-k-1}} \rli{0}{n-\alpha}{f}(t)t^{r-k-1}\right]_{t=0}^{t \lra \infty} \\
			& \ \ \ + \frac{\Gamma(1-r+n)}{\Gamma(1-r)}\mellin{\rli{0}{n-\alpha}{f}(t)}(r-n) \\
			&= \sum_{k=0}^\infty \frac{\Gamma(1-r+k)}{\Gamma(1-r)} \left[ \rld{0}{\alpha-k-1}{f}(t)t^{r-k-1}\right]_{t=0}^{t \lra \infty} \\
			& \ \ \ + \frac{\Gamma(1-r+n)}{\Gamma(1-r)}\frac{\Gamma(1-r+\alpha)}{\Gamma(1-r+n)}F(r+\alpha)\\
			&= \sum_{k=0}^{n-1} \frac{\Gamma(1-r+k)}{\Gamma(1-r)} 
			\left[ \rld{0}{\alpha-k-1}{f}(t)t^{r-k-1} \right]_{t=0}^{t \lra \infty} \\
			& \ \ \ + \frac{\Gamma(1-r+\alpha)}{\Gamma(1-r)}F(r-\alpha).
	\end{align*}
\end{proof}

\begin{lemma}
	The Mellin transform of $ t^\alpha \rld{0}{\alpha}{f}(t) $ is given by
	\begin{align*}
		\mellin{t^\alpha \rld{0}{\alpha}{f}(t)} 
			&= \sum_{k=0}^{n-1}\frac{\Gamma(1-r-\alpha+k)}{\Gamma(1-r-\alpha)} \left[ \rld{0}{\alpha - k - 1}{f}(t) t^{r+\alpha-k-1}\right]_{t=0}^{t\lra\infty} \\
			& \ \ \ + \frac{\Gamma(1-r)}{\Gamma(1-r-\alpha)}F(r).
	\end{align*}
\end{lemma}
\begin{proof}
	This follows immediatly from lemma \ref{lem:mel-rl-der} and lemma \ref{lem:mel-power}.
\end{proof}
\bibliographystyle{plain}
\bibliography{references}
\end{document}
