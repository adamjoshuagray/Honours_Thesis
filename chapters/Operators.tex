\section{Fractional Operators and Their Properties}
\label{sec:operators}
There are a variety of ways to define fractional differential operators. Good introductions to the various differential operators can be found in \cite{Gorenflo1997, Podlubny1999, Diethelm2002, Samko1993}. In this section, and for the rest of the thesis we will focus of three main fractional operators, the Riemann-Liouville integral and derivative, the Caputo derivative and the Gr{\"u}nwald-Letnikov derivative. There are an abundance of other operators such that the Miller-Ross sequential fractional derivative \cite{Miller1993, Podlubny1999}, the Weyl fractional integral \cite{Gorenflo1997, Samko1993, Raina1979}, the Caputo fractional integral \cite{Gorenflo1997} and fractional operators based on ideas from Fourier analysis \cite{Narayanan2003, Samko1993}.


\subsection{Riemann Liouville Fractional Integral}
In the introduction we considered the historical motivations of the Riemann Liouville fractional integral. Here we will reintroduce it formally and discuss it's properties.
\begin{mdframed}[innertopmargin=10pt]
\begin{definition}[Riemann Liouville Fractional Integeral]
\label{def:rld_1}
The Riemann Liouville fractional integral, based at $ a \in \mathbb{R} $ and of order $ \alpha \in (0, \infty) $ of a function $ f $ is given by
\begin{align}
    \label{eq:rld_1}
    \rli{0}{\alpha}{f}(x) := \frac{1}{\Gamma(\alpha)} \int_a^x f(t)(x-t)^{\alpha - 1} dt
\end{align}
\end{definition}
\end{mdframed}
From this point forward we refer to this as the \emph{fractional integral} of the function.

It is possible to extend to the this definition to $ \alpha \in \Cplx $ and $ a = \infty $. While we will touch on the $ a = -\infty $ case we will neglect discussion of $ \alpha \not\in \mathbb{R} $ and refer the interested reader to Samko et al. \cite{Samko1993}.  It is also worthy of note that some authors define a fractional operator with the limits of integration reversed. We will not deal with these operators, but instead refer the interested reader to \cite{Samko1993, Podlubny1999}. The definition given in \ref{def:rld_1} is the most common in the literature and is compatible with the rest of the theorems and definitions given here.

Although it might seem like a good idea to describe the class of function for which the integral in \eqref{eq:rld_1} is well defined, this is a very complicated problem and we refer the interested reader to \cite{Samko1993}. Although we don't deal with defining the set of admissible functions, this will not have any significant impact on the rest of our discussions because all the functions we deal with will be admissible. 

We firstly wish to show that this coincides with the ordinary integrals when $ n \in \Ntrl $. 
That is we wish to show that 
\begin{align}
    \int_a^x \int_a^{\xi_1} \cdots \int_a^{\xi_{n-1}} f(\xi_n) d\xi_n\cdots d\xi_1 = \frac{1}{(n-1)!} \int_a^x (x-t)^{n-1}f(t)dt
\end{align}
but this is the Cauchy theorem for repeated integration which is a well known result. It follows from applying the fundamental theorem of calculus repeatedly. 

What this means is that the the Riemann-Liouville fractional integral makes sense in that it agrees with the ordinary integral. This is obviously a very desirable property. 

The second property that we wish to explore is the semigroup property.
\begin{mdframed}[innertopmargin=10pt]
\begin{lemma}
    Two fraction integrals $ I_0^\alpha $ and $ I_0^\beta $ have the property that
    \begin{align}
        I_a^\alpha I_a^\beta = I_a^{\alpha + \beta} = I_a^\beta I_a^\alpha.
    \end{align}
\end{lemma}
\end{mdframed}
\begin{proof}
    For a function $ f $ we have that
    \begin{align}
        (I_a^\alpha I_a^\beta)(f)(t) &= \frac{1}{\Gamma(\alpha)}\int_a^t (t-\tau)^{\alpha - 1}  \frac{1}{\Gamma(\beta)}\int_a^\tau (\tau - z)^{\beta - 1}f(z) dz d\tau \\
            &=\frac{1}{\Gamma(\alpha)\Gamma(\beta)} \int_a^t \int_a^\tau  (t-\tau)^{\alpha - 1}(\tau - z)^{\beta - 1}f(z) dz d\tau \\
            \circledast  &= \frac{1}{\Gamma(\alpha)\Gamma(\beta)} \int_a^t \int_z^t (t-\tau)^{\alpha - 1} (\tau - z)^{\beta - 1} f(z) d\tau dz.
    \end{align}
    Now performing the change of variables $ x = \frac{\tau - z}{t - z} $ we get that
    \begin{align}
        \circledast
        &= \frac{1}{\Gamma(\alpha)\Gamma(\beta)}\int_a^t \int_0^1 (t-z)^{\alpha - 1} (1-x)^{\alpha-1} x^{\beta - 1} (t-z)^{\beta - 1}(t-z) dx dz \\
        &= \frac{1}{\Gamma(\alpha)\Gamma(\beta)}\int_0^1 (1-x)^{\alpha - 1} x^{\beta - 1} dx \int_a^t (t-z)^{\alpha + \beta - 1} f(z) dz \\
        &= \frac{B(\alpha,\beta)}{\Gamma(\alpha)\Gamma(\beta)} \int_a^t (t-z)^{\alpha + \beta - 1} f(z) dz \\
        &= \frac{1}{\Gamma(\alpha + \beta)} \int_a^t (t-z)^{\alpha + \beta - 1} f(z) dz \\
        &= I_a^{\alpha + \beta}(f)(t).
    \end{align}
    This shows that $ I_a^\alpha I_a^\beta = I_a^{\alpha + \beta} $ and hence immediately implies that $ I_a^\alpha I_a^\beta = I_a^\beta I_a^\alpha $.
\end{proof}
We now calculate the Laplace transform of the of the Riemann-Liouville fractional integral. While we will not immediately use this result it will prove useful when we attempt to solve fractional differential equations.
\begin{mdframed}[innertopmargin=10pt]
\begin{lemma}
    \label{lem:rli_laplace}
    The Laplace transform of $ \rli{0}{\alpha}{f} $ is given by
    \begin{align}
        \mathcal{L}\{ \rli{0}{\alpha}{f} \} = \frac{\mathcal{L}\{f\}}{s^\alpha}
    \end{align}
\end{lemma}
\end{mdframed}
\begin{proof}
    Noticing that 
    \begin{align}
        \int_0^t (t - \tau)^{\alpha - 1} f(\tau) d\tau 
    \end{align}
    is the Laplace convolution of $ t^{\alpha - 1} $ and $ f(t) $ we can simply use the Laplace convolution theorem
    to get that
    \begin{align}
        \mathcal{L}\{ \rli{0}{\alpha}{f} \} &= \mathcal{L}\{f\} \mathcal{L}\{ t^{\alpha - 1} \} \\
            &= \frac{\mathcal{L}\{f\}}{s^\alpha}.
    \end{align}
\end{proof}
Note that this proof hinges off the Laplace convolution theorem, and so it requires that the base of the fractional derivative, $ a $, has to be $ 0 $. In a similar fashion we can deal with the Fourier transform of the fractional integral but with a base, $ a $, of $ -\infty $ instead so the Fourier convolution can be invoked. We formalise that with the next lemma.
\begin{mdframed}[innertopmargin=10pt]
\begin{lemma}
    \label{lem:rli_fourier}
    The Fourier transform of $ \rli{0}{\alpha}{f} $ is given by
    \begin{align}
        \mathcal{F}\{ \rli{-\infty}{\alpha}{f} \} = \frac{\mathcal{F}\{f\}}{(i\omega)^\alpha}.
    \end{align}
\end{lemma}
\end{mdframed}
\begin{proof}
    We firstly introduce
    \begin{align}
        h_+^\alpha(t) = \begin{cases}
            \frac{t^{\alpha-1}}{\Gamma(\alpha)} & t > 0 \\
            0   & t \leq 0
        \end{cases} 
    \end{align} and then note that it is clear that
    \begin{align}
        \rli{0}{\alpha}{f}(t) &= \frac{1}{\Gamma(\alpha)} \int_{-\infty}^t (t-\tau)^{\alpha-1} f(\tau)d\tau \\
            &= \int_{-\infty}^\infty h_+^\alpha(t-\tau)f(\tau)d\tau \\
            &= (h_+^\alpha * f)(t)
    \end{align}
    where $ * $ represents the Fourier convolution.
    By employing the Fourier convolution theorem it follows that
    \begin{align}
         \mathcal{F}\{ \rli{-\infty}{\alpha}{f} \} = \mathcal{F}\{ h_+^\alpha \}\mathcal{F}\{f\}.
    \end{align}
    It remains only to show that $ \mathcal{F}\{h_+^\alpha\} = (i\omega)^{-\alpha} $. To see this note that the Laplace transform of $ h_+^\alpha $ is given by
    \begin{align}
        \mathcal{L}\{h_+^\alpha\} = s^{-\alpha} 
    \end{align}
    and so by replacing $ s $ with $ i\omega $, as in the Fourier transform we get the result. The convergence of the Fourier integral is guaranteed by Dirichlet's theorem for Fourier integrals as noted in \cite{Podlubny1999}. 
\end{proof}
\subsection{Riemann-Liouville Fractional Derivative}
Although one can approach a fractional derivative from somewhat of a \emph{first principles} approach via the Gr{\"u}nwald-Letnikov derivative \cite{Podlubny1999, Samko1993}, we will not do that here, mainly because for a large class of functions the Gr{\"u}nwald-Letnikov derivative is actually equivalent to the Riemann-Liouville fractional derivative \cite{Podlubny1999} and the Riemann-Liouville fractional derivative is more tractable from a symbolic manipulation perspective.

The idea behind the Riemann-Liouville fractional derivative is to exploit the Riemann-Liouville fractional integral to give the fractional part and then just use integer order derivatives to get the correct order. For example if one wanted to calculate the $ 1/2$-th derivative of a function you would fractionally integrate by $1/2$-th and then differentiate once.
\begin{mdframed}[innertopmargin=10pt]
\begin{definition}[Riemann-Liouville Fractional Derivative]
We define the Riemann-Liouville fractional derivative based at $ a $ and of order $ \alpha \in (0, \infty) $ of a function $ f $ as

\begin{align}
    \label{def:rld}
    \rld{a}{}{\alpha}{f}(x) := \frac{1}{\Gamma(n-\alpha)} \frac{d^n}{dx^n} \int_{a}^{x} \frac{f(t)}{(x-t)^{\alpha - n + 1}} dt
\end{align}
where $ n = \lfloor \alpha \rfloor + 1 $.
\end{definition}
\end{mdframed}
We will not continually specify this condition on $ n $ and in future cases the definition of $ n $ when dealing with fractional derivatives (both Riemann-Liouville and Caputo) the context should make the condition on $ n $ clear.


Again, it should be observed that the precise definition of which functions this transform operates on is not given. As in the fractional integral case this is a complicated question, and we will simply refer to those functions for which \eqref{def:rld} is well defined as admissible functions. The interested reader is referred to \cite{Samko1993} for a discussion of these issues.

We can also see rather trivially that
\begin{align}
    \label{eq:rld_rli_rel}
    \left(\prescript{}{a}{\mathcal{D}}^\alpha{f}\right)(t) = \frac{d^n}{dt^n}\left[ I_a^{n-\alpha}f(t) \right]
\end{align}
which aligns with our intuitive motivation of a fractional derivative above.

Like in the case of the fractional integral we would like to investigate the semi-group properties of the Riemann-Liouville fractional derivative and the relationship between the Riemann-Liouville fractional integral and derivative.
\begin{mdframed}[innertopmargin=10pt]
\begin{lemma}
    \label{lem:rld_left_inverse}
    The Riemann-Liouville derivative of order $ \alpha $ is the left inverse of the Riemann-Liouville integral of order $ \alpha $ in the sense
    that
    \begin{align}
        \left(\prescript{}{a}{\mathcal{D}}^\alpha I_a^\alpha f\right)(t) = f(t) 
    \end{align}
\end{lemma}
\end{mdframed}
\begin{proof}
For an admissible function $ f $ we have that
\begin{align}
    \prescript{}{a}{\mathcal{D}}^\alpha I_{a}^\alpha f &= \frac{1}{\Gamma(n-\alpha)\Gamma(\alpha)} \frac{d^n}{dt^n} \int_a^t (t-\tau)^{n-\alpha-1}\int_a^\tau (\tau - z)^{\alpha-1} f(z) dz d\tau \\
    &= \frac{1}{\Gamma(n-\alpha)\Gamma(\alpha)} \frac{d^n}{dt^n} \int_a^t \int_a^\tau (t-\tau)^{n-\alpha-1}(\tau - z)^{\alpha - 1}f(z)dzd\tau \\
    \circledast &= \frac{1}{\Gamma(n-\alpha)\Gamma(\alpha)} \frac{d^n}{dt^n} \int_a^t \int_z^t (t-\tau)^{n-\alpha-1}(\tau - z)^{\alpha - 1} f(z)d\tau dz.
\end{align}
Now with the change of variables $ x = \frac{\tau - z}{t - z} $ we have that
\begin{align}
    \circledast &= \frac{d^n}{dt^n}\frac{1}{\Gamma(n-\alpha)\Gamma(\alpha)}\int_a^t \int_0^1 (t-z)^{n-\alpha-1}(1-x)^{n-\alpha-1} x^{\alpha-1}(t-z)^{\alpha-1}(t-z)f(z)dxdz \\
    &= \frac{1}{\Gamma(n-\alpha)\Gamma(\alpha)} \int_0^1 (1-x)^{n-\alpha-1}x^{\alpha-1} dx \frac{d^n}{dt^n}\int_a^t (t-z)^{n-1}f(z)dz \\
    &= \frac{B(n-\alpha, \alpha)}{\Gamma(n-\alpha)\Gamma(\alpha)} \frac{d^n}{dt^n} \int_a^t (t-z)^{n-1}f(z)dz \\
    &= \frac{d^n}{dt^n} \frac{1}{(n-1)!} \int_{a}^t(t-z)^{n-1}f(z) dz \\
    &= f(t)
\end{align}
and so the result follows.
\end{proof}

We should note that  by selecting $ \alpha \in \Ntrl $ we could see almost by definition that that the Riemann-Liouville fracitonal derivative coincides with the ordinary derivative, however,  we could also use this lemma to see it as well.

The Riemann-Liouville fractional derivative is \emph{not} a right inverse of the fractional integral. The relationship is considerably more subtle and is formalized in the following lemma.
\begin{mdframed}[innertopmargin=10pt]
\begin{lemma}
    \label{lem:rld_right_res}
    For an admissable function $ f $ we have that
    \begin{align}
        \left(I_a^\alpha \prescript{}{a}{\mathcal{D}}^\alpha f\right)(t) = f(t) - \sum_{j=1}^{n} \left[ \prescript{}{a}{\mathcal{D}}^{\alpha - j}f(t)\right]_{t=a} \frac{(t-a)^{\alpha - j}}{\Gamma(\alpha - j + 1)}
    \end{align}
\end{lemma}
\end{mdframed}
The following proof follows closely to that of Podlubny \cite{Podlubny1999}.
\begin{proof}
    See that
    \begin{align}
        \left(I_a^\alpha \prescript{}{a}{\mathcal{D}}^\alpha f\right)(t) &= \frac{1}{\Gamma(\alpha)} \int_a^t (t-\tau)^{\alpha - 1} \prescript{}{a}{\mathcal{D}}^\alpha f(\tau)d\tau \\
    \label{eq:I_D_right_inverse}
    &= \frac{d}{dt} \left[ \underbrace{\frac{1}{\Gamma(\alpha + 1)} \int_a^t (t-\tau)^\alpha \prescript{}{a}{\mathcal{D}}^\alpha f(\tau) d\tau}_{\circledast}\right].
    \end{align}
So we turn our attention to computing $\circledast $,
and see that
    \begin{align}
        \circledast &= \frac{1}{\Gamma(\alpha + 1)} \int_a^t (t-\tau)^p \frac{d}{dt^n} \left( I_a^{n - \alpha} f(\tau)\right) d\tau
    \end{align}
and by repeated integration by parts we get
    \begin{align}
        \circledast &= \frac{1}{\Gamma(\alpha - n + 1)}\int_a^t (t-\tau)^{\alpha - n} \left( I_a^{n - \alpha}f(\tau) \right) d\tau \\
        & \ \ \ - \sum_{j=1}^n \left[ \frac{d^{n-j}}{dt^{n-j}} \left( I_a^{n - \alpha} f(t) \right) \right]_{t=a} \frac{(t-a)^{\alpha - j -1}}{\Gamma(2 + \alpha - j)} \\
        \label{eq:negative_frac_deriv_1}
        &= I_a^{n-\alpha - 1} \left( I_a^{\alpha - n} f(t) \right) - \sum_{j=1}^n \left[ \prescript{}{a}{\mathcal{D}}^{\alpha - j} f(t)\right]_{t=a} \frac{(t-a)^{\alpha - j + 1}}{\Gamma(2 + \alpha - j)} \\
        \label{eq:negative_frac_deriv_2}
        &= I_{a}^{1} f(t) - \sum_{j=1}^n \left[ \prescript{}{a}{\mathcal{D}}^{\alpha - j} f(t) \right]_{t=a} \frac{(t-a)^{\alpha - j + 1}}{\Gamma(2 + \alpha - j)}
    \end{align}
and then 
by applying the derivative from \eqref{eq:I_D_right_inverse} we get the result. 
\end{proof}

The astute reader might have noticed that we have abused notation in \eqref{eq:negative_frac_deriv_1} and \eqref{eq:negative_frac_deriv_2} where we have permitted a negative order of differentiation in the last term of the sums. This is ok though, because $ \alpha - n < 1 $ and we formalise this in the next lemma. 
\begin{mdframed}[innertopmargin=10pt]
\begin{lemma}
    If we permit $ -1 < \alpha < 0 $ for an order of differentiation then formally we have that
    \begin{align}
        \prescript{}{a}{\mathcal{D}}^\alpha = I_a^{-\alpha}.
    \end{align}
\end{lemma}
\end{mdframed}
The proof is trivial and follows immediately from the definitions.

What lemmas \ref{lem:rld_left_inverse} and \ref{lem:rld_right_res} essentially amount to is an \emph{extended fundamental theorem of calculus}. While these results generalise tha FTC they not more \emph{fundamental} than the ordinary fundamental theorem of calculus in that they cannot be used to prove the FTC. This is because implicitly in the definition of the Riemann-Liouville fractional integral and integral and in the proofs of \ref{lem:rld_left_inverse} and \ref{lem:rld_right_res} we have used the FTC.

It turns out that the semi-group property that we formulated for the fractional integral does not hold for the Riemann-Liouville fractional derivative.
That is, it is not neseccerily true that $ \prescript{}{0}{\mathcal{D}}^\alpha \prescript{}{0}{\mathcal{D}}^\beta =  \prescript{}{0}{\mathcal{D}}^{\alpha + \beta} =  \prescript{}{0}{\mathcal{D}}^\beta  \prescript{}{0}{\mathcal{D}}^\alpha $. Gorenflo and Mainardi give excelent examples of where these equalitities break down \cite{Gorenflo1997}.

Due to the utilitiy of the Laplace transform of the Riemann-Liouville fractional derivative we calculate it here as it will prove useful for latter results.
\begin{mdframed}[innertopmargin=10pt]
\begin{lemma}
\label{lem:lap_rld}
	The Laplace transform the of the Riemann-Liouville derivative of an admissable function $ f $ is given by
	\begin{align}
		\mathcal{L} \left\{ \rld{0}{}{\alpha}{f} \right\} = s^\alpha \mathcal{L} \left\{ f(x) \right\} - \sum_{k=0}^{n-1} s^{k} \rld{0}{}{\alpha-k-1}{f}(0).
	\end{align}
\end{lemma}
\end{mdframed}
\begin{proof}
	See that
	\begin{align}
		\laplace{ \rld{0}{}{\alpha}{f} } &= \laplace{ \der{}{t}{n} \rli{0}{n-\alpha}{f} } \\
		&= s^n\laplace{\rli{0}{n-\alpha}{f}} - \sum_{k=0}^{n-1} s^k \der{}{t}{n-k-1} \rli{0}{n-\alpha}{f}(0)
	\end{align}
	and by applying the result of lemma \ref{lem:rli_laplace} we get
	\begin{align}
			\mathcal{L} \left\{\rld{0}{}{\alpha}{f} \right\} &= s^\alpha \laplace{f} - \sum_{k=0}^{n-1} s^{k} \rld{0}{}{\alpha - k - 1}{f}(0). 
	\end{align} 
\end{proof}

We also consider the Fourier transform of the Riemann-Liouville fractional derivative as it will also prove useful for latter results.
\begin{mdframed}[innertopmargin=10pt]
\begin{lemma}
    \label{lem:rld_fourier}
    For an admissable function $ f $ the Fourier transform of the Riemann-Liouville fractional derivative with $ a = -\infty $ and of order $ \alpha $ is given by
    \begin{align}
        \mathcal{F}\left\{ \left(\prescript{}{-\infty}{\mathcal{D}}^\alpha f\right)(t) \right\} = (i\omega)^{\alpha} \mathcal{F}\left\{ f(t) \right\}.
    \end{align}
\end{lemma}
\end{mdframed}
\begin{proof}
    Since it is possible to regard the Riemann-Liouville fractional derivative in terms of the composition of integer order derivatives with the fractional integral this proof follows quickly from the result of lemma \ref{lem:rli_fourier}.
    Firstly we have that
    \begin{align}
        \left(\prescript{}{-\infty}{\mathcal{D}}^\alpha f\right)(t) = \frac{d^n}{dt^n} \left[ I_{-\infty}^{n - \alpha} f(t)\right]
    \end{align}
    and so by elementry results for Fourier transforms we have that
    \begin{align}
        \mathcal{F}\left\{ \left(\prescript{}{-\infty}{\mathcal{D}}^\alpha f\right)(t) \right\}  &=
        \mathcal{F}\left\{ \frac{d^n}{dt^n} \left[ I_{-\infty}^{n-\alpha} f(t) \right] \right\} \\
        &= (i\omega)^{n} \mathcal{F}\left\{ I_{-\infty}^{n-\alpha} f(t) \right\}
    \end{align}
    and then by using the result of lemma \ref{lem:rli_fourier} we have that
    \begin{align}
        (i\omega)^{n} \mathcal{F}\left\{ I_{-\infty}^{n-\alpha} f(t) \right\} &= (-i\omega)^{n}(-i\omega)^{\alpha - n}\mathcal{F}\left\{ f(t)\right\} \\
        &= (i\omega)^\alpha  \mathcal{F}\left\{ f(t) \right\}
    \end{align}
\end{proof}

Another very useful result is the fractional derivative of a power function. 
\begin{mdframed}[innertopmargin=10pt]
    \begin{lemma}
        \label{lem:rld_power}
        The Riemann-Liouville fractional derivative of a power function is as follows
        \begin{align}
            \prescript{}{0}{\mathcal{D}}^\alpha z^{\nu} = \frac{\Gamma(\nu+1)}{\Gamma(\nu + 1 -\alpha)} z^{\nu - \alpha}.
        \end{align}
    \end{lemma}
\end{mdframed}
\begin{proof}
    Using lemma \ref{lem:lap_rld} we have that
    \begin{align}
        \mathcal{L}\left\{ \prescript{}{0}{\mathcal{D}}^\alpha z^{\nu} \right\} &= s^\alpha \mathcal{L}\{ z^\nu \}
    \end{align}
    but it is well known that
    \begin{align}
        \mathcal{L}\{ z^\nu \} = \frac{\Gamma(\nu + 1)}{s^{\nu + 1}}
    \end{align}
    and so we have that
    \begin{align}
        \mathcal{L}\left\{ \prescript{}{0}{\mathcal{D}}^\alpha z^{\nu} \right\} &= \frac{\Gamma(\nu+1)}{s^{\nu + 1 - \alpha}} \\
            &= \frac{\Gamma(\nu+1)}{\Gamma(\nu + 1 - \alpha)} \frac{\Gamma(\nu + 1 - \nu)}{s^{\nu + 1 - \alpha}}
    \end{align}
    and so by inverting the Laplace transform we get
    \begin{align}
        \prescript{}{0}{\mathcal{D}}^\alpha z^{\nu}  = \frac{\Gamma(\nu + 1)}{\Gamma(\nu+1-\alpha)}z^{\nu - \alpha}.
    \end{align}
\end{proof}
It should be noted that this result agrees with early historical results about how a fractional derivative of a power function should look.

\subsection{Caputo Fractional Derivative}
\label{subsec:caputo}
We now turn our attention to the last fractional differential operator that we will study in detail, the Caputo fractional derivative. 
The Caputo derivative first turned up in a 1967 paper by Caputo on linear models of dissipation \cite{Caputo1967} and again in 1971 in a paper by Caputo and Mainardi \cite{Caputo1971}.  When specifying differential equations (both of the ordinary and partial type) Riemann-Liouville derivatives typically lead to intial conditions and boundary values of fractional order \cite{Caputo1971, Podlubny1999, Samko1993, Heymans2005}, however, physical intuition for what sensible values for these initial and boundary values can be is somewhat hard to come by \cite{Caputo1971, Podlubny1999, Samko1993, Gorenflo1997}. The Caputo derivative, however, leads to systems where the initial and boundary values are of integer order, which leads to much greater physical usefulness \cite{Podlubny1999, Samko1993, Gorenflo1997}

Despite the fact that fractional differential equations with Riemann-Liouville operators in them often lead to initial and boundary conditions which are fractional, recent work has been done on interpreting these systems physically. c.f. \cite{Heymans2005}

We will now make formal a definition of the Caputo fractional derivative.
\begin{definition}
    We define the Caputo fractional derivative based at $ a $ and of order $\alpha \in (0, \infty) $ of an admissable function $ f $, as
    \begin{align*}
        \capder{a}{}{\alpha}{f}(x)      := \frac{1}{\Gamma(n - \alpha)} \int_a^x \frac{\frac{d^n}{dt^n}f(t)}{(x-t)^{\alpha -n + 1}} dt
    \end{align*}
    where $ n = \lfloor \alpha \rfloor + 1 $.
\end{definition}
Here an admissable function is taken to have the same general meaning as described in the Riemann-Liouville fractional integral and derivative cases.

We can easily see that
\begin{align}
    \label{eq:cap_rli_rel}
    \capder{a}{}{\alpha}{f}(x) = I_a^{n-\alpha} \frac{d^n}{dx^n} f(x)
\end{align}
and when we compare this observation with that in \eqref{eq:rld_rli_rel} we see that the Caputo derivative essentially amounts to swapping the order of fractional integration and integer differentiation when compared to the Riemann-Liouville fractional derivative.

We would now like to develop similar a relationship between $ \prescript{C}{a}{\mathcal{D}}^\alpha $ and $ I_a^\alpha $ as was done for $ \prescript{}{a}{\mathcal{D}}^\alpha $ but to do that we will first give a relationship between the Riemann-Liouville fractional derivative and the Caputo fractional derivative.

\begin{mdframed}[innertopmargin=10pt]
\begin{lemma}
    \label{lem:rli-cap}
    For an admissable function $ f $ we have that,
    \begin{align}
		    \left(\prescript{}{0}{I}^\alpha \prescript{C}{0}{\mathcal{D}}^\alpha \right)(t) = f(t) + \sum_{k=0}^{n} \frac{f^{(k)}(0)t^{k}}{k!}.
    \end{align}
\end{lemma}
\end{mdframed}
\begin{proof}
    TODO
\end{proof}

\begin{mdframed}[innertopmargin=10pt]
\begin{lemma}
    \label{lem:cap_rld_rel}
    For an admissable function $ f $ we have that
    \begin{align}
        \left(\prescript{C}{a}{\mathcal{D}}^\alpha f\right)(t) = \left( \prescript{}{a}{\mathcal{D}}^\alpha f \right)(t) - \sum_{k=0}^{n-1} \frac{t^{k-\alpha}}{\Gamma(k - \alpha + 1)}f^{(k)}(a).
    \end{align}
\end{lemma}
\end{mdframed}
\begin{proof}
    TODO
\end{proof}
\begin{mdframed}[innertopmargin=10pt]
\begin{lemma}
    For an admissable function $ f $ we have that
    \begin{align}
        \left(\prescript{C}{a}{\mathcal{D}}^\alpha I_a^\alpha f\right)(t) = f(t) - \sum_{k=0}^{n-1} \frac{t^{k-\alpha}}{\Gamma(k-\alpha+1)} \left[\frac{d^k}{dt^k}I_a^\alpha f(t) \right]_{t=a}
    \end{align}
\end{lemma}
\end{mdframed}
The proof is immediate from lemmas \ref{lem:rld_left_inverse} and \ref{lem:cap_rld_rel}.
A similar but more complicated result exists for $  I_a^\alpha \prescript{C}{a}{\mathcal{D}}^\alpha $ by using the result of lemma \ref{lem:rld_right_res} but we will not need it for future results so we omit it. What will prove useful are results for the Laplace and Fourier transforms of the Caputo fractional derivative of a fucntion.
\begin{mdframed}[innertopmargin=10pt]
\begin{lemma}
\label{lem:cap_laplace}
    The Laplace transform of the Caputo derivative of an admissable function $ f $ is given by
    \begin{align}
        \laplace{\capder{0}{t}{\alpha}{f}} = s^{\alpha - n} \left[ s^n \laplace{f} - \sum_{k=0}^{n-1} s^{n-k-1} \left( \der{f}{t}{k} \right)(0) \right].
    \end{align}
\end{lemma}
\end{mdframed}
\begin{proof}
    See that
    \begin{align}
        \laplace{\capder{0}{t}{\alpha}{f}} &= \laplace{  \rli{0}{n-\alpha}{\der{f}{t}{n}}} \\
            &= \frac{1}{\Gamma(n-\alpha)}\laplace{ \int_0^t (t-u)^{n-\alpha-1} \der{f}{t}{n} du} \\ 
    \end{align}
    which is the Laplace transform of a convolution so
    \begin{align}
        \Gamma(n-\alpha)\laplace{ \int_0^t (t-u)^{n-\alpha-1} \der{f}{t}{n} du} &= \laplace{t^{n-\alpha-1}} \laplace{\der{f}{t}{n}} \\
        &= \frac{1}{n-\alpha} \left( s^{-(n-\alpha)} \Gamma(n-\alpha) \right) \\
        & \ \ \ \times \left( s^n \laplace{f} - \sum_{k=0}^{n-1} s^{n-k-1} \left( \der{f}{t}{k} \right)(0) \right) \\
        &= s^{\alpha - n} \left[ s^n \laplace{f} - \sum_{k=0}^{n-1} s^{n-k-1} \left( \der{f}{t}{k} \right)(0) \right].
    \end{align}
\end{proof}
\begin{mdframed}[innertopmargin=10pt]
\begin{lemma}
    \label{lem:cap_fourier}
    For an admissable function $ f $ the Fourier transform of the Riemann-Liouville fractional derivative with $ a = -\infty $ and of order $ \alpha $ is given by
    \begin{align}
        \mathcal{F}\left\{ \left(\prescript{}{-\infty}{\mathcal{D}}^\alpha f\right)(t) \right\} = (i\omega)^{\alpha} \mathcal{F}\left\{ f(t) \right\}.
    \end{align}
\end{lemma}
\end{mdframed}
\begin{proof}
This proof follows in much the same way as the proof of lemma \ref{lem:rld_fourier}.
Firstly we have that
    \begin{align}
        \left(\prescript{C}{-\infty}{\mathcal{D}}^\alpha f\right)(t) =  I_{-\infty}^{n - \alpha} \frac{d^n}{dt^n}f(t)
    \end{align}
    and so by using the result of lemma \ref{lem:rli_fourier} we have that
    \begin{align}
        \mathcal{F}\left\{ \left(\prescript{}{-\infty}{\mathcal{D}}^\alpha f\right)(t) \right\}  &=
        \mathcal{F}\left\{ \left[ I_{-\infty}^{n-\alpha} \frac{d^n}{dt^n}f(t) \right] \right\} \\
        &= (i\omega)^{\alpha - n} \mathcal{F}\left\{ \frac{d^n}{dt^n}f(t) \right\}
    \end{align}
    and then by using elementary Fourier transform results we get
    \begin{align}
        (-i\omega)^{\alpha -n} \mathcal{F}\left\{ I_{-\infty}^{n-\alpha} f(t) \right\} &= (-i\omega)^{\alpha - n}(i\omega)^{n}\mathcal{F}\left\{ f(t)\right\} \\
        &= (i\omega)^\alpha  \mathcal{F}\left\{ f(t) \right\}
    \end{align}
\end{proof}
A careful reader might notice that the results of lemmas \ref{lem:rld_fourier} and \ref{lem:cap_fourier} are the same. This coincidence is actually hiding a more remarkable result which we formalise in the following lemma.
\begin{mdframed}[innertopmargin=10pt]
\begin{lemma}
    For an admissible function $ f $ such that $ \lim_{t \lra -\infty} f^{(k)}(t) = 0 $ for all $ 0 \leq k \leq n-1 $ with $ n $ defined from $ \alpha $ in the usual way we have that
    \begin{align}
        \left(\prescript{C}{-\infty}{\mathcal{D}}^\alpha f\right)(t) = \left( \prescript{}{-\infty}{\mathcal{D}}^\alpha f \right)(t).
    \end{align}
\end{lemma}
\end{mdframed}
\begin{proof}
    By using the result of lemma \ref{lem:cap_rld_rel} we have that
    \begin{align}
        \prescript{C}{-\infty}{\mathcal{D}}^\alpha f(t) &= \prescript{}{-\infty}{\mathcal{D}}^\alpha f(t) - \sum_{k=0}^{n-1} \frac{t^{k-\alpha}}{\Gamma(k - \alpha + 1)}\lim_{a \lra -\infty}f^{(k)}(a) \\
    &=  \prescript{}{-\infty}{\mathcal{D}}^\alpha f(t). 
    \end{align}
\end{proof}
The restriction on $ f $ is not as limiting as one might imagine because for the Fourier transform to have nice properties such as being an automorphism one would want $ f $ to come from some sort of Schwartz space and the membership criteria for a Schwartz space are considerably more restrictive than the conditions given in the above lemma. In fact in defining precisely which functions are \emph{admissable} one ends up having to place similar conditions on $ f $. This is not a rabbit hole that we wish to explore further, however, and we refer the interested reader to \cite{Samko1993} for a more thorough discussion of the matter. 

\subsection{Gr{\"u}nwald-Letnikov Derivative} 
We actually won't make much use of the Gr{\"u}nwald-Letnikov derivative in future discussions, however, any treatment of fractional calculus would be incomplete without at least a brief discussion of this operator.

In some sense we could say that the Gr{\"u}nwald-Letnikov derivative is a \emph{first principles} derivative. 

We define the first derivative of a function $ f : R \lra R $ as 
\begin{align}
	\frac{df(x)}{dx} = \lim_{h\lra 0} \frac{f(x)-f(x-h)}{h}
\end{align}\footnote{Obviously we could have also chosen a forward difference. The arguments that follow would be essentially the same.} 
and if this limit exists we say that the function is differentiable. It turns out, rather unsurprisingly, that we can extend this definition to higher orders by writing
\begin{align}
	\frac{d^nf(x)}{dx^n} = \lim_{h\lra0} \frac{\Delta^n f(x)}{h^n}
\end{align}
where
\begin{align}
	\Delta f(x) &= f(x) - f(x-h) \\
	\Delta^2 f(x) &= \Delta[f(x) - f(x-h)] \\
	&\vdots \\
	\Delta^n f(x) &= \sum_{k=0}^n {n\choose k} (-1)^kf(x-jh).
\end{align}
Usually binomial coefficients are written in terms of integers, but as
\begin{align}
	{n \choose k } = \frac{n!}{k!(n-k)!}
\end{align}
we could just generalise this with gamma functions to take non-integer values of $ n $.
This would give us
\begin{align}
	{ \alpha \choose k } = \frac{\Gamma(\alpha + 1)}{k! \Gamma(\alpha - k + 1)}.
\end{align}
So it makes sense to make the following definition.
\begin{mdframed}[innertopmargin=10pt]
	\begin{definition}[Gr{\"u}nwald-Letnikov Derivative]
		We define the Gr{\"u}nwald-Letnikov derivative of an admissible function $ f $ as
		\begin{align}
			\label{eq:glet-der}
			\frac{d^\alpha f(x)}{dx^\alpha} = \lim_{h \lra 0} h^{-\alpha} \sum_{k=0}^\infty {\alpha \choose k} (-1)^k f(x - kh)
		\end{align}
	\end{definition}
\end{mdframed}
Note that the binomial sum has become an infinite binomial series. This makes sense because unlike in the integer case $ { \alpha \choose k } $ does not become $ 0 $ for $ k > \alpha $ when $ \alpha \not\in \Ntrl $.
In this case we can make some comment on what the set of \emph{admissible functions} includes. Several authors have showed that if $ f : \mathbb{R} \lra \mathbb{R} $ is bounded and has derivatives up to $ \lceil alpha \rceil $ then the fractional derivative \eqref{eq:glet-der} exists \cite{Meerschaert2011, Podlubny1999}. 
Podlubny also showed that if $ f : \mathbb{R} \lra \mathbb{R} $ is $ n - 1 $ times continuously differentiable and that $ f^{(n)} $ is integrable then the Riemann-Liouville fractional derivative exists and coincides with Gr{\"u}nwald-Letnikov derivative. 


\clearpage
