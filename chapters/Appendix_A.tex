
\section{Adams-Moulten-Bashford FDE Solver Code}

\subsection{C\# Implementation of AMB FDE Solver}

\subsubsection{Program.cs}
\lstset{language=Java}
\begin{lstlisting}
using System;
using System.Threading;
using System.Threading.Tasks;
using System.Collections.Generic;
using System.IO;
using System.Text;

namespace FDE_Solver
{
	class MainClass
	{
		/// <summary>
		/// This is the delegate use for the right hand side of the
		/// differential equation. It define the signature that the 
		/// forcing function / right hand side must satisfy.
		/// </summary>
		public delegate double ForcingFunction (double x, double y);

		/// <summary>
		/// Main entry point of the program.
		/// </summary>
		/// <param name="args">The command-line arguments.</param>
		public static void Main (string[] args)
		{
		    double[] y = Compute (0.5, new double[] { 1 }, 10, 10000, 12, new ForcingFunction (ff));
		}
		/// <summary>
		/// This is the RHS of the differential equation.
		/// For the purposes of demonstation this code is setup to
		/// solve D^{1/2}y = -y
		/// </summary>
		public static double ff(double x, double y)
		{
			return -y;
		}

		/// <summary>
		/// This calculates the a coefficient described in the K. Diethelm paper
		/// referenced in the body of thesis.
		/// </summary>
		public static double a(int mu, double alpha)
		{
			return (Math.Pow(mu + 2, alpha + 1) - 2 * Math.Pow(mu + 1, alpha + 1) + Math.Pow(mu, alpha + 1))/SpecialFunctions.Gamma(alpha + 2);
		}

		/// <summary>
		/// This calculates the c coefficient described in the K. Diethelm paper
		/// referenced in the body of thesis.
		/// </summary>
		public static double c(int mu,  double alpha)
		{
			return (Math.Pow(mu, alpha+1) - (mu - alpha)*Math.Pow(mu + 1, alpha)) / SpecialFunctions.Gamma(alpha + 2);
		}

		/// <summary>
		/// This calculates the b coefficient described in the K. Diethelm paper
		/// referenced in the body of thesis.
		/// </summary>
		public static double b(int mu, double alpha)
		{
			return (Math.Pow (mu + 1, alpha) - Math.Pow (mu, alpha)) / SpecialFunctions.Gamma (alpha + 1);
		}

		/// <summary>
		/// This calculates the sum I_{j+1} described in the K. Diethelm paper
		/// referenced in the body of thesis.
		/// </summary>
		public static double I_1(int j, double alpha, double[] y_0_diffs, double[] x)
		{
			double value = 0;
			for (int k = 0; k <= Math.Ceiling (alpha) - 1; k++) {
				value += Math.Pow (x [j + 1], k) / SpecialFunctions.Factorial (k) * y_0_diffs [k];
			}
			return value;
		}

		/// <summary>
		/// This calculates the sum H^p_{j,\ell} described in the K. Diethelm paper
		/// referenced in the body of thesis.
		/// </summary>
		public static double H_p(int j, int ell, int p, double[] x, double[] y, double alpha, ForcingFunction f)
		{
			double value = 0;
			for (int k = 0; k <= (ell - 1) * p; k++) {
				value += b (j - k, alpha) * f (x [k], y [k]);
			}
			return value;
		}
		/// <summary>
		/// This calculates the sum L^p_{j,\ell} described in the K. Diethelm paper
		/// referenced in the body of thesis.
		/// </summary>
		public static double L_p(int j, int ell, int p, double[] x, double[] y, double alpha, ForcingFunction f) 
		{
			double value = 0;
			for (int k = (ell - 1) * p + 1; k <= j; k++) {
				value += b (j - k, alpha) * f (x [k], y [k]);
			}
			return value;
		}

		/// <summary>
		/// This calculates the sum H^_{j,\ell} described in the K. Diethelm paper
		/// referenced in the body of thesis.
		/// </summary>
		public static double H(int j, int ell, int p, double[] x, double[] y, double alpha, ForcingFunction f)
		{
			double value = 0;
			value += c (j, alpha) + f (x [0], y [0]);
			for (int k = 1; k <= (ell - 1) * p; k++) {
				value += a (j - k, alpha) * f (x [k], y [k]);
			}
			return value;
		}
		/// <summary>
		/// This calculates the sum L_{j,\ell} described in the K. Diethelm paper
		/// referenced in the body of thesis.
		/// </summary>
		public static double L(int j, int ell, int p, double[] x, double[] y, double alpha, ForcingFunction f, double y_p_1)
		{
			double value = 0;
			for (int k = (ell - 1)*p + 1; k <= j; k++) {
				value += a(j-k, alpha) * f(x[k], y[k]);
			}
			value += f(x[j+1], y_p_1) / SpecialFunctions.Gamma(alpha + 2);
			return value;
		}
		/// <summary>
		/// This does the actual parallel computation for the method.
		/// This is done by setting up a series of tasks with a carefully defined
		/// continuation / dependency structure which ensures that computations which
		/// can run in parallel are allowed to, and ones which are dependent on other
		/// computations run in the right order. For a full description of the dependency structure
		/// see the body of the thesis.
		/// </summary>
		/// <param name="alpha">The order of differentiation.</param>
		/// <param name="y_0_diffs">An array containing the initial conditions in order of increasing
		/// differentiation order.</param>
		/// <param name="T">The last time to compute to.</param>
		/// <param name="N">The number of time steps to use.</param>
		/// <param name="p">Task granularity. This essentially defined the maximum level or concurrency.</param>
		/// <param name="f">The right hand side of the differential equation.</param>
		public static double[] Compute(double alpha, double[] y_0_diffs, double T, int N, int p, ForcingFunction f)
		{
			double[] x = new double[N];
			double[] y = new double[N];
			double[] y_p = new double[N];
			//Drops in the 0th order initial condition.
			y [0] = y_0_diffs [0];
			//Calculates the time step.
			double h = T / N;
			//Sets up all the x values.
			for (int i = 0; i < N; i++)
			{
				x [i] = h * i;
			}
			//Compute each block
			for (int ell = 1; ell <= Math.Ceiling ((double)N / (double)p); ell++) {
				Task<double> taskSum_p = null;
				Task<double> taskSum = null;
				//Compute each variable in each block.
				for (int i = 0; i < p && ((ell - 1) * p) + i < N - 1; i++) {
					//Calculate the j (index) for this variable.
					int j = ((ell - 1) * p) + i;
					//Setup the task dependency structure and set each task running.
					Task<double> taskI = Task.Factory.StartNew (() => I_1 (j, alpha, y_0_diffs, x));

					Task<double> taskH_p = Task.Factory.StartNew (() => H_p (j, ell, p, x, y, alpha, f));
					Task<double> taskH = Task.Factory.StartNew (() => H (j, ell, p, x, y, alpha, f));
					Task<double> taskL_p = null;
					if (taskSum != null) {
						taskL_p = taskSum.ContinueWith ((t) => L_p (j, ell, p, x, y, alpha, f));
					} else {
						taskL_p = Task.Factory.StartNew (() => L_p (j, ell, p, x, y, alpha, f));
					}
					taskSum_p = Task.Factory.ContinueWhenAll(new [] { taskL_p, taskH_p, taskI }, (ts) => y_p[j + 1] = taskI.Result + Math.Pow(h, alpha) * ( taskH_p.Result + taskL_p.Result ) );
					Task<double> taskL = taskSum_p.ContinueWith ((t) => L (j, ell, p, x, y, alpha, f, y_p [j + 1]));
					taskSum = Task.Factory.ContinueWhenAll(new [] { taskH, taskL, taskI }, (ts) => y[j+1] = taskI.Result + Math.Pow(h, alpha) * (taskH.Result + taskL.Result ));
				}
				// Wait for the block to complete.
				if (taskSum != null) {
					taskSum.Wait ();
				}
			}
			//Return the solution.
			return y;
		}



	}
}


\end{lstlisting}


\subsubsection{SpecialFunctions.cs}

\begin{lstlisting}
using System;

namespace FDE_Solver
{
	/// <summary>
	/// Provides special functions that are not available
	/// in System.Math
	/// </summary>
	public class SpecialFunctions
	{
		/// <summary>
		/// Gamma the specified z.
		/// This uses the Lanczos approximation and is only valid
		/// for positive real values of z. This code is essentially
		/// a translation of a python implementation available
		/// at http://en.wikipedia.org/wiki/Lanczos_approximation
		/// on 22nd July 2014
		/// </summary>
		/// <param name="z">The z value.</param>
		public static double Gamma(double z)
		{
			double g = 7;
			double[] p =  new double[] { 0.99999999999980993, 676.5203681218851, -1259.1392167224028,
										 771.32342877765313, -176.61502916214059, 12.507343278686905,
				-0.13857109526572012, 9.9843695780195716e-6, 1.5056327351493116e-7 };
			if (z < 0.5) {
				return Math.PI / (Math.Sin (Math.PI * z) * Gamma (1 - z));
			} else {
				z -= 1;
				double x = p [0];
				for (int i = 1; i < g + 2; i++)
				{
					x += p [i] / (z + i);
				}
				double t = z + g + 0.5;
				return Math.Sqrt (2 * Math.PI) * Math.Pow (t, z + 0.5) * Math.Exp (-t) * x;
			}
		}
		/// <summary>
		/// Calculates the factorial of k.
		/// One could use the gamma function above but it does have slight inaccuracies
		/// so the factorial function has also been provided which returns an integer.
		/// </summary>
		/// <param name="k">The value to take the factorial of.</param>
		public static int Factorial(int k)
		{
			int value = 1;
			for (int i = 1; i <= k; i++) {
				value *= i;
			}
			return value;
		}
	}
}
 
\end{lstlisting}
