\section{Analytical Results for Ordinary Fractional Differential Equations}
%\addcontentsline{toc}{section}{Solution to a Simple Fractional Differential Equation}
In this section we establish some analytical results for ordinary fractional differential equations. These results allow us to solve some simple initial value problems and establish a theoretical framework from which results in other sections can draw.

\subsection{Basic definitions}

\subsection{Solution to a linear intial value problem via Laplace transforms}
We aim to get a solution to the following fractional differential equation (in terms of Caputo derivatives)

\begin{align}
	\label{eq:fde-1}
	\left( \prescript{C}{}{\mathcal{D}_0^\alpha}y \right)(t) = \beta y(t) 
\end{align}

along with the initial conditions 
\begin{align}
	\label{eq:fde-1-ic}
	y^{(k)}(0) = 
	\begin{cases}
		1 & k = 0 \\
		0 & 1 \leq k \leq \lfloor\alpha \rfloor - 1  
	\end{cases}
\end{align}
It turns out that this initial value problem has the solution $ y(t) = E_\alpha \left( \beta t^\alpha \right) $. Where $ E_\alpha $ is the one parameter Mittag-Lefler function.

This solution can be arrived at by a Laplace transform method that is very similar to how this might be done in the integer order case.

We now consider the Laplace transform of the fractional integration and differentiation operators.

	We now define the Mittag-Lefler function and calculate its Laplace transform.

\begin{definition}
	The one parameter Mittag-Lefler $ E_\alpha $ function is defined by its power series.
	$$
		E_\alpha(t) = \sum_{k=0}^{\infty} \frac{t^k}{\Gamma(\alpha k + 1)}
	$$
\end{definition}
It is clear to see the definition of this function is inspired by the exponential function. Before we can calculate the 
Laplace transform of the Mittag-Lefler function we have to prove a simple lemma about the convergence of the 
series which is used in its definition.

\begin{lemma}
\label{lem:mit_conv}

	The series
	$$
		\sum_{k=0}^{\infty} \frac{t^k}{\Gamma(\alpha k + 1)} 
	$$
  	converges absolutely for all $ t \in \mathbb{R} $.
\end{lemma}
\begin{proof}
	Let $ a_k = \frac{t^k}{\Gamma(\alpha k + 1) }$ and see that
	$$ \lvert \frac{a_{k+1}}{a_k} \rvert = |t| \frac{\Gamma(\alpha k + 1) }{\Gamma(\alpha(k+1) + 1)} $$
	and that hence 
	$$
		\lim_{k \longrightarrow \infty} \lvert \frac{a_{k+1}}{a_k} \rvert = 0
	$$
	for all $ t \in \mathbb{R} $ so by the ratio test, the series $ \sum_{k=0}^{\infty} \frac{t^k}{\Gamma(\alpha k + 1)}  $
	converges for all $ t \in \mathbb{R} $.
\end{proof}

Using this lemma we can then go on to state and prove the following lemma.

\begin{lemma}
\label{lem:lap_mit}
	\begin{align}	
		\laplace{ E_\alpha (\beta t^\alpha)} = \frac{s^{\alpha - 1}}{s^\alpha - \beta}
	\end{align}
\end{lemma}
\begin{proof}
	See that
	\begin{align}
		\laplace{ E_\alpha (\beta t^\alpha)} = \int_0^\infty e^{-st} \sum_{k=0}^\infty \frac{(\beta t^\alpha)^k}{\Gamma(\alpha k+1)} dt
	\end{align}
	and because the series converges absolutely for all $ t \in \mathbb{R} $ (lemma \ref{lem:mit_conv}) we may interchange the integral
	and the sum to get
	\begin{align}
		\int_0^\infty e^{-st} \sum_{k=0}^\infty \frac{(\beta t^\alpha)^k}{\Gamma(\alpha k+1)} dt &= \sum_{k=0}^\infty \int_0^\infty e^{-st} \frac{(\beta t^\alpha)^k}{\Gamma(\alpha k + 1)} dt \\
			&= \sum_0^\infty \frac{\beta^k}{\Gamma(\alpha k + 1)} \int_0^\infty e^{-st} t^{\alpha k} dt. \\
	\end{align}
	By performing the change of variables $ x =st $ we get that 
	\begin{align}
		\sum_0^\infty \frac{\beta^k}{\Gamma(\alpha k + 1)} \int_0^\infty e^{-st} t^{\alpha k} dt 
			&= \sum_0^\infty \frac{\beta^k s^{-(k+1)}}{\Gamma(\alpha k + 1)} \underbrace{\int_0^\infty e^{-x} x^{\alpha k} dx}_{\Gamma(\alpha k + 1)} \\
			&= \sum_{k=0}^\infty \beta^{k} s^{-(\alpha k + 1)} \\
			&= \frac{s^{\alpha-1}}{s^\alpha - \beta}.		
	\end{align}
	So we have that 
	\begin{align}	
		\laplace{ E_\alpha (\beta t^\alpha)} = \frac{s^{\alpha - 1}}{s^\alpha - \beta}
	\end{align}	
	as required.
\end{proof}

We now have sufficient tools to attack the original problem, that is finding a solution to \eqref{eq:fde-1}, \eqref{eq:fde-1-ic}.


:wq
\begin{lemma}
	The initial value problem defined in \eqref{eq:fde-1} and \eqref{eq:fde-1-ic}, restated here for completeness 
	\begin{align}
		\left( \prescript{C}{}{\mathcal{D}_0^\alpha}y \right)(t) = \beta y(t) 
	\end{align}

	along with the initial conditions 
	\begin{align}
		y^{(k)}(0) = 
		\begin{cases}
			1 & k = 0 \\
			0 & 1 \leq k \leq \lfloor \alpha \rfloor - 1  
		\end{cases}
	\end{align}
	has solution $ y(t) = E_\alpha \left( \beta t^\alpha \right) $.
\end{lemma}
\begin{proof}
	Taking the Laplace transform of both sides of \eqref{eq:fde-1} yields
	\begin{align}
		\laplace{\capder{0}{t}{\alpha}{y}} &= \beta \laplace{y} \\
		s^{-(n+\alpha)} \left[s^n \laplace{y} - \sum_{k=0}^{n-1} s^{n-k-1} y^{(k)}(0) \right] &= \beta \laplace{y}
	\end{align}
	by the result of lemma \ref{lem:lap_cap}. 
	Then taking into account \eqref{eq:fde-1-ic} we get
	\begin{align}
		s^{-(n+\alpha)} \left[s^n \laplace{y} - s^{n-1}\right] &= \beta \laplace{y}
	\end{align}
	and so 
	\begin{align}
		\laplace{y} = \frac{s^{\alpha-1}}{s^\alpha - \beta}.
	\end{align}
	By using the result of lemma \ref{lem:lap_mit} we have that 
	\begin{align}
		y(t) = E_\alpha(\beta t^\alpha)
	\end{align}
\end{proof}

\subsection{Solution to a multi-order initial value problem via Laplace transforms}
%\addcontentsline{toc}{section}{Solution to a Multi-Order Fractional Differential Equation}
This section follows the technique outlined in \cite{Podlubny1999}.

We wish to consider the following differential equation
\begin{align}
	\label{eq:fde-multi-order}
	\rld{0}{\Lambda}{y}(t) + \rld{0}{\lambda}{y}(t) = f(t)
\end{align}

where $ 0 < \lambda < \Lambda < 1 $.


Firstly note that this differential equation is in terms of Riemann-Liouville derivatives. If we were to specify
initial conditions we would be compelled to specify them in terms of fractional derivatives, so we leave them
unspecified here to see the solution in general.

Again we will introduce a definition and prove a lemma which we will need to get a solution to \ref{eq:fde-multi-order}

\begin{definition}[Two Paramter Mittag-Lefler Function]
	\label{def:mit-lef-2}
	We define the two paramter Mittag-Lefler function with the power series
	\begin{align}
		E_{\alpha, \gamma}(t) &:= \sum_{k=0}^\infty \frac{t^k}{\Gamma(\alpha k + \gamma)}.
	\end{align}
	Note that this is just a generalisation of the one paramter Mittag-Lefler function, in that
	$ E_{\alpha}(t) = E_{\alpha, 1}(t) $.
\end{definition}

The follopwing lemma is essentially a generalisation of lemma \ref{lem:lap_mit}.
\begin{lemma}
	\label{lem:lap_mit_2}
	The Laplace transform of $ t^{\alpha m + \gamma - 1}E_{\alpha, \gamma}^{(m)}(t) $ is given by
	\begin{align}
		\laplace{ t^{\alpha m + \gamma - 1}E_{\alpha,\gamma}^{(m)} (\beta t^\alpha)} = \frac{m!s^{\alpha-\gamma}}{(s^\alpha - \beta)^{m+1}}
	\end{align}
\end{lemma}
\begin{proof}
	Firstly see that
	\begin{align}
		E_{\alpha,\gamma}^{(m)}(t) &= \sum_{k=m}^{\infty} \frac{\frac{k!}{(k-m)!}t^{k-m}}{\gamma(\alpha k + \gamma)} \\
			&= \sum_{k=0}^{\infty} \frac{(k+m)!t^k}{k!\Gamma(\alpha k + \gamma)}
	\end{align}
	so we have that
	\begin{align}
		E_{\alpha, \gamma}^{(m)}(\beta t^\alpha) &= \sum_{k=0}^{\infty} \frac{(k+m)!t^{\alpha k} \beta^k}{k! \Gamma(\alpha (k+m) + \gamma)}.
	\end{align}
	We can then write that
	\begin{align}
		\laplace{t^{\alpha m + \gamma - 1}E_{\alpha, \gamma}^{(m)}(t)} &= \int_0^\infty t^{\alpha m + \gamma - 1}  \sum_{k=0}^{\infty} \frac{(k+m)!t^{\alpha k} \beta^k}{k! \Gamma(\alpha (k+m) + \gamma)} \\
			&= \sum_{k=0}^\infty \frac{\beta^k (k+m)!}{\Gamma(\alpha(k+m) + \gamma) k!} \underbrace{\int_0^\infty e^{-st} t^{\alpha (k+m) + \gamma - 1}dt}_{\circledast}.
	\end{align}
	Considering just $ \circledast $ and performing the substitution $ x = st $ we get that 
	\begin{align}
		\circledast &= s^{-\alpha(k+m) - \gamma} \int_0^\infty e^{-x} x^{\alpha (k+m) + \gamma - 1} dx \\
			&= s^{-\alpha(k+m) - \gamma} \Gamma(\alpha(k+m) + \gamma)
	\end{align}
	and so 
	\begin{align}
		\laplace{t^{\alpha m + \gamma - 1}E_{\alpha, \gamma}^{(m)}(t)} = s^{-\alpha m - \gamma}\sum_{k=0}^\infty \left(\frac{\beta}{s^\alpha}\right)^k\frac{(k+m)!}{k!} .
	\end{align}
	Now by the derivative rule for geometric series we get
	\begin{align}
		\sum_{k=0}^\infty \left(\frac{\beta}{s^\alpha}\right)^k\frac{(k+m)!}{k!} &= \frac{m!}{(1-\frac{\beta}{s^\alpha})^{m+1}} \\
			&= \frac{s^{\alpha(m+1)} m!}{(s^\alpha - \beta)^{m+1}}
	\end{align}
	and so 
	\begin{align}
		\laplace{t^{\alpha m + \gamma - 1}E_{\alpha, \gamma}^{(m)}(t)} = \frac{m!s^{\alpha-\gamma}}{(s^\alpha - \beta)^{m+1}}.
	\end{align}
\end{proof}

\begin{lemma}
	The initial value problem, \ref{eq:fde-multi-order}, restated here for completeness,
	\begin{align}
		\rld{0}{\Lambda}{y}(t) + \rld{0}{\lambda}{y}(t) = f(t)
	\end{align}
	has solution, given by
	\begin{align}
		y(t) = C g(t) + \int_0^t g(t-\tau)f(\tau) d\tau
	\end{align}
	where
	\begin{align}
		C &= \rld{0}{\Lambda-1}{y}(0) + \rld{0}{\lambda-1}{y}(0) \\
		g(t) &= t^{\Lambda - 1} E_{\Lambda - \lambda, \Lambda}(-t^{\Lambda - \lambda}).
	\end{align}
\end{lemma}

\begin{proof}

	Taking the Laplace transform of both sides of \ref{eq:fde-multi-order} and using the result of lemma \ref{lem:lap_rld}
	we get that 
	\begin{align}
		\laplace{\rld{0}{\Lambda}{y}(t)} + \laplace{\rld{0}{\lambda}{y}(t)} &= \laplace{f(t)} \\
		s^\Lambda Y(s) + s^\lambda Y(s) - \rld{0}{\Lambda-1}{y}(0) - \rld{0}{\lambda-1}{y}(0) &= F(s).
	\end{align}
	Note that $$ C = \rld{0}{\Lambda-1}{y}(0) + \rld{0}{\lambda-1}{y}(0) $$ is a constant so we write
	\begin{align}
		Y(s) &= \frac{C + F(s)}{s^\Lambda + s^\lambda} \\
			&= \left( C + F(s)\right) \frac{s^{-\lambda}}{s^{\Lambda-\lambda} + 1}.
	\end{align}
	
	Let $$ G(s) = \frac{s^{-\lambda}}{s^{\Lambda-\lambda} + 1} $$
	and by using lemma \ref{lem:lap_mit_2} with $ \alpha = \Lambda - \lambda $ and $ \gamma = \Lambda $
	we get that $$ g(s) = t^{\Lambda  -1}E_{\Lambda - \lambda, \Lambda}(-t^{\Lambda - \lambda}) $$ where 
	$$ \laplace{g(t)} = G(s) $$.
	
	Then using the Laplace convolution theorem we get that 
	\begin{align}
		y(t) = C g(t) + \int_0^t g(t-\tau)f(\tau) d\tau
	\end{align}
	where
	\begin{align}
		C &= \rld{0}{\Lambda-1}{y}(0) + \rld{0}{\lambda-1}{y}(0) \\
		g(t) &= t^{\Lambda - 1} E_{\Lambda - \lambda, \Lambda}(-t^{\Lambda - \lambda}).
	\end{align}
\end{proof}

\subsection{Existence and uniquness of solutions to initial value problems}
\label{sec:existence_uniq}
%\addcontentsline{toc}{section}{Existence and Uniquness of Fractional Differential Equations}
After looking at the solution to a couple of fractional differential equations 
we wish to consider the existence an uniqueness of solutions to a class fractional differential equations. 
This generalizes a result and technique of Tisdell \cite{Tisdell2012} but a similar result for Miller-Ross sequential
fractional differential equations can be found in \cite{Podlubny1999}.


\begin{theorem}[Uniqueness]
\label{thm-existence-uniq}
Consider the following intial value problem,

	\begin{align}
		\label{eq-fde-ivp-1}
		\sum_{j=1}^{N} \beta_j\capder{0}{t}{\alpha_j}{x}(t) = f(t,x(t)) \\
		\label{eq-fde-ivp-ic-1}
		x(0) = A_0, x_1(0) = A_1, \ldots, x^{n_N}(0) = A_{n_N}
	\end{align}
	where $ \alpha_1 > \alpha_2 > \ldots > \alpha_N $, 
	$ n_j = \lceil \alpha_j \rceil - 1 $ and $ \beta_1 = 1 $.
	
	Define
		$$ S:= \{ (t,p) \in \Rl^2 : t \in [0, a], p \in \Rl \} $$
	Let $ f : S \lra \Rl $ be continuous. If there is a positive constant L such that 
		$$ |f(t,u) - f(t,v)| \leq L|u-v|, \text{ for all } (t,u), (t,v) \in S $$
	and the constants $ \{ \alpha_j \}_{j = 1}^{N} $, $ \{ \beta_j \}_{j=1}^N $
	are such that
	$$
		\sum_{j=2}^N \left|\beta_j\right| a^{\alpha_1 - \alpha_j} < 1
	$$
	then the intial value problem defined in \ref{eq-fde-ivp-1} and \ref{eq-fde-ivp-ic-1} has a unique solution.
\end{theorem}
To prove this we will need several lemmas. 

\begin{lemma}
	The IVP defined in \eqref{eq-fde-ivp-1}, \eqref{eq-fde-ivp-ic-1} is equivalent to the integral equation
	\begin{align}
		x(t) &= \sum_{k=1}^{n_1}\frac{A_kt^k}{k!} + \frac{1}{\beta_1} \Bigl( \frac{1}{\Gamma(\alpha_1)}\int_{0}^{t} (t-s)^{\alpha_1 - 1}f(s,x(s))ds \\
			& \ \ \ - \sum_{j=2}^{N}\beta_j \frac{1}{\Gamma(\alpha_1 - \alpha_j)}
			\int_{0}^{t}(t-s)^{\alpha_1 - \alpha_j - 1}\left(x(s) - \sum_{k=1}^{n_j}\frac{A_ks^k}{k!} \right) ds \Bigr)
	\end{align}
\end{lemma}
\begin{proof}
	Apply $ \rli{0}{\alpha}{} $ to both sides of \eqref{eq-fde-ivp-1} and recognize that
	$$
		\rli{0}{\alpha}{\capder{0}{t}{\alpha}{x}}(t) = x(t) + \sum_{k=0}^{n} \frac{x^{(k)}(0)t^{k}}{k!}
	$$
	where $ n = \lceil \alpha \rceil - 1 $.
\end{proof}


\begin{lemma}	
\label{lem-rli-mit-lef-1}
	\begin{align}
		\rli{0}{\xi}{E_\alpha(\gamma t^\alpha)} \leq t^\xi E_\alpha(\gamma t^\alpha)
	\end{align}
\end{lemma}
\begin{proof}
	See that
	\begin{align}
		\rli{0}{\xi}{E_\alpha(\gamma t^\alpha)} &= \frac{1}{\Gamma(\xi)} \int_0^t E_\alpha(\gamma s^\alpha)(t-s)^{\xi - 1} ds \\
			&= \frac{1}{\Gamma(\xi)} \int_0^t \sum_{k=0}^\infty \frac{\gamma^k s^{\alpha k}}{\Gamma(\alpha k + 1)} (t-s)^{\xi - 1} ds \\
			&= \frac{1}{\Gamma(\xi)} \sum_{k=0}^\infty \frac{\gamma^k}{\Gamma(\alpha k + 1)} \underbrace{\int_0^t s^{\alpha k}(t-s)^{\xi - 1} ds}_\circledast.
	\end{align}
	Letting $ \tau = \frac{s}{t} $ we have that 
	\begin{align}
		\circledast &= \int_0^1 (t\tau)^{\alpha k} (t - t\tau)^{\xi - 1} t d\tau \\
			&= t^{\alpha k + \xi}\int_0^1 (\tau)^{\alpha k} (1 - 1\tau)^{\xi - 1} d\tau \\
			&= t^{\alpha k + \xi} B(\alpha k + 1, \xi) \\
			&= t^{\alpha k + \xi} \frac{\Gamma(\alpha k + 1) \Gamma(\xi)}{\Gamma(\alpha k + \xi + 1)}.
	\end{align}
	This means that 
	\begin{align}
		\rli{0}{\xi}{E_\alpha(\gamma t^\alpha)} &= \sum_{k=0}^\infty \frac{\gamma^k t^{\alpha k + \xi}}{\Gamma(\alpha k + \xi + 1)} \\
			&= t^{\xi}\sum_{k=0}^\infty\frac{\gamma^k t^{\alpha k}}{\Gamma(\alpha k + \xi + 1)} \\
			&\leq t^{\xi}\sum_{k=0}^\infty\frac{\gamma^k t^{\alpha k}}{\Gamma(\alpha k + 1)} \\
			&= t^{\xi} E_\alpha(\gamma t^\alpha).
	\end{align}
\end{proof}


\begin{lemma}	
\label{lem-rli-mit-lef-2}
	\begin{align}
		\rli{0}{\alpha}{E_\alpha(\gamma t^\alpha)} = \frac{1}{\gamma} \left( E_\alpha(\gamma t^\alpha) - 1 \right)
	\end{align}
\end{lemma}
\begin{proof}
	See that
	\begin{align}
		\rli{0}{\alpha}{E_\alpha(\gamma t^\alpha)} &= \frac{1}{\Gamma(\alpha)} \int_0^t E_\alpha (\gamma s^\alpha)(t-s)^{\alpha - 1} ds \\
			&= \frac{1}{\Gamma(\alpha)} \sum_{k=0}^\infty \frac{\gamma^k}{\Gamma(\alpha k + 1)} \underbrace{\int_0^t s^{\alpha k} (t-s)^\alpha ds}_{\circledast}.
	\end{align}

	Letting $ \tau = \frac{s}{t} $ we have that 
	\begin{align}
		\circledast &= \int_0^1 (t\tau)^{\alpha k}(t-t\tau)^{\alpha - 1} t d\tau \\
			&= t^{\alpha (k + 1)}\int_0^1 \tau^{\alpha k}(1-\tau)^{\alpha - 1} d\tau \\
			&= t^{\alpha (k + 1)}B(\alpha k + 1, \alpha) \\
			&= t^{\alpha (k + 1)} \frac{\Gamma(\alpha k + 1) \Gamma(\alpha)}{\Gamma(\alpha(k + 1) + 1)}.
	\end{align}
	This then means that 
	
	\begin{align}
		\rli{0}{\alpha}{E_\alpha(\gamma t^\alpha)} &= \sum_{k=0}^\infty \frac{\gamma^k t^{\alpha(k+1)}}{\Gamma(\alpha(k + 1) + 1)} \\
			&= \frac{1}{\gamma}\sum_{k=1}^\infty \frac{\gamma^k t^{\alpha k}}{\Gamma(\alpha k+ 1)} \\
			&= \frac{1}{\gamma}\left( \sum_{k=0}^\infty \frac{\gamma^k t^{\alpha k}}{\Gamma(\alpha k+ 1)} - 1\right) \\
			&= \frac{1}{\gamma}\left( E_\alpha(\gamma t^\alpha) - 1 \right).
	\end{align}
\end{proof}


\begin{proof}[Proof of theorem \ref{thm-existence-uniq}]

	To arrive at this we only have to prove that the map
	\begin{align}
		[Fx](t) &:= \sum_{k=1}^{n_1}\frac{A_kt^k}{k!} + \frac{1}{\beta_1} \Bigl( \frac{1}{\Gamma(\alpha_1)}\int_{0}^{t} (t-s)^{\alpha_1 - 1}f(s,x(s))ds \\
			& \ \ \ - \sum_{j=2}^{N} \frac{\beta_j}{\Gamma(\alpha_1 - \alpha_j)}\int_{0}^{t}(t-s)^{\alpha_1 - \alpha_j - 1}\left(x(s) - \sum_{k=1}^{n_j}\frac{A_ks^k}{k!} \right) ds \Bigr)
	\end{align}
	is contractive in the metric space $ \left( C[0,a], d^{\alpha_1}_\gamma \right) $ where 
	$$ d_\gamma^{\alpha_1}(x,y) = \max_{t \in [0, a]} \frac{|x(t) - y(t)|}{E_{\alpha_1}(\gamma t^{\alpha_1})}. $$
	To see this note that
	\begin{align}
		d_\gamma^{\alpha_1}(Fx,Fy) &= \max_{t \in [0, a]}  \frac{1}{E_{\alpha_1}(\gamma t^{\alpha_1})} 
			\left| \frac{1}{\beta_1} \right| \Bigl| \frac{1}{\Gamma(\alpha_1)}\int_0^t (t-s)^{\alpha_1 - 1} (f(s,x(s)) - f(s,y(s))ds \\ 
			& \ \ \ - \sum_{j=2}^N \frac{\beta_j}{\Gamma(\alpha_1 - \alpha_j)} \int_0^t (t-s)^{\alpha_1 - \alpha_j - 1}(x(s) - y(s)) ds \Bigr| \\
			&\leq \max_{t \in [0, a]} \frac{1}{E_{\alpha_1}(\gamma t^{\alpha_1}) | \beta_1 |} \Big(
			 \frac{1}{\Gamma(\alpha_1)}\int_0^t (t-s)^{\alpha_1 - 1} |f(s,x(s)) - f(s,y(s))|ds \\ 
			& \ \ \ + \sum_{j=2}^N \frac{|\beta_j|}{\Gamma(\alpha_1 - \alpha_j)} \int_0^t (t-s)^{\alpha_1 - \alpha_j - 1}|x(s) - y(s))| ds \Bigr).
	\end{align}
	By exploiting the Lipshitz condition we can further write that 
	\begin{align}
		d_\gamma^{\alpha_1}(Fx,Fy) &\leq \max_{t \in [0, a]} \frac{1}{E_{\alpha_1}(\gamma t^{\alpha_1})|\beta_1|} \Bigl(
			\frac{L}{\Gamma(\alpha_1)}\int_0^t (t-s)^{\alpha_1 - 1} |x(s) - y(s)|ds \\ 
			& \ \ \ + \sum_{j=2}^N \frac{|\beta_j|}{\Gamma(\alpha_1 - \alpha_j)} \int_0^t (t-s)^{\alpha_1 - \alpha_j - 1}|x(s) - y(s))| ds \Bigr) \\
			&= \max_{t \in [0, a]} \frac{1}{E_{\alpha_1}(\gamma t^{\alpha_1})|\beta_1|} \Bigl(
			\frac{L}{\Gamma(\alpha_1)}\int_0^t (t-s)^{\alpha_1 - 1} \frac{|x(s) - y(s)|}{E_{\alpha_1}(\gamma s^{\alpha_1})}E_{\alpha_1}(\gamma s^{\alpha_1})ds \\ 
			& \ \ \ + \sum_{j=2}^N \frac{|\beta_j|}{\Gamma(\alpha_1 - \alpha_j)} \int_0^t (t-s)^{\alpha_1 - \alpha_j - 1}\frac{|x(s) - y(s))|}{E_{\alpha_1}(\gamma s^{\alpha_1})}E_{\alpha_1}(\gamma s^{\alpha_1}) ds \Bigr) \\
			&\leq d_\gamma^{\alpha_1}(x,y) \max_{t \in [0, a]} \frac{1}{E_{\alpha_1}(\gamma t^{\alpha_1})|\beta_1|} \Bigl(
			\frac{L}{\Gamma(\alpha_1)}\int_0^t (t-s)^{\alpha_1 - 1} E_{\alpha_1}(\gamma s^{\alpha_1}) ds \\
			& \ \ \ + \sum_{j=2}^N \frac{|\beta_j|}{\Gamma(\alpha_1 - \alpha_j)} \int_0^t (t-s)^{\alpha_1 - \alpha_j - 1}E_{\alpha_1}(\gamma s^{\alpha_1}) ds \Bigr) \\
			&= d_\gamma^{\alpha_1}(x,y) \max_{t \in [0, a]} \frac{1}{E_{\alpha_1}(\gamma t^{\alpha_1})|\beta_1|} \Bigl(
			L \rli{0}{\alpha_1}{E_{\alpha_1}(\gamma t^{\alpha_1}} \\
			& \ \ \ + \sum_{j=2}^N |\beta_j| \rli{0}{\alpha_1 - \alpha_j}{E_{\alpha_1}(\gamma t^{\alpha_1})} \Bigr). \\
	\end{align}
	We can now use the results of lemmas \ref{lem-rli-mit-lef-1} and \ref{lem-rli-mit-lef-2} to write
	\begin{align}
		d_\gamma^{\alpha_1}(Fx,Fy) &\leq d_\gamma^{\alpha_1}(x,y) \max_{t \in [0, a]} \frac{1}{E_{\alpha_1}(\gamma t^{\alpha_1})|\beta_1|} \Bigl(
			\frac{L}{\gamma}\left( E_{\alpha_1}(\gamma t^{\alpha_1}) - 1 \right) \\
			& \ \ \ + \sum_{j=2}^N |\beta_j| t^{\alpha_1 - \alpha_j} E_{\alpha_1}(\gamma t^{\alpha_1}) \Bigr) \\
			&= d_\gamma^{\alpha_1}(x,y) \max_{t \in [0, a]} \frac{1}{|\beta_1|} \Bigl(
			\frac{L}{\gamma}\left( 1- \frac{1}{E_{\alpha_1}(\gamma t^{\alpha_1})} \right) + \sum_{j=2}^N |\beta_j| t^{\alpha_1 - \alpha_j}\Bigr) \\
	\end{align}
	and finally we get that 
	\begin{align}
		d_\gamma^{\alpha_1}(Fx,Fy) &\leq d_\gamma^{\alpha_1}(x,y) \frac{1}{|\beta_1|}\left( \frac{L}{\gamma} + \sum_{j=2}^N |\beta_j| a^{\alpha_1 - \alpha_j} \right).
	\end{align}
	By choosing $ \gamma $ sufficiently large we get that 
	$$
		\frac{1}{|\beta_1|}\left( \frac{L}{\gamma} + \sum_{j=2}^N |\beta_j| a^{\alpha_1 - \alpha_j} \right) < 1
	$$
	and so $ F $ is a contractive mapping and thus the IVP defined in \eqref{eq-fde-ivp-1}, \eqref{eq-fde-ivp-ic-1} has a unique solution on $ [0, a] $.
\end{proof}

Note that although existence is resolved (by virtue of the solutions given above)
for the differential equations in (\ref{eq:fde-1}, \ref{eq:fde-1-ic}) and \ref{eq:fde-multi-order}, this 
guarentees uniqueness on some closed interval starting at $ 0 $ for both cases. Its also important
to note that this result can be extended to differential equations involving Riemann-Liouville derivatives, by 
virtue of the correspondence between the Caputo derivative and the Riemann-Liouville derivative \cite{Podlubny1999}. 
