\section{Existance and Uniquness Results for Ordinary Fractional Differential Equations}
In this section we establish some analytical results for ordinary fractional differential equations. These results allow us to solve some simple initial value problems and establish a theoretical framework from which results in other sections can draw.

We will first outline a set of ideas which can guarentee the existence of a solution to an initial value problem involving fractional derivatives. These ideas are outlined in \cite{Tisdell2012} but we present these ideas here because they form the basis for a generalisation we would like to present in theorem \ref{thm-existence-uniq}.

\begin{mdframed}[innertopmargin=10pt]
\begin{propdef}
    Let $ \beta > 0, \alpha > 0 $ and $ a > 0 $. For $ x,y \in C([0,a]) $ the function
    \begin{align}
        d_{\beta}(x,y) := \max_{t \in [0,a]} \frac{|x(t) - y(t)|}{E_{\alpha,1}(\beta t^\alpha)}
    \end{align}
    defines a metric on the space of functions $ C([0,a]) $. 
\end{propdef}
\end{mdframed}

This metric, as presented by Tisdell, \cite{Tisdell2012}, is a generalisation of the Bielecki metric \cite{Tisdell2012, Bielecki1956}. 

\begin{proof}
    It is well known that $ d(x,y) = \max_{t \in [0,a]}|x(t) - y(t)| $ defines a metric on $C([0,a]) $. Now it is easy to see that for $ t > 0 $, $ E_{\alpha,1}(\beta t^\alpha) >0 $. It is also well known that $ E_{\alpha,1}(t) $ is continuous on $ [0, a] $ so it follows that $ d_{\beta}(x,y) $ is a metric.
\end{proof}

\begin{mdframed}[innertopmargin=10pt]
\begin{lemma}
    The metric space $ (C([0,a]), d_\beta) $ is complete.
\end{lemma}
\end{mdframed}
\begin{proof}
    We only have to show that $ d_\beta $ is equivelent to the max metric ($ d(x,y) = \max_{t \in [0,a]}|x(t) - y(t)| $) to get this result because it is well known that $ (C([0,a]), d) $ is a complete metric space.
    It is easy to see that $ E_{\alpha, 1}(\beta t^\alpha) $ is strictly increasing on $ [0, a] $ and so 
    \begin{align}
        \frac{1}{E_{\alpha, 1}(\beta t^\alpha)} d(x,y) \leq d_\beta(x,y) \leq d(x,y).
    \end{align}
    This shows that the metrics are equivelent and hence $ (C([0,a]), d_\beta) $ is complete.
\end{proof}

Using these definitions and results we can now say something about the existance and uniqueness of solutions to an initial value problem. The following lemma is from Tisdell \cite{Tisdell2012}.

\begin{mdframed}[innertopmargin=10pt]
\begin{theorem}[Uniqueness]
    \label{thm:unique-tisdell}
    Consider the initial value problem 
    \begin{align}
        \label{eq:ivp_uniq_1}    
        \prescript{C}{0}{\mathcal{D}}x(t) = f(t,x(t))
    \end{align}
    \begin{align}
        \label{eq:ivp_uniq_ic_1}
        x(0) = a_0, x'(0) = a_1, \ldots, x^{(n-1)}(0) = a_{n-1}.
    \end{align}
    Let 
	\begin{align}
	    S:= \{ (t,p) \in \Rl^2 : t \in [0, a], p \in \Rl \} 
	\end{align}
	and let $ f : S \lra \mathcal{R} $ be continuous. If there is a positive constant $ L $ such that 
	\begin{align}
	    |f(t,u) - f(t,v)| \leq L|u-v| 
	\end{align}   
	for all $ (t,u), (t,v) \in S $
	then \eqref{eq:ivp_uniq_1} and \eqref{eq:ivp_uniq_ic_1} has a unique solution on $ [0, a] $.
\end{theorem}
\end{mdframed}
\begin{proof}
    TODO
\end{proof}

This generalizes the result above by Tisdell \cite{Tisdell2012} but a similar result for Miller-Ross sequential
fractional differential equations can be found in \cite{Podlubny1999}.

\begin{mdframed}[innertopmargin=10pt]
\begin{theorem}[Uniqueness]
\label{thm-existence-uniq}
Consider the following intial value problem,

	\begin{align}
		\label{eq-fde-ivp-1}
		\sum_{j=1}^{N} \beta_j\capder{0}{t}{\alpha_j}{x}(t) = f(t,x(t)) \\
		\label{eq-fde-ivp-ic-1}
		x(0) = A_0, x_1(0) = A_1, \ldots, x^{n_N}(0) = A_{n_N}
	\end{align}
	where $ \alpha_1 > \alpha_2 > \ldots > \alpha_N $.
	
	Define
		\begin{align}
		 S:= \{ (t,p) \in \Rl^2 : t \in [0, a], p \in \Rl \} 
		\end{align}
	Let $ f : S \lra \Rl $ be continuous. If there is a positive constant L such that 
		\begin{align}
		\label{eq:uniq-lipshitz}
		|f(t,u) - f(t,v)| \leq L|u-v|, \text{ for all } (t,u), (t,v) \in S
		\end{align}
and the set of constants $ \{ \alpha_j \}_{j = 1}^{N} $, $ \{ \beta_j \}_{j=1}^N $
such that
	\begin{align}
	    \label{eq:fde-uniq-cond}
		\sum_{j=2}^N |\beta_j| a^{\alpha_1 - \alpha_j} \frac{E_{\alpha_1,1+\alpha_1-\alpha_j}(\gamma t^{\alpha_1})}{E_{\alpha_1,1}(\gamma t^{\alpha_1})} < 1
	\end{align}
	then the intial value problem defined in \ref{eq-fde-ivp-1} and \ref{eq-fde-ivp-ic-1} has a unique solution on $ [0, a] $.
\end{theorem}
\end{mdframed}
To prove this we will need several lemmas. 
\begin{mdframed}[innertopmargin=10pt]
\begin{lemma}
	The IVP defined in \eqref{eq-fde-ivp-1}, \eqref{eq-fde-ivp-ic-1} is equivalent to the integral equation
	\begin{align}
		x(t) &= \sum_{k=1}^{n_1}\frac{A_kt^k}{k!} + \frac{1}{\beta_1} \Bigl( \frac{1}{\Gamma(\alpha_1)}\int_{0}^{t} (t-s)^{\alpha_1 - 1}f(s,x(s))ds \\
			& \ \ \ - \sum_{j=2}^{N}\beta_j \frac{1}{\Gamma(\alpha_1 - \alpha_j)}
			\int_{0}^{t}(t-s)^{\alpha_1 - \alpha_j - 1}\left(x(s) - \sum_{k=1}^{n_j}\frac{A_ks^k}{k!} \right) ds \Bigr)
	\end{align}
\end{lemma}
\end{mdframed}
\begin{proof}
	Apply $ I_0^\alpha $ to both sides of \eqref{eq-fde-ivp-1} and then apply lemma \ref{lem:rli-cap}.
\end{proof}

\begin{mdframed}[innertopmargin=10pt]
\begin{lemma}
\label{lem-rli-mit-lef-1}
For $ \xi > 0 $, $ \gamma > 0 $ and $ \alpha > 0 $ we have that
	\begin{align}
		{I}^\xi_0 E_{\alpha,1}(\gamma t^\alpha) = t^\xi E_{\alpha,1+\xi}(\gamma t^\alpha).
	\end{align}
\end{lemma}
\end{mdframed}
\begin{proof}
	This can be proved rather easily with Laplace transform methods.
	By using the results of lemmas \ref{lem:rli_laplace} and \ref{lem:lap_mit} we have that
	\begin{align}
	    \mathcal{L}\{ I_0^\xi E_{\alpha, 1}(\gamma t^\alpha) \} = \frac{s^{\alpha-\xi-1}}{s^\alpha - \gamma}
	\end{align}
	then by inverting the with the help of lemma \ref{lem:lap_mit_2} we have the result.
\end{proof}

\begin{mdframed}[innertopmargin=10pt]
\begin{lemma}	
\label{lem-rli-mit-lef-2}
For $ \gamma > 0 $ and $ \alpha > 0 $ we have that
	\begin{align}
		I_0^\alpha E_{\alpha,1}(\gamma t^\alpha) = \frac{1}{\gamma} \left( E_{\alpha,1}(\gamma t^\alpha) - 1 \right).
	\end{align}
\end{lemma}
\end{mdframed}
\begin{proof}
	Again this is easy to prove with in Laplace space.
	By using the results of lemmas \ref{lem:rli_laplace} and \ref{lem:lap_mit} we have that
	\begin{align}
    	\mathcal{L}\{ I_0^\alpha E_{\alpha, 1}(\gamma t^\alpha) \} = \frac{s^{-1}}{s^\alpha - \gamma}
	\end{align}
	also note that
	\begin{align}
	    \mathcal{L}\left\{ \frac{1}{\gamma}( E_{\alpha, 1}(\gamma t^\alpha) - 1) \right\} &= \frac{1}{\gamma} \left( \frac{s^{\alpha-1}}{s^\alpha - \gamma} - \frac{1}{s} \right) \\
	    &= \frac{1}{\gamma}\left( \frac{s^\alpha - s^\alpha + \gamma}{s(s^\alpha - \gamma)}\right) \\
	    &= \frac{1}{\gamma}\left(\frac{\gamma s^{-1}}{s^\alpha - \gamma}\right) \\
	    &= \frac{s^{-1}}{s^\alpha - \gamma}
	\end{align}
	and so it is clear that
	\begin{align}
I_0^\alpha E_{\alpha,1}(\gamma t^\alpha) = \frac{1}{\gamma} \left( E_{\alpha,1}(\gamma t^\alpha) - 1 \right).
	\end{align}
\end{proof}


\begin{proof}[Proof of theorem \ref{thm-existence-uniq}]

	To arrive at this we only have to prove that the map
	\begin{align}
		[Fx](t) &:= \sum_{k=1}^{n_1}\frac{A_kt^k}{k!} + \frac{1}{\beta_1} \Bigl( \frac{1}{\Gamma(\alpha_1)}\int_{0}^{t} (t-s)^{\alpha_1 - 1}f(s,x(s))ds \\
			& \ \ \ - \sum_{j=2}^{N} \frac{\beta_j}{\Gamma(\alpha_1 - \alpha_j)}\int_{0}^{t}(t-s)^{\alpha_1 - \alpha_j - 1}\left(x(s) - \sum_{k=1}^{n_j}\frac{A_ks^k}{k!} \right) ds \Bigr)
	\end{align}
	is contractive in the metric space $ \left( C[0,a], d^{\alpha_1}_\gamma \right) $ where 
	$$ d_\gamma^{\alpha_1}(x,y) = \max_{t \in [0, a]} \frac{|x(t) - y(t)|}{E_{\alpha_1,1}(\gamma t^{\alpha_1})}. $$
	To see this note that
	\begin{align}
		d_\gamma^{\alpha_1}(Fx,Fy) &= \max_{t \in [0, a]}  \frac{1}{E_{\alpha_1}(\gamma t^{\alpha_1})} 
			\left| \frac{1}{\beta_1} \right| \Bigl| \frac{1}{\Gamma(\alpha_1)}\int_0^t (t-s)^{\alpha_1 - 1} (f(s,x(s)) - f(s,y(s))ds \\ 
			& \ \ \ - \sum_{j=2}^N \frac{\beta_j}{\Gamma(\alpha_1 - \alpha_j)} \int_0^t (t-s)^{\alpha_1 - \alpha_j - 1}(x(s) - y(s)) ds \Bigr| \\
			&\leq \max_{t \in [0, a]} \frac{1}{E_{\alpha_1}(\gamma t^{\alpha_1}) | \beta_1 |} \Big(
			 \frac{1}{\Gamma(\alpha_1)}\int_0^t (t-s)^{\alpha_1 - 1} |f(s,x(s)) - f(s,y(s))|ds \\ 
			& \ \ \ + \sum_{j=2}^N \frac{|\beta_j|}{\Gamma(\alpha_1 - \alpha_j)} \int_0^t (t-s)^{\alpha_1 - \alpha_j - 1}|x(s) - y(s))| ds \Bigr).
	\end{align}
	By exploiting the Lipshitz condition \eqref{eq:uniq-lipshitz} we can further write that 
	\begin{align}
		d_\gamma^{\alpha_1}(Fx,Fy) &\leq \max_{t \in [0, a]} \frac{1}{E_{\alpha_1}(\gamma t^{\alpha_1})|\beta_1|} \Bigl(
			\frac{L}{\Gamma(\alpha_1)}\int_0^t (t-s)^{\alpha_1 - 1} |x(s) - y(s)|ds \\ 
			& \ \ \ + \sum_{j=2}^N \frac{|\beta_j|}{\Gamma(\alpha_1 - \alpha_j)} \int_0^t (t-s)^{\alpha_1 - \alpha_j - 1}|x(s) - y(s))| ds \Bigr) \\
			&= \max_{t \in [0, a]} \frac{1}{E_{\alpha_1}(\gamma t^{\alpha_1})|\beta_1|} \Bigl(
			\frac{L}{\Gamma(\alpha_1)}\int_0^t (t-s)^{\alpha_1 - 1} \frac{|x(s) - y(s)|}{E_{\alpha_1}(\gamma s^{\alpha_1})}E_{\alpha_1}(\gamma s^{\alpha_1})ds \\ 
			& \ \ \ + \sum_{j=2}^N \frac{|\beta_j|}{\Gamma(\alpha_1 - \alpha_j)} \int_0^t (t-s)^{\alpha_1 - \alpha_j - 1}\frac{|x(s) - y(s))|}{E_{\alpha_1}(\gamma s^{\alpha_1})}E_{\alpha_1}(\gamma s^{\alpha_1}) ds \Bigr) \\
			&\leq d_\gamma^{\alpha_1}(x,y) \max_{t \in [0, a]} \frac{1}{E_{\alpha_1}(\gamma t^{\alpha_1})|\beta_1|} \Bigl(
			\frac{L}{\Gamma(\alpha_1)}\int_0^t (t-s)^{\alpha_1 - 1} E_{\alpha_1}(\gamma s^{\alpha_1}) ds \\
			& \ \ \ + \sum_{j=2}^N \frac{|\beta_j|}{\Gamma(\alpha_1 - \alpha_j)} \int_0^t (t-s)^{\alpha_1 - \alpha_j - 1}E_{\alpha_1}(\gamma s^{\alpha_1}) ds \Bigr) \\
			&= d_\gamma^{\alpha_1}(x,y) \max_{t \in [0, a]} \frac{1}{E_{\alpha_1}(\gamma t^{\alpha_1})|\beta_1|} \Bigl(
			L \rli{0}{\alpha_1}{E_{\alpha_1}(\gamma t^{\alpha_1}} \\
			& \ \ \ + \sum_{j=2}^N |\beta_j| \rli{0}{\alpha_1 - \alpha_j}{E_{\alpha_1}(\gamma t^{\alpha_1})} \Bigr). \\
	\end{align}
	We can now use the results of lemmas \ref{lem-rli-mit-lef-1} and \ref{lem-rli-mit-lef-2} to write
	\begin{align}
		d_\gamma^{\alpha_1}(Fx,Fy) &\leq d_\gamma^{\alpha_1}(x,y) \max_{t \in [0, a]} \frac{1}{E_{\alpha_1, 1}(\gamma t^{\alpha_1})|\beta_1|} \Bigl(
			\frac{L}{\gamma}\left( E_{\alpha_1,1}(\gamma t^{\alpha_1}) - 1 \right) \\
			& \ \ \ + \sum_{j=2}^N |\beta_j| t^{\alpha_1 - \alpha_j} E_{\alpha_1, 1 + \alpha_1 - \alpha_j}(\gamma t^{\alpha_1}) \Bigr) \\
			&= d_\gamma^{\alpha_1}(x,y) \max_{t \in [0, a]} \frac{1}{|\beta_1|} \Bigl(
			\frac{L}{\gamma}\left( 1- \frac{1}{E_{\alpha_1}(\gamma t^{\alpha_1})} \right) + \sum_{j=2}^N |\beta_j| t^{\alpha_1 - \alpha_j}\frac{E_{\alpha_1,1+\alpha_1-\alpha_j}(\gamma t^{\alpha_1})}{E_{\alpha_1,1}(\gamma t^{\alpha_1})}\Bigr) \\
	\end{align}
	and finally we get that 
	\begin{align}
		d_\gamma^{\alpha_1}(Fx,Fy) &\leq d_\gamma^{\alpha_1}(x,y) \frac{1}{|\beta_1|}\left( \frac{L}{\gamma} + \sum_{j=2}^N |\beta_j| a^{\alpha_1 - \alpha_j} \frac{E_{\alpha_1,1+\alpha_1-\alpha_j}(\gamma t^{\alpha_1})}{E_{\alpha_1,1}(\gamma t^{\alpha_1})} \right).
	\end{align}
	By choosing $ \gamma $ sufficiently large we get that 
	$$
		\frac{1}{|\beta_1|}\left( \frac{L}{\gamma} + \sum_{j=2}^N |\beta_j| a^{\alpha_1 - \alpha_j} \frac{E_{\alpha_1,1+\alpha_1-\alpha_j}(\gamma t^{\alpha_1})}{E_{\alpha_1,1}(\gamma t^{\alpha_1})} \right) < 1
	$$
	and so $ F $ is a contractive mapping and thus the IVP defined in \eqref{eq-fde-ivp-1}, \eqref{eq-fde-ivp-ic-1} has a unique solution on $ [0, a] $.
\end{proof}

Note that although existence is resolved (by virtue of the solutions given above)
for the differential equations in (\ref{eq:fde-1}, \ref{eq:fde-1-ic}) and \ref{eq:fde-multi-order}, this 
guarentees uniqueness on some closed interval starting at $ 0 $ for both cases. Its also important
to note that this result can be extended to differential equations involving Riemann-Liouville derivatives, by 
virtue of the correspondence between the Caputo derivative and the Riemann-Liouville derivative \cite{Podlubny1999}. 

This result could be improved by removing the condition  in \eqref{eq:fde-uniq-cond}. This would could be achieved if the following conjecture holds true.
\begin{mdframed}[innertopmargin=10pt]
\begin{conjecture}
    For $ \alpha, \beta, \varepsilon > 0 $ we conjecture that
    \begin{align}
        \lim_{z\lra \infty} \frac{E_{\alpha, \beta + \varepsilon}(z)}{E_{\alpha, \beta}(z)} = 0.
    \end{align}
\end{conjecture}
\end{mdframed}
\emph{Justification}
Proving this is has proved rather tricky. The figure \ref{fig:mittag-quotient} shows that for $ \alpha = 1, \beta = 1 $ and $ \varepsilon = 0.1 $ it appears to hold true. It should be noted that even for relatively large values of $ \varepsilon $m ( $ \varepsilon = 0.1 $ ) $, z $ needs to be extremely large, ($ z \sim 10^{14} $), for the quotient to get within $ 0.04 $ of $ 0 $. Even if this result does not hold a weaker result which states that the limit approaches a non-zero constant would considerably simplify the condition in \eqref{eq:fde-uniq-cond}.
\begin{figure}[H]
\includegraphics[scale=0.6]{images/Mittag-Leffler-Quotient}
\caption{$ E_{\alpha, \beta + \varepsilon}(z) / E_{\alpha, \beta}(z) $ for $ \alpha = 1, \beta = 1 $ and $ \varepsilon = 0.1 $.}
\label{fig:mittag-quotient}
\end{figure}

Both theorems \ref{thm:unique-tisdell} and \ref{thm-existence-uniq} are important as they give us some guarentees about the existance and uniqueness of solutions the initial value problems involving fractional derivatives. The generalisation in \ref{thm-existence-uniq} is important as it could be applied to showing the existance of solutions equations like the Bagley-Torvick equation which is a multi-order fractional differential equation that models the motion of a plate in a viscus fluid \cite{Diethelm2002-3, Podlubny1999, Torvik1984}.

In fact these results can be taken quite a bit further. The proof techniques used in theorems \ref{thm:unique-tisdell} and \ref{thm-existence-uniq} suggest a constructive method for arriving at a solution to fractional differential equations. This is formalised by Tisdell \cite{Tisdell2012} through the method of successive approximations. A similar idea is also outlined in \cite{Podlubny1999}.

\clearpage

