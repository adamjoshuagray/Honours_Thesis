\section{Analytical Results for Ordinary Fractional Differential Equations}
%\addcontentsline{toc}{section}{Solution to a Simple Fractional Differential Equation}
In this section we establish some analytical results for ordinary fractional differential equations. These results allow us to solve some simple initial value problems and establish a theoretical framework from which results in other sections can draw.



\subsection{Existence and uniquness of solutions to initial value problems}
\label{sec:existence_uniq}
%\addcontentsline{toc}{section}{Existence and Uniquness of Fractional Differential Equations}
After looking at the solution to a couple of fractional differential equations 
we wish to consider the existence an uniqueness of solutions to a class fractional differential equations. 
This generalizes a result and technique of Tisdell \cite{Tisdell2012} but a similar result for Miller-Ross sequential
fractional differential equations can be found in \cite{Podlubny1999}.

\begin{mdframed}[innertopmargin=10pt]
\begin{theorem}[Uniqueness]
\label{thm-existence-uniq}
Consider the following intial value problem,

	\begin{align}
		\label{eq-fde-ivp-1}
		\sum_{j=1}^{N} \beta_j\capder{0}{t}{\alpha_j}{x}(t) = f(t,x(t)) \\
		\label{eq-fde-ivp-ic-1}
		x(0) = A_0, x_1(0) = A_1, \ldots, x^{n_N}(0) = A_{n_N}
	\end{align}
	where $ \alpha_1 > \alpha_2 > \ldots > \alpha_N $, 
	$ n_j = \lceil \alpha_j \rceil - 1 $ and $ \beta_1 = 1 $.
	
	Define
		$$ S:= \{ (t,p) \in \Rl^2 : t \in [0, a], p \in \Rl \} $$
	Let $ f : S \lra \Rl $ be continuous. If there is a positive constant L such that 
		$$ |f(t,u) - f(t,v)| \leq L|u-v|, \text{ for all } (t,u), (t,v) \in S $$
and the set of constants $ \{ \alpha_j \}_{j = 1}^{N} $, $ \{ \beta_j \}_{j=1}^N $
such that
	$$
		\sum_{j=2}^N \left|\frac{\beta_j}{\beta_1}\right| a^{\alpha_1 - \alpha_j} < 1
	$$
	then the intial value problem defined in \ref{eq-fde-ivp-1} and \ref{eq-fde-ivp-ic-1} has a unique solution.
\end{theorem}
\end{mdframed}
To prove this we will need several lemmas. 
\begin{mdframed}[innertopmargin=10pt]
\begin{lemma}
	The IVP defined in \eqref{eq-fde-ivp-1}, \eqref{eq-fde-ivp-ic-1} is equivalent to the integral equation
	\begin{align}
		x(t) &= \sum_{k=1}^{n_1}\frac{A_kt^k}{k!} + \frac{1}{\beta_1} \Bigl( \frac{1}{\Gamma(\alpha_1)}\int_{0}^{t} (t-s)^{\alpha_1 - 1}f(s,x(s))ds \\
			& \ \ \ - \sum_{j=2}^{N}\beta_j \frac{1}{\Gamma(\alpha_1 - \alpha_j)}
			\int_{0}^{t}(t-s)^{\alpha_1 - \alpha_j - 1}\left(x(s) - \sum_{k=1}^{n_j}\frac{A_ks^k}{k!} \right) ds \Bigr)
	\end{align}
\end{lemma}
\end{mdframed}
\begin{proof}
	Apply $ \rli{0}{\alpha}{} $ to both sides of \eqref{eq-fde-ivp-1} and recognize that
	$$
		\rli{0}{\alpha}{\capder{0}{t}{\alpha}{x}}(t) = x(t) + \sum_{k=0}^{n} \frac{x^{(k)}(0)t^{k}}{k!}
	$$
	where $ n = \lceil \alpha \rceil - 1 $.
\end{proof}

\begin{mdframed}[innertopmargin=10pt]
\begin{lemma}	
\label{lem-rli-mit-lef-1}
	\begin{align}
		\rli{0}{\xi}{E_\alpha(\gamma t^\alpha)} \leq t^\xi E_\alpha(\gamma t^\alpha)
	\end{align}
\end{lemma}
\end{mdframed}
\begin{proof}
	See that
	\begin{align}
		\rli{0}{\xi}{E_\alpha(\gamma t^\alpha)} &= \frac{1}{\Gamma(\xi)} \int_0^t E_\alpha(\gamma s^\alpha)(t-s)^{\xi - 1} ds \\
			&= \frac{1}{\Gamma(\xi)} \int_0^t \sum_{k=0}^\infty \frac{\gamma^k s^{\alpha k}}{\Gamma(\alpha k + 1)} (t-s)^{\xi - 1} ds \\
			&= \frac{1}{\Gamma(\xi)} \sum_{k=0}^\infty \frac{\gamma^k}{\Gamma(\alpha k + 1)} \underbrace{\int_0^t s^{\alpha k}(t-s)^{\xi - 1} ds}_\circledast.
	\end{align}
	Letting $ \tau = \frac{s}{t} $ we have that 
	\begin{align}
		\circledast &= \int_0^1 (t\tau)^{\alpha k} (t - t\tau)^{\xi - 1} t d\tau \\
			&= t^{\alpha k + \xi}\int_0^1 (\tau)^{\alpha k} (1 - 1\tau)^{\xi - 1} d\tau \\
			&= t^{\alpha k + \xi} B(\alpha k + 1, \xi) \\
			&= t^{\alpha k + \xi} \frac{\Gamma(\alpha k + 1) \Gamma(\xi)}{\Gamma(\alpha k + \xi + 1)}.
	\end{align}
	This means that 
	\begin{align}
		\rli{0}{\xi}{E_\alpha(\gamma t^\alpha)} &= \sum_{k=0}^\infty \frac{\gamma^k t^{\alpha k + \xi}}{\Gamma(\alpha k + \xi + 1)} \\
			&= t^{\xi}\sum_{k=0}^\infty\frac{\gamma^k t^{\alpha k}}{\Gamma(\alpha k + \xi + 1)} \\
			&\leq t^{\xi}\sum_{k=0}^\infty\frac{\gamma^k t^{\alpha k}}{\Gamma(\alpha k + 1)} \\
			&= t^{\xi} E_\alpha(\gamma t^\alpha).
	\end{align}
\end{proof}

\begin{mdframed}[innertopmargin=10pt]
\begin{lemma}	
\label{lem-rli-mit-lef-2}
	\begin{align}
		\rli{0}{\alpha}{E_\alpha(\gamma t^\alpha)} = \frac{1}{\gamma} \left( E_\alpha(\gamma t^\alpha) - 1 \right)
	\end{align}
\end{lemma}
\end{mdframed}
\begin{proof}
	See that
	\begin{align}
		\rli{0}{\alpha}{E_\alpha(\gamma t^\alpha)} &= \frac{1}{\Gamma(\alpha)} \int_0^t E_\alpha (\gamma s^\alpha)(t-s)^{\alpha - 1} ds \\
			&= \frac{1}{\Gamma(\alpha)} \sum_{k=0}^\infty \frac{\gamma^k}{\Gamma(\alpha k + 1)} \underbrace{\int_0^t s^{\alpha k} (t-s)^\alpha ds}_{\circledast}.
	\end{align}

	Letting $ \tau = \frac{s}{t} $ we have that 
	\begin{align}
		\circledast &= \int_0^1 (t\tau)^{\alpha k}(t-t\tau)^{\alpha - 1} t d\tau \\
			&= t^{\alpha (k + 1)}\int_0^1 \tau^{\alpha k}(1-\tau)^{\alpha - 1} d\tau \\
			&= t^{\alpha (k + 1)}B(\alpha k + 1, \alpha) \\
			&= t^{\alpha (k + 1)} \frac{\Gamma(\alpha k + 1) \Gamma(\alpha)}{\Gamma(\alpha(k + 1) + 1)}.
	\end{align}
	This then means that 
	
	\begin{align}
		\rli{0}{\alpha}{E_\alpha(\gamma t^\alpha)} &= \sum_{k=0}^\infty \frac{\gamma^k t^{\alpha(k+1)}}{\Gamma(\alpha(k + 1) + 1)} \\
			&= \frac{1}{\gamma}\sum_{k=1}^\infty \frac{\gamma^k t^{\alpha k}}{\Gamma(\alpha k+ 1)} \\
			&= \frac{1}{\gamma}\left( \sum_{k=0}^\infty \frac{\gamma^k t^{\alpha k}}{\Gamma(\alpha k+ 1)} - 1\right) \\
			&= \frac{1}{\gamma}\left( E_\alpha(\gamma t^\alpha) - 1 \right).
	\end{align}
\end{proof}


\begin{proof}[Proof of theorem \ref{thm-existence-uniq}]

	To arrive at this we only have to prove that the map
	\begin{align}
		[Fx](t) &:= \sum_{k=1}^{n_1}\frac{A_kt^k}{k!} + \frac{1}{\beta_1} \Bigl( \frac{1}{\Gamma(\alpha_1)}\int_{0}^{t} (t-s)^{\alpha_1 - 1}f(s,x(s))ds \\
			& \ \ \ - \sum_{j=2}^{N} \frac{\beta_j}{\Gamma(\alpha_1 - \alpha_j)}\int_{0}^{t}(t-s)^{\alpha_1 - \alpha_j - 1}\left(x(s) - \sum_{k=1}^{n_j}\frac{A_ks^k}{k!} \right) ds \Bigr)
	\end{align}
	is contractive in the metric space $ \left( C[0,a], d^{\alpha_1}_\gamma \right) $ where 
	$$ d_\gamma^{\alpha_1}(x,y) = \max_{t \in [0, a]} \frac{|x(t) - y(t)|}{E_{\alpha_1}(\gamma t^{\alpha_1})}. $$
	To see this note that
	\begin{align}
		d_\gamma^{\alpha_1}(Fx,Fy) &= \max_{t \in [0, a]}  \frac{1}{E_{\alpha_1}(\gamma t^{\alpha_1})} 
			\left| \frac{1}{\beta_1} \right| \Bigl| \frac{1}{\Gamma(\alpha_1)}\int_0^t (t-s)^{\alpha_1 - 1} (f(s,x(s)) - f(s,y(s))ds \\ 
			& \ \ \ - \sum_{j=2}^N \frac{\beta_j}{\Gamma(\alpha_1 - \alpha_j)} \int_0^t (t-s)^{\alpha_1 - \alpha_j - 1}(x(s) - y(s)) ds \Bigr| \\
			&\leq \max_{t \in [0, a]} \frac{1}{E_{\alpha_1}(\gamma t^{\alpha_1}) | \beta_1 |} \Big(
			 \frac{1}{\Gamma(\alpha_1)}\int_0^t (t-s)^{\alpha_1 - 1} |f(s,x(s)) - f(s,y(s))|ds \\ 
			& \ \ \ + \sum_{j=2}^N \frac{|\beta_j|}{\Gamma(\alpha_1 - \alpha_j)} \int_0^t (t-s)^{\alpha_1 - \alpha_j - 1}|x(s) - y(s))| ds \Bigr).
	\end{align}
	By exploiting the Lipshitz condition we can further write that 
	\begin{align}
		d_\gamma^{\alpha_1}(Fx,Fy) &\leq \max_{t \in [0, a]} \frac{1}{E_{\alpha_1}(\gamma t^{\alpha_1})|\beta_1|} \Bigl(
			\frac{L}{\Gamma(\alpha_1)}\int_0^t (t-s)^{\alpha_1 - 1} |x(s) - y(s)|ds \\ 
			& \ \ \ + \sum_{j=2}^N \frac{|\beta_j|}{\Gamma(\alpha_1 - \alpha_j)} \int_0^t (t-s)^{\alpha_1 - \alpha_j - 1}|x(s) - y(s))| ds \Bigr) \\
			&= \max_{t \in [0, a]} \frac{1}{E_{\alpha_1}(\gamma t^{\alpha_1})|\beta_1|} \Bigl(
			\frac{L}{\Gamma(\alpha_1)}\int_0^t (t-s)^{\alpha_1 - 1} \frac{|x(s) - y(s)|}{E_{\alpha_1}(\gamma s^{\alpha_1})}E_{\alpha_1}(\gamma s^{\alpha_1})ds \\ 
			& \ \ \ + \sum_{j=2}^N \frac{|\beta_j|}{\Gamma(\alpha_1 - \alpha_j)} \int_0^t (t-s)^{\alpha_1 - \alpha_j - 1}\frac{|x(s) - y(s))|}{E_{\alpha_1}(\gamma s^{\alpha_1})}E_{\alpha_1}(\gamma s^{\alpha_1}) ds \Bigr) \\
			&\leq d_\gamma^{\alpha_1}(x,y) \max_{t \in [0, a]} \frac{1}{E_{\alpha_1}(\gamma t^{\alpha_1})|\beta_1|} \Bigl(
			\frac{L}{\Gamma(\alpha_1)}\int_0^t (t-s)^{\alpha_1 - 1} E_{\alpha_1}(\gamma s^{\alpha_1}) ds \\
			& \ \ \ + \sum_{j=2}^N \frac{|\beta_j|}{\Gamma(\alpha_1 - \alpha_j)} \int_0^t (t-s)^{\alpha_1 - \alpha_j - 1}E_{\alpha_1}(\gamma s^{\alpha_1}) ds \Bigr) \\
			&= d_\gamma^{\alpha_1}(x,y) \max_{t \in [0, a]} \frac{1}{E_{\alpha_1}(\gamma t^{\alpha_1})|\beta_1|} \Bigl(
			L \rli{0}{\alpha_1}{E_{\alpha_1}(\gamma t^{\alpha_1}} \\
			& \ \ \ + \sum_{j=2}^N |\beta_j| \rli{0}{\alpha_1 - \alpha_j}{E_{\alpha_1}(\gamma t^{\alpha_1})} \Bigr). \\
	\end{align}
	We can now use the results of lemmas \ref{lem-rli-mit-lef-1} and \ref{lem-rli-mit-lef-2} to write
	\begin{align}
		d_\gamma^{\alpha_1}(Fx,Fy) &\leq d_\gamma^{\alpha_1}(x,y) \max_{t \in [0, a]} \frac{1}{E_{\alpha_1}(\gamma t^{\alpha_1})|\beta_1|} \Bigl(
			\frac{L}{\gamma}\left( E_{\alpha_1}(\gamma t^{\alpha_1}) - 1 \right) \\
			& \ \ \ + \sum_{j=2}^N |\beta_j| t^{\alpha_1 - \alpha_j} E_{\alpha_1}(\gamma t^{\alpha_1}) \Bigr) \\
			&= d_\gamma^{\alpha_1}(x,y) \max_{t \in [0, a]} \frac{1}{|\beta_1|} \Bigl(
			\frac{L}{\gamma}\left( 1- \frac{1}{E_{\alpha_1}(\gamma t^{\alpha_1})} \right) + \sum_{j=2}^N |\beta_j| t^{\alpha_1 - \alpha_j}\Bigr) \\
	\end{align}
	and finally we get that 
	\begin{align}
		d_\gamma^{\alpha_1}(Fx,Fy) &\leq d_\gamma^{\alpha_1}(x,y) \frac{1}{|\beta_1|}\left( \frac{L}{\gamma} + \sum_{j=2}^N |\beta_j| a^{\alpha_1 - \alpha_j} \right).
	\end{align}
	By choosing $ \gamma $ sufficiently large we get that 
	$$
		\frac{1}{|\beta_1|}\left( \frac{L}{\gamma} + \sum_{j=2}^N |\beta_j| a^{\alpha_1 - \alpha_j} \right) < 1
	$$
	and so $ F $ is a contractive mapping and thus the IVP defined in \eqref{eq-fde-ivp-1}, \eqref{eq-fde-ivp-ic-1} has a unique solution on $ [0, a] $.
\end{proof}

Note that although existence is resolved (by virtue of the solutions given above)
for the differential equations in (\ref{eq:fde-1}, \ref{eq:fde-1-ic}) and \ref{eq:fde-multi-order}, this 
guarentees uniqueness on some closed interval starting at $ 0 $ for both cases. Its also important
to note that this result can be extended to differential equations involving Riemann-Liouville derivatives, by 
virtue of the correspondence between the Caputo derivative and the Riemann-Liouville derivative \cite{Podlubny1999}. 
