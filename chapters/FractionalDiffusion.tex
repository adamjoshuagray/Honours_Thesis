\section{Fractional Diffusion}
In this last section we would like to take a look at fractional diffusion. This is one of the main areas where fractional calculus is getting heavily used today. Modeling over the last two decades has shown that it is extremely important for explaining anomylous diffusion in beiological contexts. It also appears to play a role in diffusion in porus media and has extensions into physics and chemistry. It is perhaps now, and in the field of fractional diffusion, that fractional calculus is finally being pulled out of the cuboard of pure mathematics and Liebniz' fabeled \emph{ueseful consequences} are finally being drawn from this field.

Fractional diffusion is particularly interesting in that it is not just a phenomenological model which appears to fit with real world data better simply by virtue of its reduced parsemony. It can actually be explained by some deeper and very justifiable physical principles. In order to give a good account of these phsycial ideas and in some sense \emph{derive} fractional diffusion we will start with ordinary diffusion and then work our way to fractional diffusion by considering continuous time random walks.

We will examine diffusion models with just one spacial dimension. The generalisation to multiple spacial dimeansions is straight forward and only serves to add unnecessary notational complexity.

\subsection{Standard Diffusion}
\subsubsection{Derivation}
Starting with a discrete time, discrete step (unbiased) random walk we can arrive at a \emph{continium} approximation the random walk as follows. 

If $ P(x,t) $ is the probability of being at location $ x $ at time $ t $ we have that
\begin{align}
    \label{eq:rw_standard}
    P(x,t) = \frac{1}{2}P(x-\Delta x, t - \Delta t) + \frac{1}{2}P(x + \Delta x, t - \Delta t)
\end{align}
where $ \Delta x $ is the spacial step size and $ \Delta t $ is the time step size.
Now taking Taylor expansions we get that
\begin{align}
    \label{eq:rw_pos}
    P(x + \Delta x, t - \Delta t) &= P(x,t) + \Delta x \frac{\partial P}{\partial x} - \Delta t \frac{\partial P}{\partial t} + \frac{(\Delta x)^2}{2} \frac{\partial^2 P}{\partial x^2} + \frac{(\Delta t)^2}{2} \frac{\partial^2 P}{\partial t^2} \\ 
    & \ \ \ - \Delta t \Delta x \frac{\partial^2 P}{\partial x \partial t} + O((\Delta t)^3) + O(( \Delta x)^3) \nonumber
\end{align}
and
\begin{align}
    \label{eq:rw_neg}
    P(x - \Delta x, t - \Delta t) &= P(x,t) - \Delta x \frac{\partial P}{\partial x} - \Delta t \frac{\partial P}{\partial t} + \frac{(\Delta x)^2}{2} \frac{\partial^2 P}{\partial x^2} + \frac{(\Delta t)^2}{2} \frac{\partial^2 P}{\partial t^2} \\ 
    & \ \ \ + \Delta t \Delta x \frac{\partial^2 P}{\partial x \partial t} + O((\Delta t)^3) + O(( \Delta x)^3) \nonumber
\end{align}
and so by substituting \eqref{eq:rw_pos} and \eqref{eq:rw_neg} into \eqref{eq:rw_standard} we get that
\begin{align}
    \frac{(\Delta x)^2}{2}\frac{\partial^2 P}{\partial x^2} = (\Delta t) \frac{\partial P}{\partial t} + O((\Delta t)^2) + O((\Delta x)^3).
\end{align}
Now sending $ \Delta x $ and $ \Delta t $ to zero \emph{at the same rate} we get that
\begin{align*}
    \label{eq:diffusion_standard}
    \frac{\partial P}{\partial t} = D \frac{\partial^2 P}{\partial x^2}
\end{align*}
where
\begin{align}
    D = \lim_{\Delta x \lra 0, \Delta t \lra 0} \frac{(\Delta x)^2}{2 \Delta t}.
\end{align}
Note that \eqref{eq:diffusion_standard} is the standard diffusion equation.

\subsubsection{Characterising Results}

We will breifly go through some results which help characterise how solutions to standard diffusion behave as these same ideas will be used in characterising the fractional differential equation.

\subsubsection{Analytic Solution Method}

There are actually relatively few situations where we can solve the diffusion equation in terms of named functions. Often these are solved by application of a Fourier transform and involve the error function which we define here for completeness.

\begin{definition}[Error Function]
    We define the error function as
    \begin{align}
        \erf(x) = \frac{2}{\sqrt{\pi}} \int_0^x e^{-t^2} dt
    \end{align}
    and the compelementry error function as
    \begin{align}
        \erfc(x) = \frac{2}{\sqrt{\pi}} \int_x^\infty e^{-t^2} dt.
    \end{align}
\end{definition}
It is rather elementry to show that
\begin{align}
    \int_0^\infty e^{-t^2} dt = \frac{2}{\sqrt{\pi}}
\end{align}
and hence we have $ \erfc(x) = 1 - \erf(x) $.

\subsection{Continuous Time Random Walks}
In the section above \emph{derived} the standard diffusion equation from a random walk. What was important about this derivation was that we took a discrete time / discrete space model, (descritized by $ \Delta t $ and $ \Delta x $ respectively) turned it into a continuous model by taking $ \Delta x $ and $ \Delta t $ to zero at the same rate. 

In this section we explore another method for dealing with random walks. One where the step size is distributed according to some probability distribution and the waiting time between steps is distributed according to some other distribution.

We will need to deal with two main densities in this process, the \emph{arrival density} and the \emph{being density}. 

We define $ a_n(x,t|x_0,t_0) $ as the conditional probability density which describes the probability that after $ n $ steps the random walk discussed above which started at $ x_0 $ at time $ t_0 $ arrives at $ x $ at time $ t $. 

Like in the case of standard diffusion we can give a recurrence relationship which begins the characterise this density. We can say that
\begin{align}
    \label{eq:fractional_recurrence}
    a_{n+1}(x,t)|x_0,t_0) = \int_{-\infty}^\infty \int_{t_0}^t \Psi(x - u, t - \tau)a_n(u, \tau|x_0, t_0) d\tau du
\end{align}
where $ \Psi(x - u, t - \tau) $ is the probability density which describes the probability that we take a step of size $ x - u $ after waiting for a time $ t - \tau $. 

The probability distribution $ \Psi $ characterises both the waiting time distribution and the step size distribution. We will assume that the waiting times and step sizes are independent of each other and hence we can \emph{factorize} this distribution as $ \Psi(x - u, t - \tau) = \lambda(x - u)\phi(t - \tau) $ where $ \lambda $ the step size probability density function and $ \phi $ is the waiting time probability density function.

We can write down the condition probability density for the random walk arriving at $ x $ at time $ t $ after \emph{any} number of steps from the starting position $ x_0 $ at $ t_0 $ as
\begin{align}
    a(x,t|x_0,t_0) = \sum_{n=0}^\infty a_n(x,t|x_0,t_0) 
\end{align}
by summing over the all the possible numbers of steps taken to arrive at $ x $. 

By performing this sum over \eqref{

\subsubsection{Standard Diffusion}
\subsubsection{Fractional Diffusion}
\subsubsection{Characterising Results for Fractional Diffusion}
\subsubsection{Solution Method for Fractional Diffusion}

\clearpage 
