\section{Introduction}
\subsection{History}
Fractional calculus is a field which extends all the way back to the birth of \emph{ordinary} calculus. In this section we wish to give an account of the development of this field. Like any historical recount this section will have its flaws, inaccuracies and ommissions. One can get bogged down in the historiogrraphy of the field and critise this recount as a story written with whiggish bias, but we seek to tell a story which is relevent to the rest of this thesis an contextualises modern fractional calculus.

In any historical introduction to fractional calculus one almost always starts with the letters exchanged between L'Hopital and Liebniz in the fall of 1695. Upon hearing of Liebniz's calculus and the differential operator
\begin{align}
    \frac{d}{dx}
\end{align}
and it's generalisation
\begin{align}
    \frac{d^n}{dx^n}
\end{align}
L'Hopital asked Liebniz what it would mean to set $ n = \frac{1}{2} $. Liebniz responded saying that 
\begin{align}
    d^\frac{1}{2}x = x\sqrt{dx:x}
\end{align} and adding that
``It will lead to a paradox, from which one day useful consequences will be drawn'' (Or at least that what most modern translations say) \cite{Abbas2012}. The notation is unfamiliar to modern mathematics, but this comment actually doesn't really reveal anything about how one might go about computing
\begin{align}
    \frac{d^\frac{1}{2}f}{dx^\frac{1}{2}}.
\end{align}

Two years latter in letters to J. Wallis and J. Bernoulli Euler proposed a possible approach to dealing with fractional derivatives of exponential functions \cite{Abbas2012}. He noticed that
\begin{align}
    \label{eq:euler_frac}
    \frac{d^m}{dx^m} e^nx = n^me^{nx} 
\end{align}
and proposed that one could simply consider non-integer values of $ m $ in order to define a fraction differential operator. 
It is perhaps on this note that we should point out that the name \emph{fractional} calculus is a bit of a misnomer. As in the case above we do not require $ m \in \mathbb{Q} $, but rather are allowed to select $ m \in \mathbb{R} $ or even $ m \in \mathbb{C} $.

Given the definition / motivation in \eqref{eq:euler_frac} one might be tempted to use this result and extend it to other functions via the theory of Fourier Analysis. We need to keep in mind, however, that this was the lat $ 17$th century and so Fourier had yet to be born, let alone develop the idea of Fourier decomposition and Fourier basis. This meant that at the time there was not \emph{obvious} way to extend this definition to other functions.

Jumping forward a quarter of a century we arive at the contributions of Euler. Euler made extremely important contributions in field of special functions, especially the Gamma function which turns out to be crucial in defining fractional integrals and fractional derivatives.

The Gamma function is essentially a solution to an interpolation problem, that is how to extend the factorial function to non-integer values of the argument. 

Although the problem of extending the factorial function had been considered by Daniel Bernoulli and Christian Goldbach in the 1720’s, it was eventually Euler who in a two letters, dated 13th October 1729 and 8th January 1730 respectively, gave two different representations of the factorial which could easily be extended to non-integer values. These were
\begin{align}
    \label{eq:euler_prod}
    n! = \prod_{k=1}^\infty \frac{\left(1+\frac{1}{k}\right)^n}{1 + \frac{n}{k}}
\end{align}
and
\begin{align}
    \label{eq:euler_log}
    n! = \int_0^1 (-\ln(s))^n ds.
\end{align}

Upon finding the first representation Euler could have simply stopped. He had found a formula which extended the factorial function to non-integer values. However he noticed that a special case of his product had already been calculated by Wallis. If we set $ n = \frac{1}{2} $ we get
\begin{align}
    \prod_{k=1}^\infty \frac{(1 + \frac{1}{k})^\frac{1}{2}}{1 + \frac{1}{2k}}
\end{align}
and squaring and multiplying by $ 2 $ we get
\begin{align}
    2 \left[ \prod_{k=1}^\infty \frac{(1 + \frac{1}{k})^\frac{1}{2}}{1 + \frac{1}{2k}}\right]^2 &= 2\prod_{k=1}^\infty \frac{1+\frac{1}{k}}{(1+\frac{1}{2k})^2} \\
    &= 2 \prod^\infty_{k=1} \frac{k^2+1}{(k + \frac{1}{2})^2} \\
    &= 2 \frac{2}{(1+\frac{1}{2})^2} \cdot \frac{5}{(2 + \frac{1}{2})^2} \cdot \frac{10}{(3+\frac{1}{2})^2}\cdots \\
    &= \frac{2}{1}\cdot\frac{2}{3}\cdot\frac{4}{3}\cdot\frac{4}{5}\cdots \\
    &= \prod_{k=1}^{\infty} \frac{4n^2}{4n^2-1}
\end{align}
and this last product had previously been calculated by Wallis and was known to be equal to $ \frac{\pi}{2} $. This meant that Euler had the result 
\begin{align}
    \frac{1}{2} ! = \frac{\sqrt{\pi}}{2}
\end{align}
We should point out that Wallis had calculated this result by considering 
\begin{align}
    \int_0^\pi \sin^n(x) dx
\end{align}
but this result is actually more easily obtained as a special case of Euler's infinite product for the sine function,
\begin{align}
    \sin(x) = x\prod_{k=1}^\infty \left( 1 - \frac{x^2}{k^2\pi^2} \right)
\end{align}
but Euler had not yet developed this.

The fact that Euler had $ \frac{1}{2} ! = \frac{\sqrt{\pi}}{2} $ pequed his curiosity and on nothing more than what appears to have been a hunch he went looking for an integral representation of $ n! $.
He arived at \eqref{eq:euler_log} and it is here that our story returns to the history of fractional calculus. Aware of the fact that a meaning for $ \frac{d^\frac{1}{2}}{dx^\frac{1}{2}} $ was sought he noted that
\begin{align}
    \label{eq:euler_frac_deriv}
    \frac{d^m}{dx^m} x^n = \frac{n!}{(n-m)!}x^{n-m}
\end{align}
and then suggested that using either \eqref{eq:euler_prod} or \eqref{eq:euler_log} could give the necessery extension of meaning in order to define the derivative of the power function for noninteger orders of differentiation. In fact using Wallis' result along with a slight extension he was able to suggest that
\begin{align}
    \frac{d^\frac{1}{2}}{dx^\frac{1}{2}} x = \sqrt{\frac{4x}{\pi}}.
\end{align}
Although technically the modern notation we use for the Gamma function and the fact that $ \Gamma(n) = (n-1)! $ is due to Legendre, and were developed some considerable time after these ideas were first worked on by Euler, we will adopt the modern gamma function notation so as to make our discussions from this point more readable to those with some background knowledge. 

We would like to point out just how \emph{good} this idea for the definition of the fractional derivative by Euler was.
Firstly note that if we take $ n = 1 $ in \eqref{eq:euler_frac_deriv} we get that
\begin{align}
    \frac{d^{-1}}{dx^{-1}} x^m &= \frac{\Gamma(m+1)}{\Gamma(m+2)} x^{m+1} \\
    &= \frac{1}{m+1} x^{m+1} \\
    &= \int_0^x t^m dt
\end{align} 
which is consitant with the fundamental theorem of calculus. We discuss a modern version of this result in section \ref{sec:operators} of this thesis. 

Also in some formal sense this definition of the fractional derivative is consistant with that of Liebniz in that if we take a Talyor expansion of the exponential function we get that
\begin{align}
    e^{mx} = \sum_{k=0}^\infty \frac{m^k}{k!} x^k
\end{align}
and so without regard for convergence or any other technicalities one might be tempted to write
\begin{align}
    \label{eq:Euler_Leibniz_Sum}\
    \frac{d^r}{dx^r} e^{mx} &= \sum_{k = r}^\infty \frac{d^r}{dx^r} \frac{m^k}{k!} x^k \\
                            &= \sum_{k = r}^\infty \frac{m^k}{\Gamma(k+1)} \frac{\Gamma(k+1)}{\Gamma(k - r + 1)} x^{k-r} \\             
                            &= \sum_{k = r}^\infty \frac{m^k}{\Gamma(k - r + 1)}x^{k-r} \\
                            &= m^r \sum_{k = r}^\infty \frac{m^{k-r}}{\Gamma(k - r + 1)}x^{k-r}.
\end{align}
Letting $ j = k - r $ we have
\begin{align*}
    m^r \sum_{k = r}^\infty \frac{m^{k-r}}{\Gamma(k - r + 1)}x^{k-r}
        &= m^r \sum_{j = 0}^\infty \frac{m^{j}}{\Gamma(j + 1)}x^{j} \\
        &= m^r \sum_{j = 0}^\infty \frac{m^{j}}{j!}x^{j} \\
        &= m^r e^{mx}.
\end{align*}
and so in some formal sense these two definitions would at first glance appear to be consistent with each other.

This analysis has several problems. Firstly although the idea of expressing functions in terms of an inifinite series dates back to perhaps the $ 14$th century the formal notiton of a Taylor series dates back to $1715$ and Brook Taylor, only a few years before Euler's work in this field and it is not clear that such techniques would have been available to Euler. 

Secondly, and more importantly from a mathematical point of view, is the slight of hand played in \eqref{eq:Euler_Leibniz_Sum} where we essentially assume that if $ m > n $, $ \frac{d^m}{dx^m} x^n = 0 $. This is not the case if $ m \not\in \mathbb{Z} $. For example, using Euler's definition above we would have that
\begin{align}
    \frac{d^\frac{3}{2}}{dx^\frac{3}{2}} x = \frac{1}{\sqrt{\pi}} x^{-\frac{1}{2}} \neq 0.
\end{align}
One can see that in the integer case we get $ 0 $ simply because of singularities in the Gamma function for the non-positive integers. This particular complication of non-zero derivatives for the case where $ m > n $ extends even into more modern fractional derivatives and is dealt with in a modern context in section \ref{sec:operators} of this thesis.

It was almost a century latter when in 1822 Fourier suggest that using the equality
\begin{align}
    \label{eq:fourier_frac}
    \frac{d^m}{dx^m} f(x) = \frac{1}{2\pi} \int_{-\infty}^\infty \int_{-\infty}^\infty f(t)u^m \cos\left(ux - tu - \frac{m\pi}{2}\right) dt du
\end{align}
could give meaning to fractional derivative of a function. This appear to have been the first definition of a fractional derivative for a general class of \emph{sufficiently good} functions, in this case, for which \eqref{eq:fourier_frac} is defined.

The history of more modern fractional differential operators really begins with the Abel integral equation which was solved by Abel in papers dating back to 1823 and 1826. It was originally posed and solved in the context of the Tautochrone problem, which is finding the curve for which objects sliding under gravity without the effects of friction take equal times to reach their lowest point, independent of their starting point.

Abel's intergral equation (of the first kind) is 
\begin{align}
    \label{eq:abel}
    \frac{1}{\Gamma(\alpha)} \int_0^x \frac{\phi(t)dt}{(x-t)^{1-\alpha}} = f(x) 
\end{align}
for $ x > 0 $ and $ 0 < \alpha < 1 $. Finding a \emph{solution} to this equation essentially means making $ \phi $ the subject. 
We demonstrate the solution method of this integral equation as it turns out to be important in latter formulations of fractional derivatives and integrals.
Firstly let's consider the integral 
\begin{equation}
	\label{eq:abel_int}
	I(x) := \int_a^x \frac{f(s)ds}{(x-s)^{1-\alpha}}.
\end{equation}
Now by substituting \eqref{eq:abel} into \eqref{eq:abel_int} we get 
\begin{align}
	I(x) &= \frac{1}{\Gamma(\alpha)} \int_a^x \frac{1}{(x-s)^{1-\alpha}} \left( \int_a^s \frac{\phi(t)dt}{(s-t)^\alpha} \right) ds \\
		&= \frac{1}{\Gamma(\alpha)} \int_a^x \left( \int_a^s \frac{\phi(t)dt}{(x-s)^{1-\alpha}(s-t)^\alpha} \right) ds
\end{align}
Now noting that the region of integration in $ \mathbb{R}^2 $ is just
\begin{align}
	a &\leq s \leq x \\
	a &\leq t \leq s 
\end{align}
which is equivalent to 
\begin{align}
	t &\leq s \leq x \\
	a &\leq t \leq x 
\end{align}
we can write 
\begin{align}
	\frac{1}{\Gamma(\alpha)} \int_a^x \left( \int_a^s \frac{\phi(t)dt}{(x-s)^{1-\alpha}(s-t)^\alpha} \right) ds 
		&= \frac{1}{\Gamma(\alpha)} \int_a^x \left( \int_t^x \frac{\phi(t)ds}{(x-s)^{1-\alpha}(s-t)^\alpha} \right) dt \nonumber \\
		\label{eqn:prebeta}
		&= \frac{1}{\Gamma(\alpha)} \int_a^x \phi(t) \left( \int_t^x (x-s)^{\alpha-1}(s-t)^{-\alpha} ds\right) dt. 
\end{align}
Now performing the substitution $ \tau = \frac{s-t}{x-t} $ yields 
\begin{align}
	\int_t^x (x-s)^{\alpha-1}(s-t)^{-\alpha} ds &= \int_0^1 \tau^{-\alpha} (1-\tau)^{\alpha - 1} d\tau \\
		&= B(1-\alpha,\alpha) \\
		&= \Gamma(1-\alpha)\Gamma(\alpha)
\end{align}
and so \eqref{eqn:prebeta} becomes
\begin{align}
	\frac{1}{\Gamma(\alpha)} \int_a^x \phi(t) \left( \int_t^x (x-s)^{\alpha-1}(s-t)^{-\alpha} ds\right) dt
		&= \frac{1}{\Gamma(\alpha)} \int_a^x \phi(t) \Gamma(\alpha)\Gamma(1-\alpha) dt \\
		&= \Gamma(1-\alpha)\int_a^x \phi(t) dt. 
\end{align}
So we have that 
\begin{align}
	\int_a^x \frac{f(s)ds}{(x-s)^{1-\alpha}} &= \Gamma(1-\alpha)\int_a^x \phi(t) dt
\end{align}
and by differentiating we get
\begin{align}
	\phi(x) &= \frac{1}{\Gamma(1-\alpha)} \frac{d}{dx} \int_a^x \frac{f(s)ds}{(x-s)^{1-\alpha}}
\end{align}
and so we have in some sense solved Abel's integral equation. Notice that we have not discussed which functions $ f $ are suitable. This is a bit of a tricky question and not really informative in the context of the rest of this thesis so we refer the interested reader to \cite{Samko1993}. 

As mentioned earlier Abel solved this integral qequation in the context of the Tautochrone problem which corresponded to $ \alpha = \frac{1}{2} $. Abel seemingly had no actual desire to define a fractional integral or derivative but it turns out this problem can be recast in terms of fractional integral equation using a Riemann Liouiville fractional integral and the result is in some sense a statement of a deneralised fundamental theorem of calculus. We'll leave these ideas and turn our attention to the development of the Riemann-Liouville fractional integral and derivative and return to see how these ideas relate.


\clearpage
