\section{Numerical Evaluation of Special Functions}

In this section we cover the numerical evaluation of the Mittag-Leffler function and the Wright function
which as seen in other sections of this thesis, are functions which commonly arise when dealing with
fractional differential equations.

Even if an analytic solution to an initial value problem, or a boundary value problem is known, actually
evaluating values of the functions involved in the function is not trivial. Despite the fact that both
the Mittag-Leffler function and the Wright function are defined in terms of series, these series do not 
converge at a satisfactory rate. [NEED REF] Also these series involve gamma functions which in themselves
have a number of numerical difficulties, at least in so much as it is not quite as clear how to evaluate 
a gamma function as it is a factorial. We discuss the evaluation of the gamma function in section \ref{sec:num-gamma}.

In the following few sections we will consider contour integral representations of the Gamma function, Mittag-Leffler 
function and the Wright functions. Many of the ideas explained in this section date back to a relatively famous series of books,
\emph{Higher Transcendental Functions}, specifically volume 3 \cite{Bateman1955}, and a paper (in Russian) by Dzherbashyan \cite{Dzherbashyan1954} in 1954. 
These ideas have been explained again by Podlubny \cite{Podlubny1999} and Gorenflo et. al. \cite{Gorenflo2002} considerably more recently (in 1999 and 2002 respectively). With strong reference to Podlubny's explanations we seek to give an account of these ideas and then review the work of  Gorenflo et. al. who use these results to present numerical algorithms for evaluating these functions. 

 
\subsection{Mittag-Lefller Function}
\subsubsection{Contour Integral Representation of $ \Gamma(z) $ and $ \frac{1}{\Gamma(z)} $}

The gamma function is defined as

\begin{align}
    \Gamma(z) = \int_0^\infty e^t t^{z-1} dt
\end{align}
where integration is done with a real valued $ t $. The idea here is to \emph{lift} this integral into
complex plane an then integrate along the contour shown in figure \ref{fig:Hankel_Loop_2}. We formalize 
this idea in the following lemma.

\begin{lemma}
    \label{lem:contour_gamma}
    A contour integral representation of $ \Gamma(z) $ is given by
    \begin{align}
        \Gamma(z) = \frac{1}{e^{2 \pi i z} - 1} \int_{C} e^{-t}d^{z-1} dt
    \end{align}
where $ C $ is the contour depicted in figure \ref{fig:Hankel_Loop_2}.
\end{lemma}
This proof follows that of Podlubny \cite{Podlubny1999}.
\begin{proof}
Consider the contour integral
\begin{align}
    \int_C e^{(z-1)\log(t) - t} dt.
\end{align}
Due to the fact that the integrand involves a $ \log $ function it is multivalued unless we specify a branch
cut. We will adopt a branch cut along the non-negative real axis. As long as $ \varepsilon $ 
is chosen sufficiently small ($ \varepsilon < 1 $) then Cauchy's theorem guarentees that this contour 
integral has the same value for all choices of $ \varepsilon > 0 $ because the integrand has only one
singularity at $ t = 0 $.

Now consider the contour in three pieces, $ C_{\varepsilon} $, the circle of radius $ \varepsilon $ 
centered at 0, $ (\varepsilon, \infty) $, the bottom half of the contour and $ (\infty, \varepsilon) $, 
the top half of the contour. 

As $ \varepsilon $ can be made arbitarily small we take both straight parts of the contour to be
real but on the lower half of the contour we must replace $ \log(t) $ with $ \log(t) + 2\pi i $. 

This results in 
\begin{align}
    \int_C e^{(z-1)\log(t)-t} dt &= \int_\infty^\varepsilon e^{(z-1)\log(t) - t} dt + \int_{C_\varepsilon} 
        e^{(z-1)log(t) - t} dt + \int_{\varepsilon, \infty} e^{(z-1)\log(t) - t + 2\pi i} dt \\
    &= \int_\infty^\varepsilon e^{-t}t^{z-1}dt + \int_{C_\varepsilon} e^{-t}t^{z-1}dt +
    e^{2(z-1)\pi i} \int_\varepsilon^\infty e^{-t}t^{z-1} dt
\end{align}

\begin{wrapfigure}{l}{210pt}
    \begin{tikzpicture}[scale=2]
    \draw[<->] (1.5,0) -- (-1.5,0);
    \draw[->,red] (1.5,0.25) -- (0.25, 0.25);
    \draw[->, red] (0.25, -0.25) -- (1.5,-0.25);
    \draw[->, red] (0.25,0.25) arc [radius = 0.35355, start angle=45, end angle=180];
    \draw[red] (0.25,-0.25) arc [radius = 0.35355, start angle=-45, end angle=-180];
    \draw[<->] (0,1.5) -- (0,-1.5);
    \draw[densely dotted] (0,0) -- (-0.24,0.24);   
    \draw (-0.25, 0.15) node {$\varepsilon$};
    \draw (1.5, 0.125) node {$Re$};
    \draw (0.135, 1.5) node {$Im$};
\end{tikzpicture}

    \caption{The Hankel contour C}
    \label{fig:Hankel_Loop_2}
\end{wrapfigure}

It can be shown by simple application of the ML lemma, as in \cite{Podlubny1999}, that the integral along
$ C_\varepsilon $ goes to zero as $ \varepsilon \lra 0 $.

Taking $ \varepsilon \lra 0 $ in the other two integrals and rearanging gives us the result
\begin{align}
    \label{eq:gamma_contour}
    \int_0^\infty e^{-t} t^{z-1} dt = \frac{1}{e^{2 \pi i z} - 1} \int_c e^{-t}t^{x-1} dt
\end{align}
\end{proof}

We would like to exploit this result to give a simple contour integral representation of $ \frac{1}{\Gamma(z)} $.

\begin{lemma}
    \label{lem:contour_1_gamma_1}
    A contour integral representation of $ \frac{1}{\Gamma(z)} $ is given by
    \begin{align}
        \frac{1}{\Gamma(z)} = \frac{1}{2 \pi i} \int_{Ha} e^\tau \tau^{-z} d\tau
    \end{align}
    where the contour $ Ha $ is shown in figure \ref{fig:hankel_loop}.
\end{lemma}
This proof follows that of Podlubny \cite{Podlubny1999}
\begin{proof}
    By taking the result of \ref{lem:contour_gamma} and substituting $ z-1 $ in \eqref{eq:gamma_contour}
    we get that
    \begin{align}
        \int_C e^{-t} t^{-z} dt = (e^{-2 \pi i z} -1) \Gamma(1-z).
    \end{align}
    Performing the substitution $ \tau = -t $ rotates the contour integral clockwise $ 180 $ degrees 
    and results in 
    \begin{align}
        \int_{C} e^{-t} t^{-z} dt = - e^{-z\pi i}\int_{Ha} e^{\tau} \tau^{-z} d\tau
    \end{align}
    and so
    \begin{align}
        \Gamma(1-z) &= \frac{1}{e^{z\pi i} - e^{-z \pi i}} \int_{Ha} e^\tau \tau^{-z} d\tau \\
            &= \frac{1}{2i \sin(\pi z)}\int_{Ha} e^\tau \tau^{-z} d\tau.
    \end{align}
    Now taking into account the well known Euler reflection formula, $ \Gamma(z)\Gamma(1-z) = \frac{\pi}{\sin(\pi z)} $, we get
    \begin{align}
        \frac{1}{\Gamma(z)} = \frac{1}{2 \pi i} \int_{Ha} e^\tau \tau^{-z} d\tau.
    \end{align} \qed
    \begin{wrapfigure}{r}{210pt}
        \begin{tikzpicture}[scale=2]
    \draw[<->] (1.5,0) -- (-1.5,0);
    \draw[<-,red] (-1.5,0.25) -- (-0.25, 0.25);
    \draw[<-, red] (-0.25, -0.25) -- (-1.5,-0.25);
    \draw[<-, red] (-0.25,0.25) arc [radius = 0.35355, start angle=135, end angle=0];
    \draw[red] (-0.25,-0.25) arc [radius = 0.35355, start angle=-135, end angle=0];
    \draw[<->] (0,1.5) -- (0,-1.5);
    \draw[densely dotted] (0,0) -- (0.24,0.24);   
    \draw (0.25, 0.15) node {$\varepsilon$};
    \draw (1.5, 0.125) node {$Re$};
    \draw (0.135, 1.5) node {$Im$};
\end{tikzpicture}

        \caption{The Hankel contour Ha}
        \label{fig:hankel_loop}
    \end{wrapfigure}
\end{proof}

We now wish to give a pair of contour integral representations of the function $ \frac{1}{\Gamma(z)} $
which we will then use in contour integrals representing $ E_{\alpha, \beta}(z) $. 

\begin{lemma}
    \label{lem:1_gamma_contour}
    The function $ \frac{1}{\Gamma(z)} $ can be represented by the following contour integrals 
    \begin{align}
        \label{eq:gamma_contour_1}
        \frac{1}{\Gamma(z)} = \frac{1}{2 \pi \alpha i} \int_{\gamma(\varepsilon, \mu)} 
                              e^{\zeta^{1 / \alpha}} \zeta^{(1-z-\alpha) / \alpha} d\zeta 
    \end{align}
    where
    \begin{align}
        \left( \alpha < 2, \frac{\pi \alpha}{2} < \mu < \min\{ \pi, \pi \alpha \} \right)
    \end{align}
    or where $ Re(z) > 0 $
    \begin{align}
        \label{eq:gamma_contour_2}
        \frac{1}{\Gamma(z)} = \frac{1}{4 \pi i} \int_{\gamma(\varepsilon, \pi)} e^{\zeta^{1 / 2}} \zeta^{-(z + 1) / 2} d\zeta
    \end{align}
    where the contour $ \gamma(\varepsilon, \varphi) $ is depicted in figure \ref{fig:hankel_wedge}.
\end{lemma}
This proof follows that of Podlubny \cite{Podlubny1999}.
\begin{proof}

Firstly we show that we can rewrite the contour integral in lemma \ref{lem:contour_1_gamma_1} can be rewritten as a contour
integral about $ \gamma(\varepsilon, \varphi) $ where $ \gamma(\varepsilon, \varphi) $ is shown in figure \ref{fig:hankel_wedge}. We restrict $ \varphi $ to $ \frac{\pi}{2} < \varphi \pi $. 

Now lets consider the contour integral diagram in figure \ref{fig:hankel_wedge_2} and note that as $ f(\tau) := e^\tau t^{-z} $ does not have any singularities we have
\begin{align}
    \int_{A^+ B^+ C^+ D^+} f(\tau) d\tau = \int_{A^-D^-C^-B^-} f(\tau) = 0.
\end{align}

Considering just the arc $ A^+B^+ $ we note that $ |\tau| = R $ and hence
\begin{align}
    |f(\tau)| &= |e^\tau \tau^{-z}| = e^{R\cos(\arg \tau) - Re(z) \log(R) + Im(z) \arg \tau} \\
      &\leq e^{-R\cos(\pi - \phi) - x \log(R) + 2 \pi y} \\
      &\sim \frac{1}{R}e^{-R}
\end{align}

and thus by application of the ML lemma we have
\begin{align}
    \lim_{R \lra \infty} \int_{A^+}^{B^+} f(\tau) d\tau = 0.
\end{align}

\begin{wrapfigure}{l}{210pt}
    \begin{tikzpicture}[scale=2]
    \draw[<->] (1.5,0) -- (-1.5,0);
    \draw[<-,red] (-1.5,1.5) -- (-0.25, 0.25);
    \draw[<-, red] (-0.25, -0.25) -- (-1.5,-1.5);
    \draw[<-, red] (-0.25,0.25) arc [radius = 0.35355, start angle=135, end angle=0];
    \draw[red] (-0.25,-0.25) arc [radius = 0.35355, start angle=-135, end angle=0];
    \draw[<->] (0,1.5) -- (0,-1.5);
    \draw[densely dotted] (0,0) -- (0.24,0.24);   
    \draw (0.25, 0.15) node {$\varepsilon$};
    \draw (1.5, 0.125) node {$Re$};
    \draw (0.135, 1.5) node {$Im$};
    \draw (-1, 0.5) node {$ G^-(\varepsilon, \varphi) $};
    \draw (1, 1) node {$G^+(\varepsilon, \varphi) $};
    \draw[<-, densely dotted] (-0.5, 0.5) arc [radius = 0.7071 , start angle=135, end angle=0];
    \draw (0.5, 0.6) node {$\varphi$};
\end{tikzpicture}

    \caption{The Hankel countour $ \gamma(\varepsilon, \varphi) $}
    \label{fig:hankel_wedge}
\end{wrapfigure}

A completely symetric argument yields
\begin{align}
    \lim_{R \lra \infty} \int_{B^-}^{A^-} f(\tau) d\tau = 0
\end{align}
and so we can write
\begin{align}
    \int_{C^+}^{B_\infty^+} f(\tau)d\tau = \int_{C^+}^{D^+} f(\tau)d\tau + \int_{D^+}^{\infty^+} f(\tau)d\tau
\end{align}
and
\begin{align}
    \int_{B_\infty^-}^{C^-} f(\tau)d\tau = \int_{\infty^-}^{D^-} f(\tau)d\tau + \int_{D^-}^{C^-} f(\tau)d\tau.
\end{align}
Then
\begin{align}
    \int_{Ha} f(\tau) d\tau = \left( \int_{B_\infty^-}^{C^-} f(\tau)d\tau + \int_{C^-}^{C^+} f(\tau)d\tau +  \int_{C^+}^{B_\infty^+} f(\tau)d\tau \right) = \int_{\gamma(\varepsilon, \varphi)}e^\tau \tau^-z d\tau.
\end{align}
To get the final results, we perform the substitution $ \tau = \zeta^{1/\alpha} $ for $ \alpha < 2 $
which yields

\begin{align}
    \frac{1}{\Gamma(z)} = \frac{1}{2 \pi \alpha i} \int_{\gamma(\varepsilon, \mu)} 
			  e^{\zeta^{1 / \alpha}} \zeta^{(1-z-\alpha) / \alpha} d\zeta 
\end{align}

so long as $ \alpha < 2 $ and $ \frac{\pi \alpha}{2} < \mu < \min\{ \pi, \pi \alpha \} $.

If, however, we perform the substitution $ \tau = \sqrt{\zeta} $ and integrate over the contour $ \gamma(\varepsilon, \frac{\pi}{2} $ we get

\begin{wrapfigure}{l}{10pt}
    \begin{tikzpicture}[scale=2]
\draw[<->] (1.5,0) -- (-1.5,0);
\draw[<-,red] (-0.66,0.66) -- (-0.25, 0.25);
\draw[<-, red] (-0.25, -0.25) -- (-0.66,-0.66);
\draw[<-, red] (-0.25,0.25) arc [radius = 0.35355, start angle=135, end angle=0];
\draw[red] (-0.25,-0.25) arc [radius = 0.35355, start angle=-135, end angle=0];
\draw[<->] (0,1.5) -- (0,-1.5);
\draw[densely dotted] (0,0) -- (0.24,0.24);
\draw (0.25, 0.15) node {$\varepsilon$};
\draw (0.75, -0.75) node {$R$};
\draw (1.5, 0.125) node {$Re$};
\draw (0.135, 1.5) node {$Im$};
\draw[<-, red] (-0.25,-0.25) arc [radius = 0.35355, start angle=-135, end angle=-170];
\draw[<-, red] (-0.25,0.25) arc [radius = 0.35355, start angle=135, end angle=170];
\draw[red] (-1,0.05) arc [radius = 1, start angle=170, end angle=130];
\draw[red] (-1,-0.05) arc [radius = 1, start angle=-170, end angle=-130];
\draw[<-, red] (-1,0.05) -- (-0.35255, 0.05);
\draw[->, red] (-1,-0.05) -- (-0.35255, -0.05);
\draw[densely dotted] (-0.66,-0.66) arc [radius = 0.86, start angle=-130, end angle=130];
\draw[->, densely dotted] (0,0) -- (0.55, -0.55);
\end{tikzpicture}
    \caption{ Integration contour for $ \gamma(\varepsilon, \varphi) $}
    \label{fig:hankel_wedge_2}
\end{wrapfigure}

\begin{align}
    \frac{1}{\Gamma(z)} = \frac{1}{4 \pi i} \int_{\gamma(\varepsilon, \pi)} e^{\zeta^{1 / 2}} \zeta^{-(z + 1) / 2} d\zeta
\end{align}
for $ Re(z) > 0 $.

\end{proof}

\subsubsection{Contour Integral Representation of $ E_{\alpha, \beta} $ }

We now develop integral representations of the Mittag-Leffler function. These contour integral representations will form the basis of the numerical method we present for evaluating the Mittag-Leffler function.

\begin{lemma}
    \label{lem:mit_lef_contour_1}
    For $ \alpha \in \mathbb{R} $ with $ 0 < \alpha < 2 $ and $ \beta \in \mathbb{C} $ and $ \varepsilon > 0 $
    such that
    \begin{align}
        \label{eq:mit_lef_region}
	    \frac{\pi \alpha}{2} < \mu \leq \min\{ \pi, \pi \alpha \}
    \end{align}
    we have that
    \begin{align}
	E_{\alpha, \beta}(z) = \frac{1}{2\alpha\pi i} \int_{\gamma(\varepsilon, \mu)} \frac{e^{\zeta^{1 / \alpha}}\zeta^{(1-\beta)/\alpha}}{\zeta - z} d\zeta
    \end{align}
    for $ z \in G^{-}(\varepsilon, \mu) $
    and
    \begin{align}
    \label{eq:mit_lef_contour_2}
	\frac{1}{\alpha} z^{(1-\beta) / \alpha} e^{z^{1 / \alpha}} + \frac{1}{2 \alpha\pi i} \int_{\gamma(\varepsilon, \mu)} + \frac{1}{2\alpha\pi i} \int_{\gamma(\varepsilon, \mu)} \frac{e^{\zeta^{1 / \alpha}}\zeta^{(1-\beta)/\alpha}}{\zeta - z} d\zeta
    \end{align}
    for $ z \in G^{+}(\varepsilon, \mu) $.
    The contour $ \gamma(\varepsilon, \mu) $ is depicted in figure \ref{fig:hankel_wedge_2} as are the regions $ G^+ $ and $ G^-$.
\end{lemma}
This proof follows that of Podlubny \cite{Podlubny1999}.
\begin{proof}
    Starting from the definition of the Mittag-Leffler function and the absolutely convergent nature of the Mittag-Leffler function from lemma \ref{lem:mit_conv} we have that
    
    \begin{align}
	E_{\alpha, \beta}(z) &= \sum_{k=0}^\infty \frac{z^k}{\Gamma(\alpha k + \beta)} \\
	    &= \sum_{k=0}^\infty \frac{1}{2\pi\alpha i} z^k \int_{\gamma(\varepsilon, \mu)} e^{\zeta^{1 / \alpha}} \zeta^{(1-\beta)/\alpha - k - 1} d\zeta \\
	&= \frac{1}{2 \pi \alpha i}  \int_{\gamma(\varepsilon, \mu)} e^{\zeta^{1 / \alpha}} \zeta^{(1-\beta)/\alpha - 1} \left(\sum_{k=0}^\infty \left( \frac{z}{\zeta}\right)^k \right) d\zeta. & \circledast
   \end{align}
and if we assume that $ |z| < \varepsilon $ then
\begin{align}
    \left| \frac{z}{\zeta}\right| < 1
\end{align}
because $ \zeta \in \gamma(\varepsilon, \mu) $ and so
\begin{align}
   \sum_{k=0}^\infty \left( \frac{z}{\zeta} \right) = \frac{\zeta}{\zeta - z}
\end{align}
which means that
\begin{align}
    \circledast = \int_{\gamma(\varepsilon, \mu)} \frac{e^{\zeta^{1 / \alpha}}\zeta^{(1-\beta)/\alpha}}{\zeta - z} d\zeta.
\end{align}

As argued in Podulbny \cite{Podlubny1999}, because of \ref{eq:mit_lef_region}, this integral is absolutely convergent in $G^-(\varepsilon, \mu) $ and is an analytic function of $ z $ and so by the principle of analytic continuation this integral equals $ E_{\alpha, \beta}(z) $ in all of $ G^-(\varepsilon, \mu) $ rather than just $ |z| < \varepsilon $.

If $ z \in G^+(\varepsilon, \mu) $ we can say that $ z \in G(\delta, \mu) $ where $ \delta $ is chosen so that $ |z| < \delta $. The idea here it to get $ z $ on the other side of a contour $ \gamma(\delta, \mu) $.
Then the above result still stands, but with a different contour. So if $ \varepsilon < |z| <\delta $ and $ -\mu < \arg(z) < \mu $ then we can use Cauchy's theorem to write
\begin{align}
    \frac{1}{2\alpha \pi i} \int_{\gamma(\varepsilon, \mu) - \gamma(\varepsilon, mu)} \frac{e^{\zeta^{1 / \alpha} \zeta^{(1-\beta)/\alpha}}}{\zeta - z} d \zeta = \frac{z^{(1-\beta)/\alpha} e^{z^{1/\alpha}}}{\alpha}
\end{align}
and hence \ref{eq:mit_lef_contour_2} follows.
\end{proof}
\begin{lemma}
    If $ Re(\beta) > 0 $ then for $ \varepsilon > 0 $ we have that
    \begin{align}
        E_{2,\beta}(z) = \frac{1}{4\pi i} \int_{\gamma(\varepsilon, \pi)} \frac{e^{\zeta^{1/2}}\zeta^{(1-\beta) / 2}}{\zeta - z}d \zeta
    \end{align}
    for $ z \in G^-(\varepsilon, \pi) $ 

and
    \begin{align}
        E_{2,\beta}(z) = \frac{1}{2} z^{(1-\beta)/2} e^{z^{1/2}} + \frac{1}{4\pi i} \int_{\gamma(\varepsilon, \pi)} \frac{e^{\zeta^{1/2}}\zeta^{(1-\beta) / 2}}{\zeta - z}d \zeta
    \end{align}
    for $ z \in G^+(\varepsilon, \pi) $.
\end{lemma}
This proof is essentially identical to that of \ref{lem:mit_lef_contour_1}, with the use of \eqref{eq:gamma_contour_2} instead of \eqref{eq:gamma_contour_1} but we go through it here explicitly for the sake of clarity. 
\begin{proof}
    Starting from the usual definition of the Mittag-Leffler function and replacing $ \frac{1}{\Gamma(\alpha k + \beta)} $ with the
    contour integral representation in \eqref{eq:gamma_contour_2} we get
    \begin{align}
        E_{\alpha, \beta}(z) &= \frac{1}{4\pi i} \sum_{k=0}^\infty z^k \int_{\gamma(\varepsilon, \pi)} e^{\zeta^{1/2}} \zeta^{-(2k-\beta + 1)/2} d\zeta \\
        &= \frac{1}{4 \pi i} \int_{\gamma(\varepsilon, \pi)} e^{\zeta^{1/2}} \zeta^{-(1-\beta)/2} \sum_{k=0}^\infty \left( \frac{z}{\zeta} \right)^k d\zeta \\
        &= \frac{1}{4\pi i} \int_{\gamma(\varepsilon, \pi)} \frac{e^{\zeta^{1/2}} \zeta^{-(1-\beta)/2}}{\zeta - z} d\zeta.
    \end{align}
    The arguments here are the same for the interchange of the integral and the sum. The arguments about convergence an analytic continuation are also the same so long as $ z \in G^-(\varepsilon, \pi) $. 

    Applying the same idea as in the previous proof we consider a \emph{larger} contour $ \gamma(\delta, \mu) $ so that $ z \in G^-(\delta, \mu) $ and apply Cauchy's theorem to arive at
    \begin{align*}
        E_{2,\beta}(z) = \frac{1}{2} z^{(1-\beta)/2} e^{z^{1/2}} +  \frac{1}{4\pi i} \int_{\gamma(\varepsilon, \pi)} \frac{e^{\zeta^{1/2}} \zeta^{-(1-\beta)/2}}{\zeta - z} d\zeta.
    \end{align*}
\end{proof}
We now wish to use these integral representations to develop asymptotic expansions.
\begin{lemma}
    F $ 0 < \alpha < 2 $ and $ \mu $ is such
    \begin{align}
        \frac{\pi \alpha}{2} < \mu < \min\{\pi, \pi\alpha \}
    \end{align}
    and $ | \arg(z) | \leq \mu $
    then for any integer $ p \geq 1 $    
    \begin{align}
        E_{\alpha, \beta}(z) = \frac{1}{\alpha} z^{(1 - \beta) / \alpha} e^{z^{1/\alpha}} - \sum_{k=1}^p \frac{z^{-k}}{\Gamma(\beta - \alpha k)} + O(|z|^{-1-p})
    \end{align}
    as $ |z| \lra \infty $.
\end{lemma}
This proof follows that of Podlubny \cite{Podlubny1999}.
\begin{proof}
    Take some $ \varphi $ such that $ \mu < \varphi \leq \min\{\pi,\pi\alpha\} $ and set
    $ \varepsilon  = 1 $.
    Note that 
    \begin{align}
        \label{eq:geometric_expansion}
        \frac{1}{\zeta - z} = \frac{1}{\zeta}\sum_{k=0}^p \left( \frac{\zeta}{z}\right)^k + \frac{\zeta^p}{z^p(\zeta - z)}.
    \end{align}
    Now substituting this result into \eqref{eq:mit_lef_contour_2}
    we get that for $ z \in G^+(1, \varphi) $
    \begin{align}
        E_{\alpha, \beta}(z) &= \frac{1}{\alpha} z^{(1-\beta)/\alpha} e^{z^{1 / \alpha}} \\
        \label{eq:mit_lef_exp_1}    
        & \ \ \ - \sum_{k=1}^p \left( \frac{1}{2\pi\alpha i} \int_{\gamma(1,\varphi)} e^{\zeta^{1\/alpha}} \zeta^{(1-\beta)/\alpha + k - 1} d\zeta \right) z^{-k}  \\ 
        \label{eq:mit_lef_exp_2}
            & \ \ \ + \frac{1}{2\pi\alpha i z^p} \int_{\gamma(1,\phi)} e^{\zeta^{\alpha / 1}} \zeta^{(1-\beta)/\alpha + p} d\zeta.
    \end{align}
    Now \eqref{eq:mit_lef_exp_1} can be evaluated by using the result of lemma \ref{lem:1_gamma_contour} which leaves us with 
    \begin{align}
        \label{eq:mit_lef_asym_sum_1}
        \sum_{k=1}^p \left( \frac{1}{2\pi\alpha i} \int_{\gamma(1,\varphi)} e^{\zeta^{1\/alpha}} \zeta^{(1-\beta)/\alpha + k - 1} d\zeta \right) z^{-k} = \sum_{k=1}^p \frac{z^{-k}}{\Gamma(\beta - \alpha k)}.
    \end{align}
    It remains to \emph{estimate} \eqref{eq:mit_lef_exp_2} when $ |arg(z)| \leq \mu $ and as $ |z| \lra \infty $. 
    
    Firstly note that
    \begin{align}
        \min_{\zeta \in \gamma(1, \varphi)} |\zeta - z| = |z|\sin(\varphi - \mu)
    \end{align}
    and therefore
    \begin{align}
        \label{eq:mit_lef_contour_int_conv}
        \frac{1}{2\pi\alpha i z^p} \int_{\gamma(1,\phi)} e^{\zeta^{\alpha / 1}} \zeta^{(1-\beta)/\alpha + p} d\zeta \leq \frac{|z|^{-1-p}}{2\pi\alpha\sin(\varphi - \mu)} \underbrace{\int_{\gamma(1,\varphi)} \left| e^{\zeta^{1/\alpha}} \right| \cdot \left| \zeta^{(1-\beta)+p}\right| d\zeta}_{\circledast}.
    \end{align}
We wish to show that $ \circledast $ converges, which is actually not hard because 
\begin{align}
    \left| e^{\zeta^{1/\alpha}}\right| = e^{|\zeta|^{1/\alpha} \cos\left(\frac{\pi}{\alpha}\right)}
\end{align}
and $ \cos\left(\frac{\varphi}{\alpha}\right) < 0 $ because $ \frac{\pi \alpha}{2} < \varphi \leq \min\{\pi, \pi \alpha\} $. 
This means that 
\begin{align}
    \label{eq:mit_lef_asym_int_1}
    \frac{1}{2\pi\alpha i z^p} \int_{\gamma(1,\phi)} e^{\zeta^{\alpha / 1}} \zeta^{(1-\beta)/\alpha + p} d\zeta = O(|z|^{-1-p})
\end{align}
and so by combining \eqref{eq:mit_lef_asym_int_1} and \eqref{eq:mit_lef_asym_sum_1} with the \eqref{eq:mit_lef_contour_2} the result follows.
\end{proof}
\subsubsection{A Numerical Algorithm for Evaluating $ E_{\alpha, \beta}(z) $}
\subsection{Wright Function}
\subsubsection{Contour Integral Representation of $ W_{\alpha, \beta}(z) $}
\subsubsection{A Numerical Algorithm for Evaluating $ W_{\alpha, \beta}(z) $}

\subsection{Additional Notes on the Numerical Evaluation of the Gamma Function}
\label{sec:num-gamma}
These notes are included because the numerical evaluation of the gamma function is important regardless of
whether it is actually used in the above methods for evaluation of the Mittag-Leffler and Wright functions.
For example the Lanczos approximation is used in the scheme outlined in [INSERT CC REF] and implemented 
in code available in Appendix A.

\subsubsection{Stirling's Approximation}

\subsubsection{Lanczos Approximation}

\subsubsection{Spouge Approximation}
The Spouge approximation, like teh Lanczos approximation 

\clearpage
