
\section{Special Functions}

The field of fractional calculus is intimately linked with the field of special functions. To express the fractional derivative
of many functions one has to consider special functions. Just like the field of \emph{ordinary} calculus has collections of functions which have a number of desirable properties like invariance (exponential) and \emph{periodic invariance} (sine, cosine, hyperbolic sine, hyperbolic cosine), fractional calculus has functions which have a very similar role.

We will look in detail at three functions, the Gamma function, the Mittag-Leffler function and the Wright function. It may seem that just looking at three functions will be limiting, but this is not the case. Both the Mittag-Leffler function and the Wright Function generalisations of more \emph{ordinary} functions and they are general enough to serve considerable utility. The Gamma function is important in the definition of both of these functions and a collection of other special functions. The field of fractional calculus is surrounded by a virtual zoo of special functions including the Meijer G-function, Fox-H function, the generalised hyper-geometric functions, MacRobert E-function and the Mainardi function. This list could be made arbitrarily large by simply reading more and more literature on the subject, but these functions give us a way of formalising remarkable relationships. [TODO ADD REFS]

\subsection{Gamma Function}

\begin{mdframed}[innertopmargin=10pt]
\begin{definition}
Define the gamma function $ \Gamma : \mathbb{R} \setminus \{ 0, -1, -2, \ldots \} \lra \mathbb{R}  $

\begin{align}
    \Gamma(z) = \int_0^\infty e^t t^{z-1} dt.
\end{align}
\end{definition}
\end{mdframed}


The idea will be to \emph{lift} this integral into the complex plain to get a contour integral representation of $ \Gamma(z) $.

\begin{mdframed}[innertopmargin=10pt]
\begin{lemma}
    \label{lem:contour_gamma}
    A contour integral representation of $ \Gamma(z) $ is given by
    \begin{align}
        \Gamma(z) = \frac{1}{e^{2 \pi i z} - 1} \int_{C} e^{-t}d^{z-1} dt
    \end{align}
where $ C $ is the contour depicted in figure \ref{fig:Hankel_Loop_2}.
\end{lemma}
\end{mdframed}

\begin{wrapfigure}{l}{210pt}
    \begin{tikzpicture}[scale=2]
    \draw[<->] (1.5,0) -- (-1.5,0);
    \draw[->,red] (1.5,0.25) -- (0.25, 0.25);
    \draw[->, red] (0.25, -0.25) -- (1.5,-0.25);
    \draw[->, red] (0.25,0.25) arc [radius = 0.35355, start angle=45, end angle=180];
    \draw[red] (0.25,-0.25) arc [radius = 0.35355, start angle=-45, end angle=-180];
    \draw[<->] (0,1.5) -- (0,-1.5);
    \draw[densely dotted] (0,0) -- (-0.24,0.24);   
    \draw (-0.25, 0.15) node {$\varepsilon$};
    \draw (1.5, 0.125) node {$Re$};
    \draw (0.135, 1.5) node {$Im$};
\end{tikzpicture}

    \caption{The Hankel contour C}
    \label{fig:Hankel_Loop_2}
\end{wrapfigure}


This proof follows that of Podlubny \cite{Podlubny1999}.
\begin{proof}
Consider the contour integral
\begin{align}
    \int_C e^{(z-1)\log(t) - t} dt.
\end{align}
Due to the fact that the integrand involves a $ \log $ function it is multivalued unless we specify a branch
cut. We will adopt a branch cut along the non-negative real axis. As long as $ \varepsilon $ 
is chosen sufficiently small ($ \varepsilon < 1 $) then Cauchy's theorem guarantees that this contour 
integral has the same value for all choices of $ \varepsilon > 0 $ because the integrand has only one
singularity at $ t = 0 $.

Now consider the contour in three pieces, $ C_{\varepsilon} $, the circle of radius $ \varepsilon $ 
centred at 0, $ (\varepsilon, \infty) $, the bottom half of the contour and $ (\infty, \varepsilon) $, 
the top half of the contour. 

As $ \varepsilon $ can be made arbitrarily small we take both straight parts of the contour to be
real but on the lower half of the contour we must replace $ \log(t) $ with $ \log(t) + 2\pi i $. 

This results in 
\begin{align}
    \int_C e^{(z-1)\log(t)-t} dt &= \int_\infty^\varepsilon e^{(z-1)\log(t) - t} dt + \int_{C_\varepsilon} 
        e^{(z-1)log(t) - t} dt + \int_{\varepsilon, \infty} e^{(z-1)\log(t) - t + 2\pi i} dt \\
    &= \int_\infty^\varepsilon e^{-t}t^{z-1}dt + \int_{C_\varepsilon} e^{-t}t^{z-1}dt +
    e^{2(z-1)\pi i} \int_\varepsilon^\infty e^{-t}t^{z-1} dt
\end{align}


It can be shown by simple application of the ML lemma, as in \cite{Podlubny1999}, that the integral along
$ C_\varepsilon $ goes to zero as $ \varepsilon \lra 0 $.

Taking $ \varepsilon \lra 0 $ in the other two integrals and rearranging gives us the result
\begin{align}
    \label{eq:gamma_contour}
    \int_0^\infty e^{-t} t^{z-1} dt = \frac{1}{e^{2 \pi i z} - 1} \int_c e^{-t}t^{x-1} dt
\end{align}
\end{proof}

We would like to exploit this result to give a simple contour integral representation of $ \frac{1}{\Gamma(z)} $.
\begin{mdframed}[innertopmargin=10pt]
\begin{lemma}
    \label{lem:contour_1_gamma_1}
    A contour integral representation of $ \frac{1}{\Gamma(z)} $ is given by
    \begin{align}
        \frac{1}{\Gamma(z)} = \frac{1}{2 \pi i} \int_{Ha} e^\tau \tau^{-z} d\tau
    \end{align}
    where the contour $ Ha $ is shown in figure \ref{fig:hankel_loop}.
\end{lemma}
\end{mdframed}
This proof follows that of Podlubny \cite{Podlubny1999}
\begin{proof}
    By taking the result of \ref{lem:contour_gamma} and substituting $ z-1 $ in \eqref{eq:gamma_contour}
    we get that
    \begin{align}
        \int_C e^{-t} t^{-z} dt = (e^{-2 \pi i z} -1) \Gamma(1-z).
    \end{align}
    Performing the substitution $ \tau = -t $ rotates the contour integral clockwise $ 180 $ degrees 
    and results in 
    \begin{align}
        \int_{C} e^{-t} t^{-z} dt = - e^{-z\pi i}\int_{Ha} e^{\tau} \tau^{-z} d\tau
    \end{align}
    and so
    \begin{align}
        \Gamma(1-z) &= \frac{1}{e^{z\pi i} - e^{-z \pi i}} \int_{Ha} e^\tau \tau^{-z} d\tau \\
            &= \frac{1}{2i \sin(\pi z)}\int_{Ha} e^\tau \tau^{-z} d\tau.
    \end{align}
    Now taking into account the well known Euler reflection formula, $ \Gamma(z)\Gamma(1-z) = \frac{\pi}{\sin(\pi z)} $, we get
    \begin{align}
        \frac{1}{\Gamma(z)} = \frac{1}{2 \pi i} \int_{Ha} e^\tau \tau^{-z} d\tau.
    \end{align} \qed
    \begin{wrapfigure}{r}{210pt}
        \begin{tikzpicture}[scale=2]
    \draw[<->] (1.5,0) -- (-1.5,0);
    \draw[<-,red] (-1.5,0.25) -- (-0.25, 0.25);
    \draw[<-, red] (-0.25, -0.25) -- (-1.5,-0.25);
    \draw[<-, red] (-0.25,0.25) arc [radius = 0.35355, start angle=135, end angle=0];
    \draw[red] (-0.25,-0.25) arc [radius = 0.35355, start angle=-135, end angle=0];
    \draw[<->] (0,1.5) -- (0,-1.5);
    \draw[densely dotted] (0,0) -- (0.24,0.24);   
    \draw (0.25, 0.15) node {$\varepsilon$};
    \draw (1.5, 0.125) node {$Re$};
    \draw (0.135, 1.5) node {$Im$};
\end{tikzpicture}

        \caption{The Hankel contour Ha}
        \label{fig:hankel_loop}
    \end{wrapfigure}
\end{proof}

We now wish to give a pair of contour integral representations of the function $ \frac{1}{\Gamma(z)} $
which we will then use in contour integrals representing $ E_{\alpha, \beta}(z) $. 
\begin{mdframed}[innertopmargin=10pt]
\begin{lemma}
    \label{lem:1_gamma_contour}
    The function $ \frac{1}{\Gamma(z)} $ can be represented by the following contour integrals 
    \begin{align}
        \label{eq:gamma_contour_1}
        \frac{1}{\Gamma(z)} = \frac{1}{2 \pi \alpha i} \int_{\gamma(\varepsilon, \mu)} 
                              e^{\zeta^{1 / \alpha}} \zeta^{(1-z-\alpha) / \alpha} d\zeta 
    \end{align}
    where
    \begin{align}
        \left( \alpha < 2, \frac{\pi \alpha}{2} < \mu < \min\{ \pi, \pi \alpha \} \right)
    \end{align}
    or where $ Re(z) > 0 $
    \begin{align}
        \label{eq:gamma_contour_2}
        \frac{1}{\Gamma(z)} = \frac{1}{4 \pi i} \int_{\gamma(\varepsilon, \pi)} e^{\zeta^{1 / 2}} \zeta^{-(z + 1) / 2} d\zeta
    \end{align}
    where the contour $ \gamma(\varepsilon, \varphi) $ is depicted in figure \ref{fig:hankel_wedge}.
\end{lemma}
\end{mdframed}
\begin{wrapfigure}{l}{210pt}
    \begin{tikzpicture}[scale=2]
    \draw[<->] (1.5,0) -- (-1.5,0);
    \draw[<-,red] (-1.5,1.5) -- (-0.25, 0.25);
    \draw[<-, red] (-0.25, -0.25) -- (-1.5,-1.5);
    \draw[<-, red] (-0.25,0.25) arc [radius = 0.35355, start angle=135, end angle=0];
    \draw[red] (-0.25,-0.25) arc [radius = 0.35355, start angle=-135, end angle=0];
    \draw[<->] (0,1.5) -- (0,-1.5);
    \draw[densely dotted] (0,0) -- (0.24,0.24);   
    \draw (0.25, 0.15) node {$\varepsilon$};
    \draw (1.5, 0.125) node {$Re$};
    \draw (0.135, 1.5) node {$Im$};
    \draw (-1, 0.5) node {$ G^-(\varepsilon, \varphi) $};
    \draw (1, 1) node {$G^+(\varepsilon, \varphi) $};
    \draw[<-, densely dotted] (-0.5, 0.5) arc [radius = 0.7071 , start angle=135, end angle=0];
    \draw (0.5, 0.6) node {$\varphi$};
\end{tikzpicture}

    \caption{The Hankel countour $ \gamma(\varepsilon, \varphi) $}
    \label{fig:hankel_wedge}
\end{wrapfigure}
\begin{wrapfigure}{r}{10pt}
    \begin{tikzpicture}[scale=2]
\draw[<->] (1.5,0) -- (-1.5,0);
\draw[<-,red] (-0.66,0.66) -- (-0.25, 0.25);
\draw[<-, red] (-0.25, -0.25) -- (-0.66,-0.66);
\draw[<-, red] (-0.25,0.25) arc [radius = 0.35355, start angle=135, end angle=0];
\draw[red] (-0.25,-0.25) arc [radius = 0.35355, start angle=-135, end angle=0];
\draw[<->] (0,1.5) -- (0,-1.5);
\draw[densely dotted] (0,0) -- (0.24,0.24);
\draw (0.25, 0.15) node {$\varepsilon$};
\draw (0.75, -0.75) node {$R$};
\draw (1.5, 0.125) node {$Re$};
\draw (0.135, 1.5) node {$Im$};
\draw[<-, red] (-0.25,-0.25) arc [radius = 0.35355, start angle=-135, end angle=-170];
\draw[<-, red] (-0.25,0.25) arc [radius = 0.35355, start angle=135, end angle=170];
\draw[red] (-1,0.05) arc [radius = 1, start angle=170, end angle=130];
\draw[red] (-1,-0.05) arc [radius = 1, start angle=-170, end angle=-130];
\draw[<-, red] (-1,0.05) -- (-0.35255, 0.05);
\draw[->, red] (-1,-0.05) -- (-0.35255, -0.05);
\draw[densely dotted] (-0.66,-0.66) arc [radius = 0.86, start angle=-130, end angle=130];
\draw[->, densely dotted] (0,0) -- (0.55, -0.55);
\end{tikzpicture}
    \caption{ Integration contour for $ \gamma(\varepsilon, \varphi) $}
    \label{fig:hankel_wedge_2}
\end{wrapfigure}
This proof follows that of Podlubny \cite{Podlubny1999}.
\begin{proof}

Firstly we show that we can rewrite the contour integral in lemma \ref{lem:contour_1_gamma_1} can be rewritten as a contour
integral about $ \gamma(\varepsilon, \varphi) $ where $ \gamma(\varepsilon, \varphi) $ is shown in figure \ref{fig:hankel_wedge}. We restrict $ \varphi $ to $ \frac{\pi}{2} < \varphi \pi $. 

Now lets consider the contour integral diagram in figure \ref{fig:hankel_wedge_2} and note that as $ f(\tau) := e^\tau t^{-z} $ does not have any singularities we have
\begin{align}
    \int_{A^+ B^+ C^+ D^+} f(\tau) d\tau = \int_{A^-D^-C^-B^-} f(\tau) = 0.
\end{align}

Considering just the arc $ A^+B^+ $ we note that $ |\tau| = R $ and hence
\begin{align}
    |f(\tau)| &= |e^\tau \tau^{-z}| = e^{R\cos(\arg \tau) - Re(z) \log(R) + Im(z) \arg \tau} \\
      &\leq e^{-R\cos(\pi - \phi) - x \log(R) + 2 \pi y} \\
      &\sim \frac{1}{R}e^{-R}
\end{align}

and thus by application of the ML lemma we have
\begin{align}
    \lim_{R \lra \infty} \int_{A^+}^{B^+} f(\tau) d\tau = 0.
\end{align}


A completely symmetric argument yields
\begin{align}
    \lim_{R \lra \infty} \int_{B^-}^{A^-} f(\tau) d\tau = 0
\end{align}
and so we can write
\begin{align}
    \int_{C^+}^{B_\infty^+} f(\tau)d\tau = \int_{C^+}^{D^+} f(\tau)d\tau + \int_{D^+}^{\infty^+} f(\tau)d\tau
\end{align}
and
\begin{align}
    \int_{B_\infty^-}^{C^-} f(\tau)d\tau = \int_{\infty^-}^{D^-} f(\tau)d\tau + \int_{D^-}^{C^-} f(\tau)d\tau.
\end{align}
Then
\begin{align}
    \int_{Ha} f(\tau) d\tau = \left( \int_{B_\infty^-}^{C^-} f(\tau)d\tau + \int_{C^-}^{C^+} f(\tau)d\tau +  \int_{C^+}^{B_\infty^+} f(\tau)d\tau \right) = \int_{\gamma(\varepsilon, \varphi)}e^\tau \tau^-z d\tau.
\end{align}
To get the final results, we perform the substitution $ \tau = \zeta^{1/\alpha} $ for $ \alpha < 2 $
which yields

\begin{align}
    \frac{1}{\Gamma(z)} = \frac{1}{2 \pi \alpha i} \int_{\gamma(\varepsilon, \mu)} 
			  e^{\zeta^{1 / \alpha}} \zeta^{(1-z-\alpha) / \alpha} d\zeta 
\end{align}

so long as $ \alpha < 2 $ and $ \frac{\pi \alpha}{2} < \mu < \min\{ \pi, \pi \alpha \} $.

If, however, we perform the substitution $ \tau = \sqrt{\zeta} $ and integrate over the contour $ \gamma(\varepsilon, \frac{\pi}{2} $ we get

\begin{align}
    \frac{1}{\Gamma(z)} = \frac{1}{4 \pi i} \int_{\gamma(\varepsilon, \pi)} e^{\zeta^{1 / 2}} \zeta^{-(z + 1) / 2} d\zeta
\end{align}
for $ Re(z) > 0 $.

\end{proof}

\subsection{Mittag-Leffler Function}

Just like the exponential function has an invariance property under differentiation, we would like a function that has a similar property for fractional differentiation. With that in mind we define the Mittag-Leffler function.
\begin{mdframed}[innertopmargin=10pt]
\begin{definition}
    Define the two parameter Mittag-Leffler function as
    \begin{align}
        E_{\alpha, \beta}(z) = \sum_{k=0}^\infty \frac{z^k}{\Gamma(\alpha k + \beta)}.
    \end{align}
\end{definition}
\end{mdframed}
The reason we defined it as the \emph{two paramter} Mittag-Leffler function is because some authors use the phrase Mittag-Leffler function to refer to a one parameter form, $ E_{\alpha}(z) = E_{\alpha, 1}(z) $. The two parameter version of this function is in common use and some more general results can be shown about the two parameter form. For the rest of our discussions we will use the phrase Mittag-Leffler function to refer to the two paramter version.

The first thing that any reader should notice about the Mittag-Leffer function is that it is an immediate generalisation of the exponential function, with $ E_{1, 1}(z) = \exp(z) $. 

It should be clear to see that at least for $ \alpha, \beta \in \mathbb{R} $ this series uniformly converges on compact subsets in much that same way that that the series for the exponential function converges. This fact is important as it will allow us to interchange sums and integrals in several of the results which follow.

Interestingly a considerable number of other functions can be expressed in terms of Mittag-Leffler functions. By setting $ \alpha = 0, \beta = 1 $ we can easily see that we have a geometric series and so when $ |z| < 1 $ we have
\begin{align}
    E_{0,1}(z) = \frac{1}{z}.
\end{align}

Setting $ \alpha = 2, \beta = 1 $ we get
\begin{align}
    E_{2,1}(z) &= \sum_{k=0}^\infty \frac{z^k}{\Gamma(2k + 1)} \\
            &= \sum_{k=0}^\infty \frac{z^k}{(2k)!}
\end{align}
Now also notice that Taylor expansion of $ \cosh(\sqrt{z}) $ about 0 is
\begin{align}
    \cosh(\sqrt{z}) = \sum_{k=0}^\infty \frac{z^k}{(2k)!}
\end{align}
and so $ E_{2,1}(z) = \cosh(\sqrt{z}) $. 

There are considerably more relationships. We refer the interested reader to \cite{Podlubny1999} or \cite{Samko1993} for a more extensive listing of these relationships.

We will now calculate the Laplace transform of some special cases of the Mittag-Leffler function. These will prove useful for latter results. 
\begin{mdframed}[innertopmargin=10pt]
\begin{lemma}
    \label{lem:lap_mit}
    The Laplace transform of $ E_{\alpha, 1}(\gamma z^\alpha ) $ is given by
    \begin{align}
        \mathcal{L} \left\{ E_{\alpha, 1}(\gamma z^\alpha ) \right\} &= \frac{s^{\alpha - 1}}{s^\alpha - \gamma}.
    \end{align}
\end{lemma}
\end{mdframed}
\begin{proof}
    We have that
    \begin{align}
        \mathcal{L}\left\{ E_{\alpha, 1}(\gamma z^\alpha)\right\} &= \int_0^\infty e^{-st} \sum_{k=0}^\infty \frac{(\gamma t^\alpha)^k}{\Gamma(\alpha k + 1)} dt \\
        &= \sum_{k=0}^\infty \int_0^\infty \frac{e^{-st} (\gamma t^\alpha)^k}{\Gamma(\alpha k + 1)} dt \\
        \circledast &= \sum_{k=0}^\infty \frac{\gamma^k}{\Gamma(\alpha k + 1)}\int_0^\infty e^{-st}t^{\alpha k} dt.
    \end{align}
    With a change of variables $ x = st $ we get that
    \begin{align}
        \circledast &= \sum_{k=0}^\infty \frac{\gamma^k s^{-(\alpha k+1)}}{\Gamma(\alpha k + 1)} \int_0^\infty e^{-x} x^\alpha kdx \\
        &= \sum_{k=0}^\infty \frac{\gamma^k s^{-(\alpha k + 1)\Gamma(\alpha k + 1)}}{\Gamma(\alpha k + 1)} \\
        &= \sum_{k=0}^\infty \gamma^k s^{-(\alpha k+1)} \\
        &= \frac{s^{\alpha - 1}}{s^\alpha - \gamma}.
    \end{align}
    So we have that
    \begin{align}
        \mathcal{L}\left\{ E_{\alpha,1}( \gamma t^\alpha ) \right\} = \frac{s^{\alpha - 1}}{s^\alpha - \gamma}
    \end{align}
\end{proof}


The following lemma is essentially a generalisation of lemma \ref{lem:lap_mit} but it is important as we will make use of it in future results.
\begin{mdframed}[innertopmargin=10pt]
\begin{lemma}
	\label{lem:lap_mit_2}
	The Laplace transform of $ t^{\alpha m + \gamma - 1}E_{\alpha, \gamma}^{(m)}(t) $ is given by
	\begin{align}
		\laplace{ t^{\alpha m + \gamma - 1}E_{\alpha,\gamma}^{(m)} (\beta t^\alpha)} = \frac{m!s^{\alpha-\gamma}}{(s^\alpha - \beta)^{m+1}}
	\end{align}
\end{lemma}
\end{mdframed}
\begin{proof}
	Firstly see that
	\begin{align}
		E_{\alpha,\gamma}^{(m)}(t) &= \sum_{k=m}^{\infty} \frac{\frac{k!}{(k-m)!}t^{k-m}}{\gamma(\alpha k + \gamma)} \\
			&= \sum_{k=0}^{\infty} \frac{(k+m)!t^k}{k!\Gamma(\alpha k + \gamma)}
	\end{align}
	so we have that
	\begin{align}
		E_{\alpha, \gamma}^{(m)}(\beta t^\alpha) &= \sum_{k=0}^{\infty} \frac{(k+m)!t^{\alpha k} \beta^k}{k! \Gamma(\alpha (k+m) + \gamma)}.
	\end{align}
	We can then write that
	\begin{align}
		\laplace{t^{\alpha m + \gamma - 1}E_{\alpha, \gamma}^{(m)}(t)} &= \int_0^\infty t^{\alpha m + \gamma - 1}  \sum_{k=0}^{\infty} \frac{(k+m)!t^{\alpha k} \beta^k}{k! \Gamma(\alpha (k+m) + \gamma)} \\
			&= \sum_{k=0}^\infty \frac{\beta^k (k+m)!}{\Gamma(\alpha(k+m) + \gamma) k!} \underbrace{\int_0^\infty e^{-st} t^{\alpha (k+m) + \gamma - 1}dt}_{\circledast}.
	\end{align}
	Considering just $ \circledast $ and performing the substitution $ x = st $ we get that 
	\begin{align}
		\circledast &= s^{-\alpha(k+m) - \gamma} \int_0^\infty e^{-x} x^{\alpha (k+m) + \gamma - 1} dx \\
			&= s^{-\alpha(k+m) - \gamma} \Gamma(\alpha(k+m) + \gamma)
	\end{align}
	and so 
	\begin{align}
		\laplace{t^{\alpha m + \gamma - 1}E_{\alpha, \gamma}^{(m)}(t)} = s^{-\alpha m - \gamma}\sum_{k=0}^\infty \left(\frac{\beta}{s^\alpha}\right)^k\frac{(k+m)!}{k!} .
	\end{align}
	Now by the derivative rule for geometric series we get
	\begin{align}
		\sum_{k=0}^\infty \left(\frac{\beta}{s^\alpha}\right)^k\frac{(k+m)!}{k!} &= \frac{m!}{(1-\frac{\beta}{s^\alpha})^{m+1}} \\
			&= \frac{s^{\alpha(m+1)} m!}{(s^\alpha - \beta)^{m+1}}
	\end{align}
	and so 
	\begin{align}
		\laplace{t^{\alpha m + \gamma - 1}E_{\alpha, \gamma}^{(m)}(t)} = \frac{m!s^{\alpha-\gamma}}{(s^\alpha - \beta)^{m+1}}.
	\end{align}
\end{proof}

We will use this result in solving fractional differential equations, along with proving some invariance results about the Mittag-Leffler function.

It should be clear that
\begin{align}
    \frac{d}{dz}E_{\alpha, \beta}(z) = \sum_{k=0}^\infty \frac{(k+1)z^k}{\Gamma(\alpha(k+1) + \beta)}
\end{align}
and further that
\begin{align}
    \label{eq:mit_int_der}
    \frac{d^n}{dz^n} E_{\alpha, \beta}(z) = \sum_{k=0}^\infty \frac{z^k}{\Gamma(\alpha(k+n)+\beta)}\prod_{j=1}^n(k+j).
\end{align}
So the Mittag-Leffler function retains a \emph{similar} structure under differentiation. Obviously if $ \alpha = 1 $ and $ \beta = 1 $ this just reduces to the invariance of the exponential function under the differentiation map. It turns out that there are a whole collection of near-invariance and invariance properties of the Mittag-Leffler function. For a very general result we have the following lemma from \cite{Podlubny1999}.
\begin{mdframed}[innertopmargin=10pt]
\begin{lemma}
    \label{lem:rld_mittag}
    We have that the fractional derivative of $ z^{\alpha k+\beta-1}E_{\alpha,\beta}^{(k)}(\lambda z^\alpha) $ is given by 
    \begin{align}
        \prescript{}{0}{\mathcal{D}}^\gamma z^{\alpha k+\beta-1}E_{\alpha,\beta}^{(k)}(\lambda z^\alpha) = z^{\alpha k + \beta - \gamma - 1 } E_{\alpha,\beta-\gamma}^{(k)}(\lambda z^\alpha)
    \end{align}
\end{lemma}
\end{mdframed}
\begin{proof}
    We could prove this result by using lemma \ref{lem:rld_power} and equation \eqref{eq:mit_int_der} but that turns out to be very convoluted. A simpler method is to get this result in Laplace space.
    Using lemma \ref{lem:lap_rld} we have that 
    \begin{align}
        \mathcal{L}\left\{ \prescript{}{0}{\mathcal{D}}^\gamma z^{\alpha k+\beta-1}E_{\alpha,\beta}^{(k)}(\lambda z^\alpha) \right\} = s^\gamma \mathcal{L}\left\{z^{\alpha k+\beta-1}E_{\alpha,\beta}^{(k)}(\lambda z^\alpha) \right\}
    \end{align}
    and by using the result of lemma \ref{lem:lap_mit_2} we have
    \begin{align}
        \mathcal{L}\left\{z^{\alpha k+\beta-1}E_{\alpha,\beta}^{(k)}(\lambda z^\alpha) \right\} = \frac{k!s^{\alpha - \beta}}{(s^\alpha - \lambda)^{k+1}}
    \end{align}
    which means that
    \begin{align}
        \mathcal{L}\left\{ \prescript{}{0}{\mathcal{D}}^\gamma z^{\alpha k+\beta-1}E_{\alpha,\beta}^{(k)}(\lambda z^\alpha) \right\} = \frac{k!s^{\alpha - (\beta - \gamma)}}{(s^\alpha - \lambda)^{k+1}}
    \end{align}
    and by inverting the Laplace transform by using the result of lemma \ref{lem:lap_mit_2} again we get that
    \begin{align}
        \prescript{}{0}{\mathcal{D}}^\gamma z^{\alpha k+\beta-1}E_{\alpha,\beta}^{(k)}(\lambda z^\alpha) = z^{\alpha k + \beta - \gamma - 1 } E_{\alpha,\beta-\gamma}^{(k)}(\lambda z^\alpha).
    \end{align}
\end{proof}
We can already use this result to construction fractional differential equations for which some form of the Mittag-Leffler function would the the solution to. 

It's not hard to see that by setting $ \gamma = \alpha = n $ with $ n \in \Ntrl $ and selecting $ \beta \in \{ 0 , 1, \ldots, m\} $ we have that
\begin{align}
    \frac{d^n}{dz^n} z^{\beta - 1} E_{n,\beta}(z^n) = z^{\beta - 1}E_{n,\beta}(z^n)
\end{align}
which generalises the derivative invariance property of exponential 
functions.

\subsection{Wright Function}

The Wright function is another two parameter special function. It again has a definition is at least in appearance similar to the exponential function. 

\begin{mdframed}[innertopmargin=10pt]
\begin{definition}
Define the Wright function as 
\begin{align}
	W_{\rho, \mu}(z) = \sum_{k=0}^\infty \frac{z^k}{k!\Gamma(\rho k + \mu)}
\end{align}
\end{definition}
\end{mdframed}

\clearpage
